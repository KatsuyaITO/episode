¥Chapter{Counterexamples in Topology 傑作撰(宮本)}
¥Section{¥S 1.概説}
¥subsubsection*{反例の本}
数学は定理と証明を繰り返すだけではなかなか身につきにくく,適切な{¥bf 反例}を考えることで概念の有用性を理解できることが多い,というのは皆さん経験したことだと思います.歴史的にも,ワイエルシュトラスが連続かつ至るところ微分不可能な「病理的関数」を発見したことで,それまで「連続関数は大体のところで微分可能なんじゃないの〜?」と信じられていた雰囲気がぶち壊され,解析学が$¥varepsilon - ¥delta$論法による厳密化へと進展した.このようなことが度々あります.証明によって世界を広げていくというのが数学の基本だと思いますが,反例が「ある」という事実もまた心強いものです.¥par
今回は位相空間論の反例集である``¥textbf{Counterexamples in Topology}''  (Lynn Steen and J. Arthur Seebach, Jr)の内容から幾つか参考になりそうな例を紹介してみたいと思います.
¥subsubsection*{内容紹介}
本書は位相空間論の反例を大量に収録したある意味「病的な」本といえるかもしれませんが,学習の上で参照するにはうってつけの本だと思います.本書最大の特徴は,豊富なチャートにあります.例えば,正則であって正規でない位相空間を探そうと思ったら,分離公理について書かれた19ページの表を見て,反例90番が該当するということがすぐに分かります.定義間の主従関係が反例付きで載っていますし,巻末には各例について,どの公理を満たし,どの公理を満たさないかが網羅された表が掲載されており,検索に供するようになっています.¥par
第一部では位相の様々な定義(分離公理,コンパクト性,連結性,距離空間)を簡単に復習することができます.第二部がメインコンテンツの反例集で,143個の例が掲載されています.第三部は「位相空間はいつ距離空間になるか?」という距離化定理にまつわるサーベイとなっています(ここにも反例が載っています).付録の第四部では前述の網羅された表や演習問題,参考文献等が掲載されています.

¥Section{¥S 2.分離公理に関する反例}
¥subsubsection*{正則・正規}
まずは定義を簡単に復習しましょう.
¥begin{defi}[正則空間]
位相空間Xが{¥bf 正則}であるとは,以下の2つの条件をみたすことを言う.
¥begin{enumerate}
¥item $X$の任意の$1$点集合は閉集合である.
¥item $X$の任意の閉集合$A$と,$A$に含まれない任意の点$x$は開集合で分離される.すなわち,開集合$U$,$V$が存在して,$A¥subset U$,$x¥in V$かつ$U¥cap V=¥varnothing$が成立する.
¥end{enumerate}
¥end{defi}
¥begin{defi}[正規空間]
位相空間$X$が{¥bf 正規}であるとは,以下の2つの条件をみたすことを言う.
¥begin{enumerate}
¥item $X$の任意の$1$点集合は閉集合である.
¥item $X$の任意の交わらない閉集合の組$A,B$は開集合で分離される.すなわち,開集合$U$,$V$が存在して,$A¥subset U$,$B¥subset V$かつ$U¥cap V=¥varnothing$が成立する.
¥end{enumerate}
¥end{defi}
定義から明らかですが,正規性からは正則性が従います.また,距離空間は正規ですから,普通のユークリッド空間$¥mathbb{R}^n$は正規になります.正則であって,正規でない空間とはどのようなものでしょうか?
¥begin{ex}[Sorgenfrey空間]
$¥mathbb{R}$の開集合の基底を$U=[a,b)¥ (a,b¥in¥mathbb{R})$とした空間を$S$とする(Sorgenfrey直線).Sorgenfrey空間Xを$X=S¥times S$で定義する.
¥end{ex}
$X$での典型的な開集合は$[a,b)¥times[c,d)$という形をしており,これが$X$の開集合の基底になっています.ここで部分集合$¥Delta=¥{(x,-x)¥mid x¥in¥mathbb{R}¥}$を考えてみると,これは閉集合です.$¥Delta$内の任意の点$x$に対して$¥{ x ¥}$は誘導位相の下で開集合になるので,$¥Delta$に誘導される位相は離散位相となり,$¥Delta_1=¥{(x,-x)¥mid x¥in¥mathbb{Q}¥}$,$¥Delta_2=¥Delta¥setminus¥Delta_1$とすると,これらは$X$の閉集合です.ところが$¥Delta_1$を含む開集合と$¥Delta_2$を含む開集合は必ず交わってしまうことが示せるので$X$は正規ではありません.¥par
また,Sorgenfrey直線$S$は正則かつ正規であり,$2$つの正則空間の直積は正則なので$X$は正則です.このことから$X$が「正規空間の直積は正規」という命題の反例ということも分かりますね.¥par
 ¥par
正規空間は部分空間をとるという操作とも相性が悪く,正規空間の部分空間は正規とは限りません.
¥begin{defi}[全部分正規]
任意の部分空間が正規となるような正規空間$X$を{¥bf 全部分正規}であるという.
¥end{defi}
全部分正規でないような正規空間の例は,集合論的に少しむずかしい操作をして構成します.
¥begin{defi}
$¥omega$を最小の超限順序数(自然数の集合$¥mathbb{N}$に対応する順序数¥footnote{選択公理を認めさせてください.}),$¥Omega$を最小の非可算順序数とする.順序数には濃度の比較により全順序が入るので順序数$¥alpha,¥beta$に対して$[¥alpha,¥beta]=¥{x¥mid ¥alpha¥leq x¥leq ¥beta¥}$とする.
¥end{defi}
¥begin{ex}[Tychonoffの板]
$[0,¥omega]$,$[0,¥Omega]$にそれぞれ区間から定まる位相を入れ,$T=[0,¥omega]¥times[0,¥Omega]$とする..
¥end{ex}
順序数$¥alpha$に対して$[0,¥alpha]$に区間位相を入れた空間はコンパクトハウスドルフになり,その直積である$T$もコンパクトハウスドルフですから,コンパクトハウスドルフ空間が正規なことより$T$は正規空間です¥footnote{「$[0,¥alpha]$が正規だから,正規空間の直積である$T$も正規である」というのはダメですよね.}.¥par
ところが,$T$から$(¥omega,¥Omega)$という一点を除いた空間$T_{¥infty}$は正規で無くなってしまうのです.これは$T_{¥infty}$の部分空間$A=¥{(¥omega,¥alpha)¥mid 0¥leq ¥alpha<¥Omega¥}$,$B=¥{ (n,¥Omega)¥mid 0¥leq n<¥omega ¥}$が分離できない$2$つの閉集合であることを示すことで分かります.

¥subsubsection*{距離づけ可能定理と第二可算性}
位相空間というのは点同士の「近さ」が定義されている空間ですから,ある空間に「距離¥footnote{集合$X$上の距離とは以下の3つの公理を満たす$X$上の非負実数値関数$d$のことを言います.(i)$d(x,y)=0¥Leftrightarrow x=y$,(ii)$d(x,y)=d(y,x)$,(iii)$d(x,y)+d(y,z)¥geq d(x,z)$}」が定まっていれば,その空間は自動的に位相空間になります.これを{¥bf 距離空間}といいます.では逆に,位相空間を与えたとき,その位相構造と両立する距離が存在する({¥bf 距離づけ可能である})のはどんなときか?ということが気になります.この問は距離化の理論(Metrization theory)と言われる一般位相空間論の大テーマに発展しており,本書でもこの話題を扱った章に詳しく書かれています.¥par
距離空間は正規ですから,正規空間になにか$+¥alpha$すれば距離空間になるであろうと思うと,次のような定理があります.
¥begin{thm}[Urysohnの定理]
正規かつ第二可算公理¥footnote{位相空間$X$が可算個の基底をもつとき,$X$を第二可算であるといいます.}をみたす位相空間$X$は距離づけ可能である.
¥end{thm}
証明の方針は以下のようになります.
¥begin{enumerate}
¥item 距離づけ可能な位相空間の直積は距離づけ可能になる(可算個の直積でもOK).
¥item $X$の可算な開集合の基底を$¥mathfrak{B}$として,$¥mathfrak{M}=¥{(U,V)¥mid ¥bar{U}¥subset V ¥}¥subset ¥mathfrak{B}¥times¥mathfrak{B}$とする(この集合の濃度は勿論可算).
¥item 正規性よりUrysohnの補題を使って$¥mathfrak{M}$の各元に対して$¥bar{U}$上で$0$,$V$の補集合上で$1$をとる連続関数を作る.
¥item これによって$X$から$[0,1]^¥mathbb{N}$へ連続写像が定まる.これが中への同相写像であることを示す.
¥item $I$は距離づけ可能なので1.より$I^¥mathbb{N}$も距離づけ可能.その部分空間(とみなせる)$X$も距離づけ可能である.
¥end{enumerate}
実は,正則$+$第二可算は正規性を導くので,Urysohnの定理は通常「正則第二可算ならば距離づけ可能」と書かれます.Sorgenfrey直線は正規ですが,第二可算でなく距離づけ可能ではありません.また,第二可算でない距離空間が存在する¥footnote{簡単な反例があります,よね?}ので,この定理は必要十分条件を与えているわけではないことにも注意しないといけません.距離づけ可能定理は他にも色々存在します.¥par
 ¥par
多様体は局所的にユークリッド空間と見れるハウスドルフ空間なので,局所コンパクトハウスドルフ,したがって正則となります.ふつう「多様体」というときは第二可算公理を仮定するので,多様体は距離づけ可能であることが分かります.第二可算性を満たさないような異常な「多様体」の例として「{¥bf 長い直線}」は有名です.
¥begin{ex}[長い直線]
$¥Omega$を最小の非可算順序数として,$¥Omega$と区間$[0,1)$の直積集合$L$に辞書式順序から定まる順序位相を入れる.これを長い半直線という.長い半直線の端をつなげて出来るのが長い直線である.
¥end{ex}
普通の半直線$[0,¥infty)$は$¥omega¥times[0,1)$($¥omega$は可算順序数)と思えます.順序数で非可算なものを使ってやることで「長い」感を出しているわけですね.直感的に考えて長い直線は「長すぎる」ために第二可算公理を満たさない.距離づけ可能でない多様体となっているわけです.
¥Section{¥S 3.コンパクト性に関する反例}
コンパクト性にまつわる概念にもいろいろあるのですが,ここでは最も基本的な{¥bf コンパクト性}と{¥bf 点列コンパクト性}の関係を見て行きましょう.
¥subsubsection{コンパクト vs 点列コンパクト}
¥begin{defi}
位相空間Xが
¥begin{itemize}
¥item {¥bf コンパクト}であるとは,$X$の任意の開被覆に対して有限部分被覆が存在することをいう.
¥item {¥bf 点列コンパクト}であるとは,$X$内の任意の点列が収束する部分列を持つことをいう.
¥end{itemize}
¥end{defi}
この2つの定義は距離空間では同値になりますが,そうでないときは互いに何の関係もない性質です.
¥begin{ex}[$I^I$]
$I=[0,1]$(単位閉区間)とし,$I^I=¥displaystyle ¥prod_{i¥in I}I_i$(ただし$I_i=I$)とする.位相は$I$にユークリッド位相を入れたものの直積位相とする.
¥end{ex}
Tychonoffの定理¥footnote{コンパクト位相空間の任意個の直積空間が再びコンパクトになるという定理.}より$I^I$はコンパクトですが,点列コンパクトではありません¥footnote{点列コンパクト性は可算個の直積に対しては保存されます.}.$I^I$は$I$から$I$への写像全体の集合と見れて,$I^I$内の点列$¥{a_k¥}$が収束する部分列をもつということは,対応する写像の列$¥{f_k¥}$が部分列$¥{f_{k_i}¥}$をもって各点$x¥in I$で$¥{ f_{k_i}(x) ¥}$が収束するということです.ここでたとえば$f_k(x)=(x¥text{を2進展開したときの小数点第}k¥text{位)}$とすると,どんな部分列$¥{f_{k_i}¥}$に対しても$¥{f_{k_i}(p)¥}$が収束しないような小数$p$が存在してしまいます.¥par 
 ¥par
点列コンパクトだが,コンパクトでない例としては前節で定義した$[0,¥Omega]$から$¥{¥Omega¥}$を取り除いた$[0,¥Omega)$があります.$[0,¥Omega]$はコンパクト¥footnote{順序数の性質からわかります.}なのですが,$[0,¥Omega)$はコンパクトではありません¥footnote{$[0,¥alpha)$のような開集合たちで覆うと有限部分被覆が取れません.}.また,$[0,¥Omega]$のコンパクト性から$[0,¥Omega)$内の任意の点列は$[0,¥Omega]$内に集積点をもちますが,$¥Omega$が集積点になることはないので,$[0,¥Omega)$は点列コンパクトであることがわかります.

¥Section{¥S 4.連結性に関する反例}
連結性とは位相空間が「つながっている(ちぎれていない)」ことを表す概念です.本書では不連結な概念についても多く語られていますが,ここでは基本的な連結性に関する2つの概念について見ていきましょう.
¥begin{defi}
位相空間$X$が
¥begin{itemize}
¥item 連結であるとは,2つの開集合$U,V$が$U¥cup V=X$,$U¥cap V=¥varnothing$をみたすならば,$U=X$または$V=X$をみたすことをいう.
¥item 弧状連結であるとは,$X$上の任意の$2$点$x,y$に対して,単位閉区間$[0,1]$からの連続写像$¥phi$が存在して$¥phi (0)=x$,$¥phi (1)=y$をみたすことをいう.
¥end{itemize}
¥end{defi}
連結性に関する反例といえば,弧状連結$¥Rightarrow$連結であるが逆は成立しないということの説明として「{¥bf 位相幾何学者の正弦曲線}」を持ち出すのが常です¥footnote{勿論本書にも収録されています,というか表紙に書いてある.}が,あまりにも有名なので割愛します¥footnote{そういえば「長い直線」も連結だが弧状連結ではないという例になっていますね.}.
¥subsubsection*{直積と連結性}
Tychonoffの定理(コンパクト集合の任意個の直積はコンパクト)と同様に,連結集合の任意個の直積も連結な位相空間です.ところで位相空間の直積をとる際,直積位相というのは各射影$¥rm{pr_{¥alpha}}¥colon¥displaystyle ¥prod_{¥lambda¥in A} U_¥lambda¥to U_¥alpha$がすべて連続となるような「最弱の」位相というように定めました.開集合を「各空間の開集合の直積」とする位相とは異なる(コッチのほうが開集合の数が多い)わけですが,このようにすると変なことが起こります¥footnote{Tychonoffの定理も当然不成立になってしまいます.}.
¥begin{ex}[Box product topology]
$B=¥mathbb{R}^¥omega$を無限数列全体の集合とし,$B$の開集合を$¥displaystyle ¥prod_{i=1}^{¥infty} U_i$($U_i$は$¥mathbb{R}$にユークリッド位相を入れたときの開集合)とする.
¥end{ex}
$B$は連結ではありません.なぜなら有界な無限数列全体の集合と非有界な無限数列全体の集合がともに開集合になってしまうからです.

¥Section{¥S 5.おわりに}
今回紹介した反例は本書の反例のごくごく一部です.これらの他にも面白い反例が沢山あるので是非手にとって読んでみてください.本書は値段も1800円ほどと(他の数学書と比べて)安いですし,位相空間論の(日本語だと特に)ややこしい定義が英語でスッキリとまとめられているので,日本語以外の数学書を初めて読む際などにも参考になると思います(反例の紹介部分の書き方は読みにくい感じですが……).¥par
Counterexampleシリーズはこの本がきっかけになってAnalysisやProbabilityも刊行されているようです.Group theoryとかCommutative Algebraとかがあればバカ売れだと思うんですけど,どうでしょうか.誰か書きませんか?
