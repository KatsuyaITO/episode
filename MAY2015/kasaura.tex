\Chapter[あなたにも(間違いが)分かるフェルマーの定理不完全証明(笠浦)]{あなたにも(間違いが)分かる\\フェルマーの定理不完全証明(笠浦)}
%\Section{あなたにも(間違いが)分かるフェルマーの定理不完全証明(笠浦)}
\Section{\S 0.はじめに}

厳密な論証のみを主張の根拠とする数学の分野においては,「トンデモ」とよばれるものは存在しないと思われがちだが,実はそうでは無い.

長年数学雑誌の編集に携わってきた亀井哲治郎氏によると,「フェルマーの定理」や「四色問題」といった有名な数学の問題を「解いた」という投書が後を絶たないのだという.当然それらの「証明」が数学雑誌に掲載されることはないため,彼らがいかなる誤りを犯していたのかは知るよしもない.しかし中には,自身の「証明」の出版にまでこぎつけたアマチュア数学者もいる.

この記事ではそうした例の一つを紹介する.

あるとき私は,東大駒場図書館の数学書の棚に,数論の教科書に交じって「あなたも解けるフェルマーの定理完全証明」という本が置かれているのを見つけた.

フェルマーの定理とは,$n\geq 3$のとき方程式
\[a^n+b^n=c^n\]
は自然数解を持たないという主張である.

17世紀の数学者フェルマーが本の余白にこの式を書き込んで以後,三百年以上もの間,多くの数学者が証明に取り組み,ついに1995年にアンドリュー・ワイルズによって完全な証明がなされた.(証明までの経緯はサイモン・シンによるドキュメンタリー「フェルマーの最終定理」に詳しく書かれている.)

ワイルズによる証明は極めて専門的かつ長大であるので,「あなたにも解ける」とはどういうことかと中身を覗いてみたところ,案の定「アレな本」であった.こんな本を開架においておく駒場図書館もどうかと思ったが,これはきっと「情報の正しさを自分で判断するリテラシーを身につけてほしい」という司書さんからのメッセージだと判断し,「証明」の検証を行うことにした.

\Section{\S 1.証明は驚くほどたやすかった}

小野田襄二著「あなたにも解けるフェルマーの定理完全証明」(めいけい出版 2003年)は300ページほどの本だが,6部構成になっており,定理の証明自体は最初の20ページほどを占める第1部「フェルマーの定理の証明I」で済んでいる.残りの第2部~第6部はその解説や別証明に充てられている.第1部の扉には(証明は驚くほどたやすかった)と書かれており,なるほど人類を三百年以上も悩ましてきた難問がなんの前準備もなく20ページで解けるならば驚くべきことだろう.

第1部の扉の文言をさらに引用すると,

\begin{quotation}
ビックリ玉手箱の蓋を開けたら【証明の王子】がささやいた\\

2003年4月17日のことでした.【証明の王子様】がささやいてくれたのです.それはそれは,驚くほど簡単で,喉から手が出るほど欲していたものでした.【証明の王子様】のしめくくりのことばを紹介します.
\begin{center}
\textbf{【人間の英知】のために人類に贈り物を捧げよう.}
\end{center}
\end{quotation}

ユーモアのつもりなのかもしれないが,完全に危ない人である.\textbf{玉手箱の蓋を開けてはいけない.}

それでは小野田氏による証明を見ていこう.なお,この本の文章はところどころわかりにくいところがあり,わたしが読み取れた範囲で論理を組み立てなおしている部分があることをお断りしておく.変数名や括弧の付け方は小野田氏のオリジナルに従った.

まずフェルマーの方程式
\[a^n+b^n=c^n\]
の両辺を$a,b,c$の最大公約数で割り,$a,b,c$はそれぞれ互いに素としてよいことを示している.ここまでは普通である.つぎに「$n=3$の証明によって,$n\geq 3$のすべてを証明してくれる」と主張して$n=3$の場合の証明に移っている.


\Subsubsection{$n=3$のときの証明 矛盾とその解消}

小野田氏はまず,
\[b^3=c^3-a^3=P\]
とおき,この式を\textcircled{\scriptsize 1}としている.さらに
\[c^3-a^3=b\cdot b^2\]
と変形し,この式を\textcircled{\scriptsize 2}としている.さきほどとほとんど変わらない式なのでわざわざ\textcircled{\scriptsize 2}とおく必要もない気がするが,まあいいだろう.

\textcircled{\scriptsize 2}の左辺を因数分解して,
\[(c-a)(c^2+ca+a^2)=b\cdot b^2\]
とし,さらに$(c-a)^2 \neq c^2+ca+a^2$であることから,
\[(c-a)\neq b\]
\[(c^2+ca+c^2)\neq b^2\]
を導いている.ここまではよい.

ところがこの式が書いてある部分の小見出しは「\textbf{★ \textcircled{\scriptsize 1}と\textcircled{\scriptsize 2}は矛盾を引き起こす}」である.べつに$(c-a)(c^2+ca+a^2)=b\cdot b^2$と上の式とは矛盾していない.これはどういうことなのだろうか?

さらにこの3ページ後,「\textbf{★ \textcircled{\scriptsize 1}と\textcircled{\scriptsize 2}との矛盾をクリアーする}」という謎の小見出しが登場する.「矛盾をクリアー」するとはどういうことなのだろう.

「矛盾をクリアーする」という節の内容を読んでみると,

\begin{quote}
最初の因数分解
\begin{eqnarray*}
(c^3-a^3)&=&b\cdot b^2\\
(c^3-a^3)&=&(c-a)(c^2+ca+a^2)
\end{eqnarray*}
が引き起こした矛盾は,素数の要素が二つ以上あることを念頭に入れていなかったからである.つまり,素数の要素が一つのときは,互いに素な二つの因数,$(c-a)$と$(c^2+ca+a^2)$に$b^3$を分けられない,あまりに当たり前のことを語っていただけのことである.
\end{quote}

要するに,「\textbf{$b$が素数の時しか考えてなかったから矛盾したような気がしたけど,よく考えたらそんなことなかったぜテヘペロ}」ということらしい.しかし読者としては,小野田氏が$b$が素数の場合しか考えてなかったことなど知る由もないし,こんな初歩的な混乱に読者をつき合わせる必要がどこにあるのだろう.

さらに言うと,$(c-a)\neq b$という式は今後使わない.なんのための議論だったのか.


\Subsubsection{$n=3$のときの証明 つづき}

気を取り直して先に進もう.
小野田氏によると,$a,c$は互いに素であることから,$(c-a)$と$(c^2+ca+c^2)$の最大公約数は1か3である.$c^2+ca+a^2=(c-a)^2+3ca$なのでこれは正しい.このことから小野田氏は

\begin{itemize}
\item $(c-a)$と$(c^2+ca+a^2)$が互いに素なとき,
\begin{eqnarray*}
(c-a)&=&(b_1)^3\\
(c^2+ca+a^2)&=&(b_2)^3\\
b&=&b_1b_2
\end{eqnarray*}

\item $(c-a)$と$(c^2+ca+a^2)$の最大公約数が3のとき,
\begin{eqnarray*}
(c-a)&=&3(b_1)^3\\
(c^2+ca+a^2)&=&3^{3q-1}(b_2)^3\\
b&=&3^qb_1b_2
\end{eqnarray*}
または,
\begin{eqnarray*}
(c-a)&=&3^{3q-1}(b_1)^3\\
(c^2+ca+a^2)&=&3(b_2)^3\\
b&=&3^qb_1b_2
\end{eqnarray*}
\end{itemize}

と場合分けする.二番目の場合分けは実際には不要だが,ここまではいい.
問題はそのあとの節である.

小野田氏は,第一の場合の二番目の式
\[(c^2+ca+a^2)=(b_2)^3\]
の左辺を因数分解して,
\[(c^2+ca+a^2)=(c-\sqrt{ca}+a)(c+\sqrt{ca}+a)\]
とする.$ca$が平方数でなければ,右辺の因数は整数にはならない.ここで小野田氏は奇妙な主張をする.

\begin{quote}
$\sqrt{ca}$が整数でなければ(無理数ならば),$(c^2+ac+a^2)$は,$1\cdot (c^2+ca+a^2)$を除いて,整数の積で表せないことを意味する.
\end{quote}

\textbf{意味するとは思えない.}小野田氏の主張によれば,$a,c$がいくつかの条件を満たしさえすれば$(c^2+ca+a^2)$は必ず素数になることになるが,そんなに簡単に素数が作れたら大変である.実際,$a=3,c=5$とでも置いてみれば$c^2+ca+a^2=49=7^2$となるが,$ca=15$は平方数ではない.どうして意味すると思ったのかよくわからない.

この主張は完全に誤りであるが,これを認めてしまえば証明はすぐそこである.

$(b_2)^3$は素数ではないので,$ca$は平方数で,$a,c$は互いに素なので$a=A^2,c=C^2$と書ける.すると
\[(C^2+CA+A^2)(C^2-CA+A^2)=(b_2)^3\]

ここで$A$と$C$は互いに素であることから,$(C^2+CA+A^2)$と$(C^2-CA+A^2)$も互いに素であることが導かれる.このことから,
\begin{eqnarray*}
(C^2+CA+B^2)&=&(B_1)^3\\
(C^2-CA+B^2)&=&(B_2)^3\\
\end{eqnarray*}
と書ける.

ここで,
\begin{eqnarray*}
(C^2+CA+B^2)&=&(C-\sqrt{CA}+A)(C+\sqrt{CA}+A)\\
(C^2-CA+B^2)&=&(C-\sqrt{3CA}+A)(C+\sqrt{3CA}+A)\\
\end{eqnarray*}
を用いれば,先ほどと同様の(破綻した)論法により$CA$と$3CA$はともに平方数となるが,これは$\sqrt{3}$が無理数であることに矛盾する.証明終了.

先ほどの場合分けの第二,第三の場合も本質的には同じである.


\Subsubsection{証明の問題点}
上記の証明の問題点は,そもそも論理が破綻しているというのもあるのだが,$(c-a)=(b_1)^3$という条件を使わず,$(c^2+ca+a^2)=(b_2)^3$という式のみから矛盾を導こうとしていることである.小野田氏は「★$\mathbf{(c^2+ca+a^2)=(b_2)^3}$\textbf{の条件吟味が全てだ}」と小見出しで宣言しているが,これはまずい.なぜなら$(c^2+ca+a^2)=(b_2)^3$には

\begin{eqnarray*}
1^2+1\cdot 18+ 18^2&=&7^3\\
17^2+17\cdot 36+ 36^2&=&13^3\\
17^2+17\cdot 73+ 73^2&=&19^3
\end{eqnarray*}
などの解があるし,$(c^2+ca+a^2)=3(b_2)^3$のほうも,
\begin{eqnarray*}
17^2+17\cdot 20+ 20^2&=&3\cdot 7^3\\
19^2+19\cdot 70+ 70^2&=&3\cdot 13^3
\end{eqnarray*}
などの解がある.

したがって第一の条件を無視し第二の条件に的を絞った小野田氏の方針は,失敗が宿命づけられていたといえる.

\Subsubsection{$n\geq 3$のときの証明}
$n=3$のときの証明ですでに破綻しているが,それ以上に気になるのは$n$が一般の場合に拡張できないことである.「$n=3$の証明によって,$n\geq 3$のすべてを証明してくれる」とはなんだったのか.

「$n\geq 3$のときの証明」の章は次のように始まっている.

\begin{quotation}
\[b^n=c^n-a^n\]
において,$b^n$の約数$b^{n-1},b^{n-2},\dots b^2$に対応する$(c^n-a^n)$の約数が存在するためには
\[c^n-a^n=(c-a)(c+a)^{n-1}\]
で表されなければならない.
\end{quotation}

\textbf{なぜだ.}どうしてそうなるのかわたしには全く分からない.

$c^n-a^n=(c-a)(c+a)^{n-1}$が成り立つのは$n=2$のときのみだ,だから$n\geq 3$のときは自然数解は存在しない,証明終了,となっているが,証明になっていないだけでなく,$n=3$のときの証明ともつながっていない.


実は「$n=3$のときの証明」の章の最後で小野田氏は,$(c^2+ca+a^2)=(b_2)^3$の左辺が$(c^2+2ca+c^2)$か$(c^2-2ca+a^2)$だったら,$(c+a)^2$や$(c-a)^2$と因数分解できるので無数の整数解を簡単に見つけられることを指摘している.それはそうなのだが,しかしそのことがこの式にどうつながるのだろうか? なんにせよ飛躍が多すぎて証明になっていない.


\Subsubsection{第二の証明}
この本の第四部は「フェルマーの定理の証明II」と題されており,第1部とは別の「証明」が書かれている.この証明は「第1部の証明を知ってしまった今となっては,まことに色あせたもの」らしく,なるほど第1部より長く,さらに読みづらい.

しかしどうも第1部と同種の誤りを犯しているようである.
まず,$(c^2+ca+a^2)^{\frac{1}{3}}$が整数になりえないことを示す,という間違った方針をとっている.
\[L=(c^2+ca+a^2)^{\frac{1}{3}}\]
とおき,$M=c+a$とおいてこれを
\[ca=M^2-L^3\]
と変形する.ここまではいい,しかしこのあと,$M^2-L^3$の「因数分解」として
\[M^2-L^3=(M^{\frac{2}{3}}-L)(M^{\frac{4}{3}}+M^{\frac{2}{3}}L+L^2)\]
および
\[M^2-L^3=(M-L^{\frac{3}{2}})(M+L^{\frac{3}{2}})\]
なる計算を行い,左辺が合成数なので,右辺の因数が整数になると考えてしまっているようである.もちろん実際にはこれらが整数になるとは限らない.

\Subsubsection{まとめ}

どうして小野田氏はこのような奇妙な「証明」を行ってしまったのだろう.

どうも小野田氏は,式として因数分解できるということと,その式に値を入れた結果が因数分解できるということを混同しているように思える.たしかに式として因数分解できれば値を入れても(因数の値が$\pm 1$にならなければ)因数分解されることになるが,逆は成り立たない.たとえば$f(X)=X$という多項式はもちろん既約だが,$X=6$を代入すれば$f(6)=6$となり素数にはならない.

そのうえで,中途半端に無理式を考慮している.$(c^2+ca+a^2)$を無理式の積で表すなら,
\[(c^2+ca+a^2)=(c+\sqrt{3ca+3a^2}+2a)(c-\sqrt{3ca+3a^2}+2a)\]
でもよいはずなのだが,なぜ上記の$\sqrt{ca}$が出てくる式だけ考えるのかよくわからない.

\Section{\S2 数学王国}

証明の検討はこれくらいにして,この本のほかの部分を拾い読みしてみよう.

まず本を開くと,目次より先に「社主激白」としてめいけい出版社長の言葉が載っている.「激白」とあるので出版の裏事情の暴露でもするのかと思ったら,小野田氏に対する狂わんばかりの賞賛の言葉が並んでいる.特に印象的なのが次の文句だ.

\begin{quote}
いま,社主が夢みることは,小野田氏を核に集う数学大好き人間が打ち立てる数学王国の実現である.危機の日本を救うのは数学立国の建設を措いてない.小野田氏の魅惑的なプランはもう次々と仕掛けられている.日本の未来は明るく輝いてきた.ひとなつこい小野田氏の輝く笑顔を見ると,だれでもこの人とともに,数学王国の建設に着手したい明るい気持ちになってくる.
\end{quote}

どこまで本気なのかよくわからない文章だが,なんだか「日本シャンバラ化計画」みたいで恐ろしい.\\

第5部「証明への道のり」には,小野田氏の「証明」完成までの経緯が書かれてる.
それによると,「コンピュータのハードを開発している友人」に「小野田さん,フェルマーの定理を証明したら」と「挑発」されたのが発端らしい.\textbf{罪作りな友人である}.そのあとの記述.

\begin{quote}
友人の挑発に乗り,証明のイメージをつかむことはできたのだが,高校1年で数学を捨て,大学入学と同時に数学と完全に縁を切った私である.
\end{quote}

\textbf{やっぱりな},という感じである.小野田氏の証明文の妙な読みにくさは,数学の高等教育を受けていないためなのだろう.

そして小野田は「証明」を完成させると,「読売年鑑の人名事典を調べ,およそ20人に近い数学者に原稿を送った」とのこと.さらにどうも証明を改訂するたびに数人の数学者に原稿を送っていたらしい.そのうち織田孝幸教授と浪川幸彦教授からは否定的な返事をもらい,織田教授は親切にも Paul Ribenboim 著『13 Lectures on Fermat's Last Theorem』からの抜粋をコピーして送ってくださったらしい.しかし小野田氏は英語が読めず,それどころか$A^3+B^3+C^3=0$という式だけを見て「\textbf{何というセンスの無さ}」というコメントをしてる有様である.

また著者の友人である田吉隆夫教授とは膨大な量の手紙のやり取りをしているらしい.この本にはそのうちのいくつかが載っており,田吉教授はきわめて誠実な態度で著者の「証明」の不明瞭な点を指摘していることがわかる.\textbf{持つべきものは友達である.}


\Section{\S3 おまけ そのほかの小野田襄二氏の本}

小野田襄二氏にはほかにもさまざまな著作があるらしいので,おまけとして最後に紹介しておく.

\begin{itemize}

\item 自然数解が存在する全構造の解明 (社会評論社)

今回紹介した本の続編らしい.\textbf{タイトルが壮大すぎる.}

\item 相対性理論の誤りを完全解剖する (小野田書店)

理系トンデモ本の王道(?)の相対性理論批判であり,『完全証明』のカバーに広告が載っている.面白いのは次の指摘だ.

\begin{quote}
動物の目は網膜に入った光に反応し,空間を伝播している光を見ているのではない.ゆえに,光の相対速度は観測できない.
\end{quote}

何を言っているのかよくわからないが,いままでの光速度の測定は人間が肉眼で行ってきたとでも思っているのだろうか.

\item やりなおし基礎数学 (ちくま新書)

小中学生向けの数学の本らしい.注目すべきはこれがちくま新書から出ていることである.\textbf{ブランド名にだまされてはいけない}.

\item 夫婦耐性実験 (小野田書店)

小説である.「父親・夫として破綻したゲオが,ノイローゼの駿台生を引きとることによって巻き起こす家族騒動物語」らしい.タイトルが面白い.\textbf{ちょっとよみたい.}
\end{itemize}


\Section{\S4 おわりに}

この手の本を読むことはよい暇つぶしになるが,そんなことをする暇があったらまともな数学書を読みましょう.
