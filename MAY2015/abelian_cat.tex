%\Section{アーベル圏入門(鯖白(奴隷))}
\Chapter{アーベル圏入門(鯖白(奴隷))}
\Section{\S 1圏について}

\begin{defi}
圏$\mathcal{C}$は,
\begin{itemize}

\item 
{\bf 対象}(object)のクラス$\mathrm{Ob}(\mathcal{C})$.誤解の恐れがない時は$\mathrm{Ob}(\mathcal{C})$を単に$\mathcal{C}$と書く.
$X$が$\mathcal{C}$の対象であるとき,$X \in \mathrm{Ob}(\mathcal{C}) (または X \in \mathcal{C})$と書く.

\item 
任意の対象$X,Y$に対し{\bf $X$から$Y$への射}(arrow, morphism)の集合が存在する.
$\mathrm{Hom}_\mathcal{C}(X,Y)$で表す.
$f \in \mathrm{Hom}_\mathcal{C}(X,Y)$のとき,$X$は$f$の{\bf ドメイン},$Y$は$f$の{\bf コドメイン},と呼ぶ.
この$f$について,$f:X \to Y$,または$X \xrightarrow{f} Y$と書く.

\item 
任意の対象$X,Y,Z$に対し,$\mathrm{Hom}_\mathcal{C}(X,Y) \times \mathrm{Hom}_\mathcal{C}(Y,Z)$から$\mathrm{Hom}_\mathcal{C}(X,Z)$への写像が存在し,{\bf 合成}と呼ばれる.$(f,g)$の像は$g \circ f$または$gf$で表し,$f$と$g$の合成と呼ぶ.

\end{itemize}
以上から構成され,次の2つの条件を満たす.
\begin{itemize}
\item
$X \xrightarrow{f} Y \xrightarrow{g} Z \xrightarrow{h} W$のとき,$h \circ (g \circ f) = (h \circ g) \circ f$ (結合律)
\item
任意の対象$X$に対し{\bf 恒等射}(identity morphism) $\mathrm{id}_X \in \mathrm{Hom}_\mathcal{C}(X,X)$が存在し,任意の$f:X \to Y , g:Z \to X$に対し
\[
	f \circ \mathrm{id}_X = f, 	\mathrm{id}_X \circ g = g,
\]
が成り立つ.
\end{itemize}
\end{defi} \proofend

% \begin{defi}
%対象のクラスが集合である圏を{\bf 小圏}(small category)という.
%\end{defi} \proofend

\begin{ex}
集合の圏$\mathbf{Sets}$.対象は任意の集合であり,$X,Y \in \mathbf{Sets}$ に対し$\mathrm{Hom}_\mathbf{Sets}(X,Y)$はXからYへの写像全体,合成は通常の写像の合成である.
\end{ex} \proofend

\begin{ex}
アーベル群の圏$\mathbf{Ab}$.対象は任意のアーベル群であり,$X,Y \in \mathbf{Ab}$ に対し$\mathrm{Hom}_\mathbf{Ab}(X,Y)$はXからYへの準同型写像全体,合成は通常の写像の合成である.
\end{ex} \proofend

\begin{ex}
位相空間の圏$\mathbf{Top}$.対象は任意の位相空間であり,$X,Y \in \mathbf{Ab}$ に対し$\mathrm{Hom}_\mathbf{Top}(X,Y)$はXからYへの連続写像全体,合成は通常の写像の合成である.
\end{ex} \proofend

\begin{ex}
任意の半順序集合$C$は以下のようにすると圏$\mathcal(C)$とみなせる.

$\mathrm{Ob}(\mathcal{C}) = C$とし,$x, y \in C$に対し $x \leq y$のとき$\mathrm{Hom}_\mathcal{C}$は一元集合,$x \nleq y$のときは
$\mathrm{Hom}_\mathcal{C} = \emptyset$とする.このとき合成は推移律により定義でき,恒等射は反射律により存在する.

逆に,圏$\mathcal{C}$,$\mathrm{Ob}(\mathcal{C})$が集合であり,$X,Y \in \mathcal{C}$ に対し $\mathrm{Hom}_\mathcal{C}(X,Y)$が1元以上持たないならば,$\mathrm{Ob}(\mathcal{C})$は$X \leq Y \stackrel{\mathrm{def.}}{\Leftrightarrow} \mathrm{Hom}_\mathcal{C} \neq\emptyset$ と定めることにより半順序集合となる.
\end{ex} \proofend

\begin{ex}
任意のモノイド$M$は,1つのみの対象$e$を持つ圏$\mathcal{C}$と見ることができる.
実際$\mathrm{Ob}(\mathcal{C}) = {e},\mathrm{Hom}_{\mathcal{C}}(e,e) = M$とすると,$\mathcal{C}$は二項演算を合成として圏をなす.逆に1つのみの対象を持つ圏$\mathcal{C}$について,$\mathcal{C} = \{e\}$とすると$\mathrm{Hom}_{\mathcal{C}}(e,e)$は合成を二項演算としてモノイドになる.\\
特に群$G$は上のようにしてモノイドとみなせる.
\end{ex} \proofend

\begin{defi}
$\mathcal{C},\mathcal{C'}$を圏とする.$\mathcal{C'}が\mathcal{C}$の{\bf 部分圏}(subcategory)であるとは,
\begin{itemize}
\item $\mathrm{Ob}(\mathcal{C'}) \subset \mathrm{Ob}(\mathcal{C})$
\item 任意の$X,Y \in \mathcal{C'}$に対し $ \mathrm{Hom}(\mathcal{C'}) \subset \mathrm{Hom}(\mathcal{C})$
\item 任意の$X,Y, Z \in \mathcal{C'}, f:X \to Y , g:Y \to Z$ に対し,$g \circ f$が$\mathcal{C'}$と$\mathcal{C}$で一致する.
\item 任意の$X \in \mathcal{C'}$に対し$X$の恒等射が$\mathcal{C'}$と$\mathcal{C}$で一致する.
\end{itemize}
をみたすことである.\\
さらに$ \mathrm{Hom}(\mathcal{C'}) = \mathrm{Hom}(\mathcal{C})$を満たすとき,$\mathcal{C'}$は$\mathcal{C}$の{\bf 充満部分圏}(full subcategory)という.
\end{defi} \proofend

大雑把に,圏とは「ある構造を持つモノ(対象)とその間の写像など(射)を全体で考えよう」というものです.対象と射は,集合と写像のようなものですが,対象に関しては一般の圏では何らかの集合とは限らず,ただのラベルに過ぎません.射に関しても同じで,射は一般には集合から集合への写像とは限らず,ドメインとコドメインが指定されているだけです.ここで,一般の圏において写像の全射,単射に相当する概念を定義しよう.
\begin{defi}
$\mathcal{C}$を圏とする.\\
$m:Y \to Z$が{\bf mono}とは任意の$f,g:X \to Y$に対し,
\[ m \circ f = m \circ g  \; \Rightarrow \; f = g \]
が成り立つことである.\\
$e:X \to Y$が{\bf epi}とは任意の$f,g:Y \to Z$に対し,
\[ f \circ e = g \circ e \; \Rightarrow \; f = g \]
が成り立つことである.
\end{defi} \proofend

\begin{prop}
$f:X \to Y , g:Y \to Z$において,$g \circ f$がmonoならば$f$はmono,$g \circ f$がepiならば$f$はepiである.
\end{prop}
(証明)\;略\proofend

$\mathbf{Sets}$では$f$が単射であることとmonoであること,全射であることとepiであることは一致する.

\begin{defi}
$f: X \to Y , g: Y \to X$を圏$\mathcal{C}$の射とする.% $g \circ f = \mathrm{id}_X$であるとき,$f$は$g$の{\bf 右逆射},$g$は$f$の{\bf 左逆射}と呼ぶ.$f$に対し右逆射$g_r :Y \to X$と左逆射$g_l: Y \to X$が存在するとき,$g_l = g_l \circ \mathrm{id}_Y = g_l \circ (f \circ g_r) = (g_l \circ f )\circ g_r =\mathrm{id}_X \circ g_r$ より$g_r$と$g_l$は一致する.このとき$f$を{\bf 同型射}という.
$g \circ f = \mathrm{id}_X , f \circ g = \mathrm{id}_Y$が成り立つ時,$f,g$を{\bf 同型射}(isomorphism)という.このとき,XとYは{\bf 同型}(isomorphic)であるといい,$X \sim Y$で表す.
\end{defi} \proofend
$\mathbf{Sets}$ではmonoかつepiなら同型射ですが,一般には成り立たない.例えば$\mathbf{Top}$で,$X=\{0,1\}$に離散位相,密着位相を入れたものをそれぞれ$(X,\mathscr{O}_0),(X,\mathscr{O}_1)$とする.$\mathrm{id}_X : (X,\mathscr{O}_0) \to (X,\mathscr{O}_1)$はmonoかつepiだが,その逆写像は連続ではない.

\begin{defi}
$\mathcal{C}$を圏とする.$\mathcal{C}$の{\bf 始対象}(initial object)とは,対象$I \in \mathcal{C}$であり,任意の対象$X \in \mathcal{C}$に対し$I$から$X$へ唯一の射が存在するものである.$\mathcal{C}$の{\bf 終対象}(terminal object)とは,対象$T \in \mathcal{C}$であり,任意の対象$X \in \mathcal{C}$に対し$X$から$T$へ唯一の射が存在するものである.始対象かつ終対象である対象を{\bf ゼロ対象}と言う.
\end{defi} \proofend
始対象,終対象,ゼロ対象は存在すれば同型を除いて一意である.\\
$\mathbf{Sets}$において,空集合$\emptyset$は始対象で,任意の一元集合は終対象となる.$\mathbf{Sets}$にはゼロ対象は存在しない.$\mathbf{Ab}$において,自明な群$0 = \{e\}$は始対象かつ終対象,すなわちゼロ対象となる.

\begin{defi}
$\mathcal{C}$を圏,$A,B\in \mathcal{C}$とする.$A,B$の{\bf 直積}(product)とは対象$A \times B$であり,射$\pi_A : A \times B \to A , 
\pi_B : A \times B \to B$が存在し,任意の$X \in \mathcal{C}, f_A : X \to A , f_B: X \to B$に対し $f_A = \pi_A \circ f , f_B = \pi_B \circ f$
を満たす$f : X\ to A \times B$が一意的に存在するようなものをいう(このような性質を直積の普遍性という).$A \prod B$とも書く.\\
\[
\xymatrix{
	& \ar[dl]_{f_A} X  \ar@{-->}[d]^{f}  \ar[dr]^{f_B}	& \\
A 	& \ar[l]^{\pi_A} A \times B \ar[r]_{\pi_B} 	& B 
}
\]
$A,B$の{\bf 直和}(coproduct)とは対象$A \bigoplus B$であり,射$\iota_A : A  \to A \bigoplus B, 
\iota_B : B \to  A \bigoplus B$が存在し,
任意の$X \in \mathcal{C}, f_A : A \to X , f_B: B \to X$に対し $f_A = f \circ \iota_A , f_B = f \circ \iota_B$
を満たす$f : A \bigoplus B \to X$が一意的に存在するようなものをいう(このような性質を直和の普遍性という).$A \coprod B$とも書く.\\
\[
\xymatrix{
	& \ar@{<-}[dl]_{f_A} X  \ar@{<--}[d]^{f}  \ar@{<-}[dr]^{f_B}	& \\
A \ar[r]_{\iota_A}	&  A \bigoplus B  & \ar[l]^{\iota_B} B 
}
\] 
\end{defi} \proofend

\begin{defi}
$\mathcal{C,D}$を圏とする.$\mathcal{C}$から$\mathcal{D}$への{\bf 関手}(functor)$F$とは
\begin{itemize}
\item 
$\mathrm{Ob}(\mathcal{C})$から$\mathrm{Ob}(\mathcal{D}$)への対応\\
 $\mathrm{Ob}(\mathcal{C}) \ni X \mapsto F(X) \in \mathrm{Ob}(\mathcal{D})$
 
 \item 
$X,Y \in \mathrm{Ob}(\mathcal{C})$,$\mathrm{Hom}_\mathcal{C}(X,Y) $から$\mathrm{Hom}_\mathcal{D}(F(X),F(Y))$への写像\\
$\mathrm{Hom}_\mathcal{C}(X,Y) \ni f \mapsto F(f) \in \mathrm{Hom}_\mathcal{D}(F(X),F(Y))$
\end{itemize}
 であって,以下を満たすものである.
 
 \begin{itemize}
 \item
 $X \in \mathcal{C}, \mathrm{id}_X \in \mathrm{Hom}_\mathcal{C}(X,X)$に対し$F(\mathrm{id}_X) = \mathrm{id}_{FX} \in \mathrm{Hom}_\mathcal{D}(FX,FX)$
 \item
 $X,Y,Z \in \mathcal{C}, X \xrightarrow{f} Y \xrightarrow{g} Z$に対し$F(g \circ f) = F(g) \circ F(f)$ \\
 $F(X),F(f)$を単に$FX,Ff$と書いたりもする.
  
 \end{itemize}
 写像$\mathrm{Hom}_\mathcal{C}(X,Y) \ni f \mapsto F(f) \in \mathrm{Hom}_\mathcal{D}(F(X),F(Y))$が全射のとき,Fを{\bf 充満関手}(full functor),単射のとき,{\bf 忠実関手}(faithful functor)という.
\end{defi} \proofend

%\begin{ex}
%関手の例?
%end{ex} \proofend

\begin{defi}
$\mathcal{C},\mathcal{D}$を圏.$F,G$を$\mathcal{C}$から$\mathcal{D}$への関手とする.$F$から$G$への{\bf 自然変換}$\eta$とは,各対象$X \in \mathcal{C}$に対して射$\eta_X : FX \to GX $が定まっていて,次の図式を可換にするようなものである.
\[ 
\xymatrix {
	FX	\ar[r]^{Ff} \ar[d]_{\eta_X}	& FY \ar[d]^{\eta_Y} \\
	GX	\ar[r]^{Gf} 				& GY
}
\]
\end{defi} \proofend

\Section{\S 2アーベル圏について}
\begin{defi}
圏$\mathcal{A}$が{\bf 加法圏}(additive category)であるとは,任意の$X,Y \in \mathcal{A}$に対し,$\mathrm{Hom}_\mathcal{A}(X,Y)$がアーベル群となり,かつ$\mathcal{A}$がゼロ対象$0$と,直積$X \times Y$(任意の$X,Y \in \mathcal{A}$) を持つことである.
\end{defi} \proofend
加法圏のモデルは加法群の圏$\mathbf{Ab}$である.$\mathbf{Ab}$は実際Hom集合はアーベル群となり,ゼロ対象$0 = {e}$を持ち,
アーベル群$G_1,G_2$に対して直積$G_1 \times G_2 $が存在するので,加法圏である.\\
加法圏$\mathcal{A}$において,対象$X$から$0$,$0$から$X$への唯一の射を同じ記号$0_X$で表す.に対し
$0_ : X \to Y$を$0_{XY} = 0_Y \circ 0_X$とする.$\mathcal{A}$は加法圏より,$\mathrm{Hom}_\mathcal{A}(X,Y)$はアーベル群をなすが,その単位元は$0_XY$であることが容易に確かめられる. また,$f:W \to X , g:Y\to Z$ に対し$0_{XY} \circ f = 0_{WY} ,
g \circ 0_{XY} = 0_{XZ}$ もわかる.以下,$0_{XY}$などを単に0と書く.
\[
\xymatrix{
&	X \ar[r]^{0_{XY}} \ar@/_/[rrd]^{0_{XZ}}& Y \ar[dr]^{g} \\
W  \ar[ur]^f \ar@/_/[urr]_{0_{WY}}&&&Z
}
\]
$\mathbf{Ab}$において(圏論的)直積と(圏論的)直和は一致するが,実は一般の加法圏においてもこれが成り立つ.
\begin{prop}
加法圏$\mathcal{A}$の任意の対象$X,Y$に対し,直和$X \bigoplus Y$が存在し.直積と一致する.
\end{prop}
(証明)
$\mathrm{id}_X : X \to X, 0 :Y \to X$に対し,直積$X \times Y$の普遍性を用いると,
\[
\pi_X \circ \iota_X = \mathrm{id}_X\;\;\;\;\;\pi_Y \circ \iota_X = 0
\]
をみたす$\iota_X : X \to X\times Y$が一意的に存在する.
\[
\xymatrix{
	& \ar[dl]_{\mathrm{id}_X} X  \ar@{-->}[d]^{\iota_X}  \ar[dr]^{0}	& \\
X 	& \ar[l]^{\pi_X} X \times Y \ar[r]_{\pi_Y} 	& Y 
}
\]
同様にして,$0: Y \to X, 0 \mathrm{id}_Y:Y \to Y$に対し,直積$X \times Y$の普遍性を用いると,
\[
\pi_X \circ \iota_Y = 0\;\;\;\;\;\pi_Y \circ \iota_Y = \mathrm{id}_Y
\]
をみたす$\iota_X : X \to X\times Y$が一意的に存在する.
\[
\xymatrix{
	& \ar[dl]_0 Y  \ar@{-->}[d]^{\iota_Y}  \ar[dr]^{\mathrm{id}_Y}	& \\
X 	& \ar[l]^{\pi_X} X \times Y \ar[r]_{\pi_Y} 	& Y 
}
\]
\\
さて,次の図式は上の$\iota,\pi$の式より可換である.
\[
\xymatrix{
	& \ar@/_/[dl]_0 X \times Y  \ar[d] | {\iota_X \circ \pi_X + \iota_Y \circ \pi_Y}  \ar@/^/[dr]^{\mathrm{id}_Y}	& \\
X 	& \ar[l]^{\pi_X} X \times Y \ar[r]_{\pi_Y} 	& Y 
}
\]
\\
また,上の$\iota_X \circ \pi_X + \iota_Y \circ \pi_Y$を$\mathrm{id}_{X \times Y}$に取り替えても可換である.直積の普遍性より,
上の図式を可換にする$X \times Y$から$X \times Y$への射は唯一なので,
\[
\iota_X \circ \pi_X + \iota_Y \circ \pi_Y = \mathrm{id}_{X \times Y}
\]
が成り立つ.
この$\iota_X,\iota_Y$が直和の普遍性を満たすことを示そう.\\
$Z\in \mathcal{A},f:X \to Z, g:Y \to Z$とする.$h : X \times Y \ o Z $を$h = f \circ \pi_X + g \circ \pi_Y$とすると,このhは下図を可換にする.
\[
\xymatrix{
& \ar@{<-}[dl]_f Z  \ar@{<-}[d]^{h}  \ar@{<-}[dr]^{g}  \\
X \ar[r]^{\iota_X} &  X \times Y & 	\ar[l]_{\iota_Y}  Y 
}
\]
\\
よって可換にするものの存在はいえた.このようなものが唯一であることをを示すために,$h_1,h_2: X \times Y \to Z $は
\[
h_1 \circ \iota_X = f \;\;\;\;\; h_1 \circ \iota_Y = g 
\]\[
h_2 \circ \iota_X = f \;\;\;\;\; h_2 \circ \iota_Y = g
\]
としよう.これより
\[
(h_1-h_2)\circ \iota_X = 0\;\;\;\;\;(h_1-h_2)\circ \iota_Y = 0
\]
となる,それぞれ右から$\pi_X,\pi_Y$を合成して和を取ると
\begin{align*}
0	&= (h_1 - h_2)\circ (\iota_X \circ \pi_X + \iota_Y \circ \pi_Y) \\
	&= (h_1 - h_2)\circ \mathrm{id}_{X \times Y} \\
	&= h_1 - h_2 
\end{align*}
より$h_1 = h_2$がわかる.
\proofend

さて,加法圏において,$\mathbf{Ab}$の準同型の核,像に相当するものを定義したい.しかしmono,epiのときと同様,対象は一般には集合ではないので,アーベル群と同じようには定義できない.まず,核,余核を,普遍性を使って定義する.

\begin{defi}
加法圏$\mathcal{A}$の射$f:X \to Y$の{\bf 核}とは,射$\mathrm{ker}(f): \mathrm{Ker}(f) \to X$で,$f \circ \mathrm{ker}(f) = 0$をみたし,次の普遍性を満たすものである. \\
{\rm(核の普遍性)}\;$k \circ f = 0$をみたす任意の$k:K \to X$に対し,下図を可換にする一意的な射$i:K \to \mathrm{Ker}(f)$が存在する.
\[
\xymatrix{
\mathrm{Ker}(f) \ar[r]^{\mathrm{ker}(f)} & X \ar[r]^f &Y  \\
K \ar[ur]_{k}  \ar@{.>}[u]^i
}
\]\\

また,加法圏$\mathcal{A}$の射$f:X \to Y$の{\bf 余核}とは,射$\mathrm{cok}(f): Y \to \mathrm{Cok}(f)$で,$\mathrm{cok}(f) \circ f = 0$をみたし,次の普遍性を満たすものである. \\
{\rm(余核の普遍性)}\;$f \circ c = 0$をみたす任意の$c:Y \to C'$に対し,下図を可換にする一意的な射$p:\mathrm{Cok}(f) \to C$が存在する.
\[
\xymatrix{
X \ar[r]^f 	& Y \ar[r]^{\mathrm{cok}(f)} &\mathrm{Cok}(f) \ar@{.>}[d]^p \\
&&C \ar@{<-}[ul]_{c}  
}
\]
\end{defi} \proofend
加法圏の核,余核は常に存在するとは限らない.対象より射の方に着目することが多いので,ここでは射の核,余核は射を指すことにする.$\mathrm{Ker}(f), \mathrm{Cok}(f)$は,2つあったら同型である.\\
$\mathbf{Ab}$の準同型$f:G_1 \to G_2$において包含写像$i:\mathrm{Ker}(f) \to G_1$は$f$の核であり,自然な全射$p:G_2 \to \mathrm{Cok}(f)$は$f$の余核である.

\begin{defi}
加法圏$\mathcal{A}$の射$f:X \to Y$の{\bf 像}とは,余核$\mathrm{cok}(f): Y \to \mathrm{Cok}(f)$の核$\mathrm{im}(f): \mathrm{Im}(f) \to Y$である.\\
また,加法圏$\mathcal{A}$の射$f:X \to Y$の{\bf 余像}とは,核$\mathrm{ker}(f): \mathrm{Ker}(f) \to X$の余核$\mathrm{coim}(f): X \to \mathrm{Coim}(f) $である.
\end{defi} \proofend
\begin{prop}
加法圏において,核はmonoであり,余核はepiである.
\end{prop}
(証明)\;略\proofend
\begin{prop}
加法圏$\mathcal{A}$の射$f:X \to Y$に対して
\begin{itemize}
\item $f$がmono $\Leftrightarrow f\circ e = 0$ならば$e=0$\;($e:W \to X$)
\item $f$がepi \;\;\;\;$\Leftrightarrow g\circ f = 0$ならば$g=0$\;($g:Y \to Z$)
\end{itemize}
\end{prop}
(証明)\;略\proofend

\begin{defi}
圏$\mathcal{A}$が{\bf アーベル圏}であるとは,$\mathcal{A}$が加法圏であり,任意の射$f$について
\begin{itemize}
\item 任意の射は核と余核をもつ.
\item 任意のmono射$m$はその余核の核である.つまり$m = \mathrm{ker}(\mathrm{cok(m)})$
\item 任意のepi射$e$はその核の余核である.つまり$e = \mathrm{cok}(\mathrm{ker(e)})$
\end{itemize}
をみたすことである.

\end{defi} \proofend

\begin{prop}
アーベル圏$\mathcal{A}$の射$f$は,monoかつepiならば同型射になる.
\end{prop}
(証明) $f$はmonoより$ f = \mathrm{ker}(\mathrm{cok}(f))$.$\mathrm{cok}(f) \circ f = 0 = 0 \circ f$.
$f$はepiなので$\mathrm{cok}(f) = 0$これより,$\mathrm{id}_Y:Y \to Y$は$0 = \mathrm{cok}(f)$の核になることがわかる.
核のコドメインは同型より,$f$は同型射となる.
\[
\xymatrix{
X \ar[rr]^{\mathrm{ker}(\mathrm{cok}(f))}_{f} | = && Y\ar[r]_{0}^{\mathrm{cok}(f)} | = & C  \\
&Y \ar@{-}[ul]^{\cong} \ar[ur]_{\mathrm{id}_Y}
}
\]
\proofend
アーベル圏$\mathcal{A}$の射$f : X \to Y$について,次の分解がある.
\[
\xymatrix{
&\mathrm{Im(f)} \ar[rd]^m  \\
X\ar[ru]^e \ar[rr]^f  && Y
}
\]
ここで,$m=\mathrm{im(f)}$である.この分解は,次のように得る.\\
余核の定義により$\mathrm{cok}(f) \circ f = 0$.$\mathrm{im}(f) = \mathrm{ker}(\mathrm{cok}(f))$なので,核の普遍性
より,下図を可換にする一意的な射$e: X \to \mathrm{Im}(f)$がのびる.
\[
\xymatrix{
&\mathrm{Im}(f) \ar[rd]^{\mathrm{im(f)}} \\
X\ar@{.>}[ru]^e \ar[rr]^f  && Y \ar[rd]^{\mathrm{cok}(f)} \\
&&& \mathrm{Cok}(f)
}
\]
$m=\mathrm{im(f)}$は核なのでmonoである.実は$e$がepiであることも示せる.
\begin{lem}
上の$f$に対し,mono射$m': K \to Y$と射$e': X \to K$について$f=m' \circ e'$が成り立つとき,次を可換にする一意的な射
$t :\mathrm{Im}(f) \to X$が存在する.
\[
\xymatrix{
X \ar[d]_{e'} \ar[r]^{e} & \mathrm{Im}(f) \ar[d]^m  \ar@{.>}[ld]_t\\
K \ar[r]_{m'} & Y
}
\]
\end{lem}
(証明) 
$p = \mathrm{cok}(m) ,\;p' = \mathrm{cok}(m')$とおく.$m$はmonoなので$m' = \mathrm{ker}(p')$.
$p'  m' = 0$より$p'  m'  e' = p'  f = 0$ .
よって$\mathrm{Cok}(f)$の普遍性より一意的な射$w : \mathrm{Cok}(f) \to C$が存在して
$p' = w p$.
\[
\xymatrix{
X \ar[d]_{e'} \ar[r]^e \ar[rd]^f 	& \mathrm{Im}(f)  \ar[d]^m  \\
K \ar[r]_{m'} 				& Y \ar[r]^{p'} \ar[d]_{p} &	C \\
						& \mathrm{Cok}(f) \ar@{.>}[ru]_w
}
\]
これより$p'  m = w  p  m = 0$.$m' = \mathrm{ker}(p')$なので,核の普遍性より
一意的な射$t:\mathrm{Im}(f) \to K$が存在して$m = m't$.これより右下の可換がいえた.
また$m'te = me = f = m'e'$.$m'$はmonoなので,$te = e'$.これより左上の可換が言えた.
\proofend

\begin{thm}
アーベル圏$\mathcal{A}$の射$f : X \to Y$の分解
\[
\xymatrix{
&\mathrm{Im(f)} \ar[rd]^m  \\
X\ar[ru]^e \ar[rr]^f  && Y
}
\]
において,$m$はmono,$e$はepiである.
\end{thm}
(証明)$m$は核なのでmonoである.$e$がepiを示す.射$r,s : \mathrm{Im(f)}  \to Z$で$re = se$をみたすとする.
$k = \mathrm{ker}(r-s)$とおく.核の普遍性より,下図を可換にする$e' : X \to \mathrm{Ker}(r-s)$が一意的に存在する.
\[
\xymatrix{
\mathrm{Ker}(r-s) \ar[rd]^k \\
&\mathrm{Im}(f) \ar@<1ex>^r[r] \ar@<-1ex>_s[r] & Z \\
X \ar@{.>}[uu]^{e'} \ar[ru]^e
}
\]
$k,m$はmonoよりその合成もmono.よって上の補題より,下図を可換にする一意的な射$t$が存在する.
\[
\xymatrix{
X \ar[d]_{e'} \ar[rr]^e && \mathrm{Im}(f) \ar[d]^m  \ar@{.>}[lld]_t\\
\mathrm{Ker}(r-s) \ar[r]_{k} & \mathrm{Im}(f)  \ar[r]_m& Y
}
\]
$m = m k t$で$m$はmonoより$kt=\mathrm{id}_{\mathrm{Im}(f)}$ .$k = \mathrm{ker}(r-s)$であったので
$(r-s)k = 0$.右から$t$を合成して$(r-s)kt=r -s = 0$.すなわち$r=s$.
\proofend

$\mathrm{Coim}(f)$を使っても同様の分解が得られる.monoとepi,核と余核は双対であるので,
上と同様に$f$はmono射$m$とepi射${\mathrm{coim}(f)}$によって分解されることが示される.
\[
\xymatrix{
&\mathrm{Coim(f)} \ar[rd]^{m}  \\
X\ar[ru]^{\mathrm{coim}(f)} \ar[rr]^f  && Y
}
\]
$\mathrm{Im(f)},\mathrm{Coim(f)} $による分解によって得られたepi射,mono射をそれぞれ$e,m$とする.
$\mathrm{Im(f)},\mathrm{Coim(f)} $の普遍性を使うと下図を可換にする$i:\mathrm{Im(f)} \to \mathrm{Coim(f)} $が一意的に存在する.

\[
\xymatrix{
&&\mathrm{Coim}(f)\ar[rrd]^m \ar[r]_i^{\cong}&\mathrm{Im}(f) \ar[rd]^{\mathrm{im}(f)} \\
&X\ar[ru]^{\mathrm{coim}(f)} \ar[rru]^e \ar[rrr]^f&&&Y \ar[rd]^{} \\
\mathrm{Ker}(f) \ar[ru]^{\mathrm{ker}(f)} &&&&& \mathrm{Cok}(f)
}
\]
$e = i \circ {\mathrm{coim}(f)}, m = {\mathrm{im}(f)} \circ i$なので$i$はmonoかつepi,したがって同型射である.よって次を得る.
\begin{thm}
アーベル圏$\mathcal{A}$の任意の射$f : X \to Y$に対し$\mathrm{Coim}(f) \cong \mathrm{Im}(f)$である.
\end{thm}
\proofend

\begin{defi}
アーベル圏$\mathcal{A}$の射の列$X \stackrel{f}{\to} Y \stackrel{g}{\to}Z$がYにおいて{\bf 完全}(exact)とは,
$\mathrm{im}(f) = \mathrm{ker}(f)$となることである. 
\end{defi} \proofend

\begin{defi}
$\mathcal{C}$を圏,$f:X \to Z, g:Y \to Z$を射とする.$f,g$の{\bf 引き戻し}(pullback)とは,下図の可換図式であり,以下の普遍性を満たすものである.
\[
\xymatrix
{
P \ar[r]^{q} \ar[d]^{p} 	& Y \ar[d]^g\\
X \ar[r]^{f}			& Z
}
\]
普遍性.$p' : P' \to Y, q': P' \to X$で$f \circ p' = g\circ q'$を満たすような$p',q'$に対し一意的な射$t:P' \to P$が存在し,下図を可換にする.
\[
\xymatrix
{
P' \ar@/_/[rdd]^{p'} \ar@/^/[rrd]^{q'}  \ar@{.>}[rd]^t\\
&	P \ar[d]^{p} \ar[r]^{q} 	& Y \ar[d]^g\\
&	X \ar[r]^{f}			& Z
}
\]
\end{defi} \proofend

\begin{defi}
$\mathcal{C}$を圏,$f:X \to Y, g:X \to Z$を射とする.$f,g$の{\bf 押し出し}(pushout)とは,下図の可換図式であり,以下の普遍性を満たすものである.
\[
\xymatrix
{
X \ar[r]^{g} \ar[d]^{f} 	& Z \ar[d]^q\\
Y \ar[r]^{p}			& P
}
\]
普遍性.$p' : Y \to P', q': Z \to P'$で$p' \circ f  =q' \circ g$を満たすような$p',q'$に対し一意的な射$t:P \to P'$が存在し,下図を可換にする.
\[
\xymatrix
{
X \ar[r]^{g} \ar[d]^{f} 	& Z \ar[d]^q\ar@/^/[rdd]^{q'}\\ 
Y \ar[r]^{p} \ar@/_/[rrd]^{p'}& P \ar@{.>}[rd]^t  \\
&&P'
}
\]
\end{defi} \proofend

上の対象$P$はどちらも同型を除いて一意である.

\begin{prop}
圏$\mathcal{C}$において,$f:X \to Z, g:Y \to Z$を射とし,
\[
\xymatrix
{
P \ar[d]^{p} \ar[r]^{q} 	& Y \ar[d]^g\\
X \ar[r]^{f}			& Z
}
\]
を引き戻しとする.このとき,$f$がmonoならば$p$もmonoである.
\end{prop}
(証明)$f$はmonoであるとする.$r,s:W \to P$を射とし,$pr=ps$であるとする.このとき$fpr = gqr = gqs = fps$.$f$はmonoなので$pr=qr$.これより引き戻しの普遍性から$r=s$がわかる.

\[
\xymatrix
{
W \ar@/^/@<1ex>[rrd]^{pr=ps} \ar@/_/@<-0.5ex>[rdd]_{qr=qs}  \ar@<0.5ex>[rd]^r\ar@<-1ex>[rd]_s\\
&	P \ar[d]^{p} \ar[r]^{q} 	& Y \ar[d]^g\\
&	X \ar[r]^{f}			& Z
}
\]
\proofend
アーベル圏ではepiの場合も成り立つ.
\begin{prop}
アーベル圏$\mathcal{A}$において,$f:X \to Z, g:Y \to Z$を射とし,
\[
\xymatrix
{
P \ar[r]^{q} \ar[d]^{p} 	& Y \ar[d]^g\\
X \ar[r]^{f}			& Z
}
\]
を引き戻しとする.このとき,$f$がepiならば$p$もepiである.
\end{prop}
(証明)
$f$はepiとする.直積$X \times Y$からZへの射$ f \pi_X - g  \pi_Y$の核を$m:P \to X \times Y$とすると,下図は引き戻しになる.
($m$が$f \pi_X - g  \pi_Y$の核であることから可換性がわかり,直積と核の普遍性により引き戻しの普遍性が導かれる.)
\[
\xymatrix
{
P \ar[r]^{\pi_Y m} \ar[d]_{\pi_X m} 	& Y \ar[d]^g\\
X \ar[r]^{f}			& Z
}
\]
上のPは同型を除いて一意であったので,上の図で$f$がepiならば$\pi_Y m$もepiを示せばよい.
このとき$f \pi_X - g  \pi_Y$はepiである.ある$h:W \to X \times Y$について$h(f \pi_X - g  \pi_Y) = 0$とする.命題16より
直積は直和でもあるので,$\iota_X$を合成すると
\[
	0 = h(f \pi_X - g  \pi_Y)\iota_X = hf \pi_X \iota_X =hf
\]
$f$はepiだったので$h=0$,よって$f \pi_X - g  \pi_Y$はepi.これより$f \pi_X - g  \pi_Y = \mathrm{cok}(m)$となる.
ある$v : Y \to V$について$v \pi_Y m=0$とする.$\mathrm{cok}(m)$の普遍性により
$v\pi_Y = v'(f \pi_X - g  \pi_Y)$をみたす$v' :V \to Z$が一意的に存在する.
\[
\xymatrix
{
P\ar[r]^m &X \times Y \ar[rd]_{f \pi_X - g  \pi_Y} \ar[r]^{v\pi_Y} & V \\
&&Z  \ar@{.>}[u]_{v'}
}
\]
$\iota_X$を右から合成すると,
$\pi_Y\iota_X=0$なので
\[
	0 = v\pi_Y\iota_X = v'(f \pi_X - g  \pi_Y)\iota_X = v'f\pi_X\iota_X =v'f
\]
$f$はepiなので$v'=0$,よって$v\pi_Y = 0$,$v= v\pi_Y\iota_Y=0$.
\proofend
\begin{cor}
アーベル圏において常に引き戻しが存在する.
\end{cor}
\proofend
\begin{prop}
アーベル圏$\mathcal{A}$において,$f:X \to Z, g:Y \to Z$の引き戻しを考える.$k=\mathrm{ker}(f)$とすると下図のような
一意的な分解$k = pk'$が得られ,$k'=\mathrm{ker}(q)$となる
\[
\xymatrix
{
								&P \ar[r]^{q} \ar[d]^{p} 	& Y \ar[d]^g\\
\mathrm{Ker}(f)\ar[r]^k \ar@{.>}[ru]^{k'}	&X \ar[r]^{f}			& Z
}
\]
\end{prop}
(証明)
$\mathrm{Ker}(f)$から$Y$へのゼロ射$0$を考えることにより,引き戻しの普遍性から一意的な射$k':\mathrm{Ker}(f)\to P$が
存在して$pk'=k,qk'=0$となる.
\[
\xymatrix
{
& K \ar@(l,u)@{.>}[ldd]_{u'} \ar[d]^{u}\\
&P \ar[r]^{q} \ar[d]^{p} 	& Y \ar[d]^g\\
\mathrm{Ker}(f)\ar[r]^k \ar@{.>}[ru]^{k'} &X \ar[r]^{f}			& Z
}
\]
この$k'$が$q$の核となることを示そう.$u:K \to P$を$qu=0$をみたす射とする.右から$g$を
合成すると$0=uqg = upf$となる.これより$\mathrm{Ker}(f)$の普遍性から一意的な射$u':K \to \mathrm{Ker}(f)$が存在して
$ku' = pu$.これにより引き戻しの普遍性が使えて,$u=k'u'$が導かれる.これより$k'$は$q$の核となる.
\[
\xymatrix
{
K \ar@/^/@<1ex>[rrd]^{0} \ar@/_/@<-0.5ex>[rdd]_{ku'=pu}  \ar@<0.5ex>[rd]^{k'u'}\ar@<-1ex>[rd]_{u}\\
&	P \ar[d]^{p} \ar[r]^{q} 	& Y \ar[d]^g\\
&	X \ar[r]^{f}			& Z
}
\]

\proofend
引き戻しについて上で示されたことの双対は,押し出しについても同様に示される.

\begin{defi}
アーベル圏$\mathcal{A}$の対象$X\in \mathcal{A}$をコドメインに持つ射$f$をXの{\bf メンバ}と呼び,$f \in {}_mX$で表す.
\end{defi} \proofend
$\mathcal{A}$をアーベル圏,$X\in \mathcal{A}$とする.Xのメンバ$f,g$について,あるepi射$u.v$が存在して$fu=gv$をみたすとき,
$f \equiv g$と定義する.明らかにこれは反射律,対称律をみたす. 推移律を確かめる.
\[
\xymatrix
{
\nkgr \ar[d]^{v'} \ar[r]^{w'} & \nkgr \ar[d]^v \ar[r]^u& \nkgr \ar[d]^f\\ 
\nkgr \ar[d]^r \ar[r]^w & \nkgr \ar[d]^g \ar[r]^g& \nkgr \\
\nkgr \ar[r]^h          & \nkgr
}
\]

$f \equiv g,g \equiv h$とすると,epi射$r,u,v,w$
が存在して$fu=gv,gw=hr$をみたす.下図において,$v,w$の引き戻しを考える.$v,w$epiなので命題29より$v',w'$もepi.よって$f \equiv z$である.
アーベル圏の任意の対象$X$について,$0 \in _mX$であり,$f \in _mX$なら$-f \in _mX$である.\\
$f:X \to Y$をアーベル圏の射とする.$p \in _mX$に対し$fp \in _mY$となる.$p,q \in _mX,p \equiv q$とすると$fp \equiv fq$となる.これより射$f$は,あたかも写像のようにXのメンバをYのメンバにうつす.
\begin{thm}
(図式追跡のための基本法則).アーベル圏におけるメンバについて,以下の性質が成り立つ.
\begin{enumerate}
\item $f:X \to Y $がmono $\Leftrightarrow$ 任意の$x \in _mX$に対し$fx \equiv 0$ならば$x\equiv0$
\item $f:X \to Y $がmono $\Leftrightarrow$ 任意の$x ,x' \in _mX$に対し$fx \equiv fx'$ならば$x\equiv x'$
\item $f:X \to Y $がepi $\Leftrightarrow$  任意の$y \in _mY$に対し$x \in _mX$が存在し$gx \equiv x$
\item $f:X \to Y $がゼロ射 $\Leftrightarrow$  任意の$x \in _mX$に対し$fx \equiv 0$
\item $\xymatrix@1{ X \ar[r]^f & Y \ar[r]^g & Z } $が$Y$で完全 $\Leftrightarrow$ $gf = 0$かつ$gy \equiv 0$を満たす$y \in _mY$に対し
$x \in _mX$が存在して$y \equiv fx$となる.
%\item これいるのかな?
\end{enumerate}
\end{thm}
(証明)\; 3.と5.だけ示す.\\
3.\; $f:X \to Y $をepi,$y \in _mY$とする. $f$と$y$に対し引き戻しを考える.
\[
\xymatrix{ \ar@{}[rd] |{p.b.}
\nkgr \ar[r]^e \ar[d]^x& \nkgr \ar[d]^y \\
X \ar[r]^f & Y
}
\]

このとき$f$はepiなので$e$もepi,よって$y \equiv fx$がわかる.

逆に$f$がepiでないとすると,どの$x\in _mX$に対しても$fx$はepiでないので,$fx \neq \mathrm{id}_Y \in {}_mY$である.


5.\; $\xymatrix@1{ X \ar[r]^f & Y \ar[r]^g & Z } $がYで完全とする.$\mathrm{Im}(f)$による$f$の分解$f = me$を考える.$Y$で完全より
$m=\mathrm{im}(f)=\mathrm{ker}(g)$.よって$gf = gme = g \mathrm{ker}(f) e = 0$.$y \in _mY$で,$gy \equiv 0$なものを考える.$\equiv$の定義から直ちに$gy = 0$がいえる.ここで,次の図式を考える.
\[
\xymatrix{
\nkgr \ar@{.>}[d]^{x}  \ar@{.>}[r]^{e'}& \nkgr \ar@{=}[r] \ar@{.>}^p[d]& \nkgr \ar[d]^y \\
X \ar[r]^e & \mathrm{Ker}(g) \ar[r]^m & Y
}
\]
核の普遍性から上を可換にする$p$が唯一存在する.$e,p$の引き戻しを上の図のように考えると$ye' = mex = fx$.
$e$はepiなので$e'$もepi.よって$x \in {}_mX$が存在して$y \equiv fx$となる.\\
逆に$gf = 0$かつ$gy \equiv 0$を満たす$y \in _mY$に対し$x \in _mX$が存在して$y \equiv fx$となるとする.$k = \mathrm{ker}(g)$は
$gk \equiv 0$を満たすので$x \in _mX$が存在して$k \equiv fx$.よって$ \mathrm{cok}(f) k \equiv \mathrm{cok}(f) fx \equiv 0 $なので
$ \mathrm{cok}(f) k = 0$.よって$\mathrm{Im}(f)$の普遍性から下図を可換にする射$p$が存在する.
\[
\xymatrix{
\nkgr \ar@{>>}[r] \ar@{>>}[d] & \mathrm{Ker}(g)\ar[rd]^k \ar@{.>}[r]^p & \mathrm{Im}(f) \ar[d]^{\mathrm{im}(f)}\\
\nkgr \ar[r]^x & X \ar[r]^f &Y \ar[d]^{\mathrm{cok}(f)} \ar[r]^g& Z \\
&&\mathrm{Cok}(f) 
}
\]
また$gf = 0$と$f$の分解$f = me$より$gf = gme = 0$.eはepiより$gm = 0$がわかる.よって$\mathrm{Ker}(g)$の普遍性から,下図を可換にする射$q$が存在する.
\[
\xymatrix{
&\mathrm{Im}(f)\ar[rd]^m \ar@{.>}[r]^	q&\mathrm{Ker}(g)\ar[d]^{\mathrm{ker}(g)} \\
X\ar[ru]^e \ar[rr]^f&&Y\ar[r]^g&Z
}
\]
この$p,q$に対して,$qp,pq$を考えると核と像の普遍性によりどちらも恒等射になる.これよりYで完全であることがわかった.
\proofend

この定理を用いると,アーベル圏における図式の補題が図式の追跡により示すことができる.

\begin{lem}五項補題(five lemma)\\
下図の可換図式
\[
\xymatrix{
X_1 \ar[d]^{f_1} \ar[r]^{g_1} & X_2 \ar[d]^{f_2} \ar[r]^{g_2} & X_3 \ar[d]^{f_3} \ar[r]^{g_3} & X_4 \ar[d]^{f_4} \ar[r]^{g_4} &
X_5 \ar[d]^{f_5}  \\
Y_1 \ar[r]^{h_1} & Y_1 \ar[r]^{h_2} & Y_3 \ar[r]^{h_3} & Y_4 \ar[r]^{h_4} & Y_5 
}
\]
において,行は完全であるとする.このとき,$f_1,f_2,f_4,f_5$が同型射ならば,$f_3$も同型射である.
\end{lem}
(証明)
$f_3$がmonoを示せば,双対的にepiも示されるので,$f_3$がmonoを示せば十分である.$x \in {}_mX$で$f_3x \equiv 0$とする.
これと図の可換から$f_4g_3x = h_4f_3x \equiv 0$.$f_4$はmonoなので$g_3x \equiv 0$.
完全性を用いて定理33の5.より$x_2 \in {}_mX$が存在して$x \equiv g_2x$.$h_2f_2x_2 \equiv f_3x \equiv 0 $.これより再び定理33の5.を用いると$y_1 \in {}_mY$が存在して$f_2x_2 \equiv h_1y_1$.$f_1$epiなので定理33の3.より$x_1\in {}_mX$が存在して$y_1 \equiv f_1x_1$.
$f_2g_1x_1 \equiv f_2x_2$と$f$:monoより$x_2 \equiv g_1x_1$.よって$x \equiv g_2g_1x_1 \equiv 0$.
\[
\xymatrix{
x_1\ar@{|.>}^{\mathrm{epi}}[d] & x_2 \ar@{|.>}[r] \ar@{|->}[d]&x \ar@{|->}[d] \ar@{|->}[r]& g_3x \ar@{|->}^{\mathrm{mono}}[d] \\
y_1 \ar@{|.>}[r]& f_2x_2 \ar@{|->}[r]& 0\ar@{|->}[r] & f_4g_3x
}
\]
\proofend
もうひとつ図式の補題を示すために,以下の列が完全である可換図式を考えよう.
\[
\xymatrix{
0 \ar[r] & X \ar[d]^f \ar[r]^m & Y \ar[d]^g \ar[r]^e & Z \ar[d]^h \ar[r] & 0 \\
0 \ar[r] & X' \ar[r]^{m'} &Y' \ar[r]^{e'} & Z' \ar[r] & 0 
}
\]
この図式で$f,g,h$の核,余核をとると,普遍性から以下の可換図式が得られ,一番上と一番下の行は完全である.
\[
\xymatrix{
0 \ar[r] & \mathrm{Ker}(f) \ar[d] \ar[r]^{m_0} &\mathrm{Ker}(g) \ar[d] \ar[r]^{e_0} & \mathrm{Ker}(h) \ar[d]   \\
0 \ar[r] & X \ar[d]^f \ar[r]^m & Y \ar[d]^g \ar[r]^e & Z \ar[d]^h \ar[r] & 0 \\
0 \ar[r] & X' \ar[d] \ar[r]^{m'} &Y' \ar[d] \ar[r]^{e'} & Z' \ar[d] \ar[r] & 0 \\
& \mathrm{Cok}(f) \ar[r]^{m_1} &\mathrm{Cok}(g) \ar[r]^{e_1} & \mathrm{Cok}(h) \ar[r] & 0  \\
}
\]
蛇の補題では上の完全列と下の完全列をつなぐ射の存在を示す.
\begin{lem}蛇の補題(snake lemma)\\
上の図式に対し,以下の列が完全になるような射$\delta:\mathrm{Ker}(h) \to \mathrm{Cok}(f)$が存在する.
\[
\xymatrix@1{
0 \ar[r] & \mathrm{Ker}(f)  \ar[r]^{m_0} &\mathrm{Ker}(g) \ar[r]^{e_0} & \mathrm{Ker}(h) \ar[r]^{\delta}
 & \mathrm{Cok}(f) \ar[r]^{m_1} &\mathrm{Cok}(g) \ar[r]^{e_1} & \mathrm{Cok}(h) \ar[r] & 0
}
\]
\end{lem}
(証明)
$K=\mathrm{Ker}(h),k=\mathrm{ker}(h),C=\mathrm{Cok}(f),c=\mathrm{cok}(f)$とする.$k$と$e$の引き戻し$u,k'$を取ると,
$e$はepiより$u$もepi.$m$の$S = \mathrm{Im}(f)$による分解を$m=pq$とすると,命題32より$s$による$p$の分解を得る.
($q$はmonoかつepiなので実は同型射である).下半分についても双対で同様のことを考える.下図はすべて可換になる.
\[
\xymatrix{
S \ar[rd]^p \ar@{.>}[r]^{s} &P \ar@{.>}[d]^{k'} \ar@{.>}[r]^{u} & K \ar[d]^k   \\
X \ar[u]^q_{\cong} \ar[d]^f \ar[r]^m & Y \ar[d]^g \ar[r]^e & Z \ar[d]^h  \\
X' \ar[d]^c \ar[r]^{m'} &Y' \ar[rd]^{q'} \ar@{.>}[d]^{c'} \ar[r]^{e'} & Z'    \\
C \ar@{.>}[r]^{v} &Q \ar@{.>}[r]^{t} & T \ar[u]_{p'}^{\cong} \\
}
\]
$\delta_0 = c'gk'$とする.$u=\mathrm{cok}(s)$かつ$v = \mathrm{ker}(t)$から$\delta_0$は一意的に
\[
\delta_0 = v\delta u : \xymatrix@1{P \ar[r]^u & K \ar[r]^{\delta} & C \ar[r]^v & Q}
\]
と分解できる.この$\delta$が求める射である.\\
メンバ$x \in {}_mK$から$\delta x \in {}_mC$は次の図式の追跡によって得られることを示そう.
\[
\xymatrix{
&&x \ar@{|->}[d] \\
&y \ar@{|->}[r] \ar@{|->}[d] &kx \ar@{|->}[d] \\
z \ar@{|->}[d] \ar@{|->}[r]   &gy \ar@{|->}[r] &0 \\
z_1
}
\] 
$e$はepiなので$y \in {}_mY$が存在して$ey \equiv kx$となる.$e'gy = hkx = 0$と,$Y'$での完全性から
$z \in {}_mX'$が存在して$gy \equiv m'z$.$z_1 = cz$とする.このとき$z_1 \equiv \delta x$となる.これを示そう.\\
まず$ey \equiv kx$と引き戻しの普遍性より,$x = ux_0,y = k'x_0$をみたすメンバ$x_0 \in {}_mP$が
一意的に存在することがわかる.
\[
\xymatrix
{
\nkgr \ar@/_/[rdd]_{y} \ar@/^/[rrd]^{x}  \ar@{.>}[rd]^{x_0}\\
&	P \ar[d]^{k'} \ar[r]^{u} 	& Y \ar[d]^k\\
&	X \ar[r]^{e}			& Z
}
\]
これより
\[
v\delta x \  v \delta ux_0 =\delta_0 x_0 = c'gy \equiv vz_1
\]
$v$はmonoより$\delta x \equiv z_1$となる.この式により,$z_1$の同値類は図式の追跡の際の
メンバの選び方に依存しないこともわかる.この追跡により,$\mathrm{Ker}(h),\mathrm{Cok}(f)$の完全性が示される.
(定理34を使う.頑張って!)
\proofend


\Section{参考文献}
\begin{description}
\item{[1]}S.マックレーン著,三好博之/高木理 訳,『圏論の基礎』,シュプリンガー・ジャパン,2012
\item{[2]}梶浦宏成著,『数物系のための圏論』,サイエンス社,2010
\item{[3]}Albercht Dold,"Lectures on Algebraic Topology",Springer,1980(Reprint)
\item{[4]}Charles A. Weibel,"An introduction to homological algebra",Cambridge University Press,1994
\end{description}

