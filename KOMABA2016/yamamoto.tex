%まず最初に使ったプリアンブルをここに書いてください.
%ただしコンパイルの都合上コメントアウトしてください.
%実際に確認する際は,各自の環境でmain.texにこのプリアンブルを追加してください.

%\usepackage{mathrsfs}
%\usepackage{amsmath}
%\usepackage[all]{xy}
%\newcommand{\proofend}{\begin{flushright} $\blacksquare$ \end{flushright}}
%\renewcommand{\labelenumi}{(\roman{enumi})}
%\newcommand{\nkgr}{・}
%\theoremstyle{definition}
%\newtheorem{theorem}{定理}
%\renewcommand{\thetheorem}{}
%\newtheorem{defi}{定義}
%\newtheorem{thm}[defi]{定理}
%\newtheorem{lem}[defi]{補題}
%\newtheorem{cor}[defi]{系}
%\newtheorem{prop}[defi]{命題}
%\newtheorem{ex}[defi]{例}


\Chapter{円周率$\pi$がひょこっと現れる話(山本)}
\Section{はじめに}

$e, \pi , i$の中で唯一義務教育までで習う数,それが円周率$\pi$です.
$3.14159 \cdots$という並びは皆さんも人生で一度は見たことがあるでしょう.
「円周率」の名の通り,$\pi$という数字は「円周の長さを直径で割ったもの」として定義される,図形由来の数です.円の面積を求めるときだったり,大学受験では回転体の体積を求めるときだったりに現れることが多いですね.
今日はこのように図形的な側面の強い円周率$\pi$が数学のひょんな所にひょこっと現れる話をしたいと思います.

(Caution:当文章においては,「大体どのような感じか」を理解していただくことを重要視するために,積分と無限和の順序を注意なしに交換する箇所が何か所かございます.あらかじめご了承ください.)

\Section{1.三角関数とFourier級数展開}

「$\pi$と関係する関数」と言われて,真っ先に思い浮かぶものは何でしょうか?
高校までの範囲でくくると,おそらく三角関数を連想する人が一番多いのではないでしょうか.
そこで,最初はこの三角関数についてお話しようと思います.

まず三角関数,特に$\sin \theta$や$\cos \theta$のグラフの形を思い出してみましょう.
これらのグラフは「正弦波」と呼ばれる綺麗な形の波になっています.
これは,音叉を叩いたときの音の波形などに現れます.
また,$\sin \theta$や$\cos \theta$は
\[
\sin ( \theta +2 \pi )= \sin \theta, \cos( \theta +2 \pi )= \cos \theta
\]
なる関係を満たしていました.
これは $\theta$ がちょうど$2 \pi$増えたときの関数の値が元のものと同じであること,すなわち$2 \pi$が関数の周期になっていることを意味します.
より一般に,関数$f(x)$に対して$f(x+2 \pi )=f(x)$が成り立つとき,$f(x)$は$2 \pi$を周期として持つといいます.
これに沿えば,正の整数$n$に対して,$\sin (n \theta ), \cos (n \theta)$もまた$2 \pi$を周期として持つことが分かります.

このように,$2 \pi$を周期に持つ関数の簡単な例として三角関数が挙げられます.
これらは比較的簡単な形をしており,解析もしやすいです.
さて実際には周期$2 \pi$の関数はこれだけではないわけで,音や振動を解析する際には単純な正弦波の形をしていないものを対象とする場合がほとんどです.
そのような場合の関数はどのように扱うのがいいでしょうか?

大学1年で習うTaylor展開は,関数をある点の付近で多項式により近似するものでした.
これに倣うと,「一般の関数をより簡単な関数の和で近似する」ことを考えるのがいいかもしれません.

18〜19世紀の数学者Fourierは,「すべての周期関数は,同じ周期を持つ無限個の三角関数の和で表される」という主張をしました.
この主張は元々熱伝導に関する問題を解く際に用られたものであり,主張を認めればその問題の解にたどり着くことができたのでした.Fourierによるこの大胆な主張は,真偽が定かでなかったために数学界に議論を巻き起こしましたが,結果的には「大体」正しい主張であったことが後に分かります.
主張がどこまで正しいのか・また正しいとして,その無限和の収束やふるまいは良いものか?という当時の問いは,その後の解析学,ひいては数学そのものを大きく発展させたと言われています.

さて,Fourier級数展開の具体的な主張を見てみましょう.
\begin{center}
$f(x)$を「性質の良い」周期$2 \pi$の関数とする.
このとき,
\begin{equation}
	f(x)=a_{0}+\sum_{n=1}^{\infty}\left( a_{n}\cos(nx)+b_{n}\sin (nx) \right)
	\label{f1}
\end{equation}
となるような実数$a_n, b_n$($n$は自然数)が存在する.
\end{center}
ここで,「性質の良い」というのは,例えば「定義域全体で微分可能で,さらに導関数も連続」などが相当します.
上に出てきた各係数$a_n, b_n$は,大雑把には次のように計算されます.
まず$a_0$については,(\ref{f1})の両辺を$x$について$0$から$2 \pi$まで積分して
\begin{eqnarray*}
	\int_{0}^{2 \pi}f(x)dx &=& \int_{0}^{2\pi} \left( a_{0}+\sum_{n=1}^{\infty}\left( a_{n}\cos(nx)+b_{n}\sin (nx) \right) \right) dx \\
	&=& \int_{0}^{2 \pi} a_{0} dx +\sum_{n=1}^{\infty} \int_{0}^{2\pi}\left( a_{n}\cos(nx)+b_{n}\sin (nx) \right) dx \\
	&=& 2\pi a_{0}+0=2 \pi a_{0}
\end{eqnarray*}
すなわち
\[
	a_0=\frac{1}{2\pi}\int_{0}^{2 \pi}f(x)dx
\]
を得ます.正の整数$m$に対する$a_m$については,次の公式
\begin{eqnarray*}
	\int_{0}^{2 \pi} \cos (nx) \cos (mx) &=&
	\begin{cases}
		\pi & (n=m) \\
		0 & (n \neq m) \\
	\end{cases}
	\\
	\int_{0}^{2 \pi} \cos (nx) \sin (mx) &=& 0
\end{eqnarray*}
を使えば,次のように計算できます.
(\ref{f1})の両辺に$\cos mx$をかけ,それを$x$について$0$から$2 \pi$まで積分すると
\begin{eqnarray*}
	\int_{0}^{2 \pi} \cos (mx) f(x) dx &=& \int_{0}^{2 \pi} \cos (mx)\left( a_{0}+\sum_{n=1}^{\infty}\left( a_{n}\cos(nx)+b_{n}\sin (nx) \right) \right) dx \\
	&=& \int_{0}^{2 \pi} a_{0} \cos (mx) dx +\sum_{n=1}^{\infty} \int_{0}^{2\pi} \cos (mx)\left( a_{n}\cos(nx)+b_{n}\sin (nx) \right) dx \\
	&=& \pi a_{m}
\end{eqnarray*}
を得ます.すなわち,$m \ge 1$に対して
\[
	a_{m}=\frac{1}{\pi} \int_{0}^{2 \pi} \cos (mx) f(x) dx
\]
となります.同様にして,$b_m$も
\[
	b_{m}=\frac{1}{\pi} \int_{0}^{2 \pi} \sin (mx) f(x) dx
\]
と計算できることになります.これで,係数が計算できました.

それでは,とある関数を実際に三角関数の無限和で表してみましょう.

周期$2 \pi$の関数$f(x)$を,$0 \le x \le 2\pi$において$f(x)=-x(x-2 \pi)$となるように定めます.
これは$x=2\pi n$($n$は整数)において微分可能ではありませんが,先に述べた「性質の良い」関数の1つです.
これを認めれば,積分計算により$a_0, a_n, b_n$を求めてやることで$f(x)$を三角関数に表すことが出来ます.
実際に計算してみると(部分積分を使えば高校生にもできる計算ですので,やってみてください),
\begin{eqnarray*}
	a_0 &=& \frac{2}{3}\pi ^2 \\
	a_n &=& -\frac{4}{n^2} \\
	b_n &=& 0
\end{eqnarray*}
となるので,結局
\[
	f(x)=\frac{2}{3}\pi ^2-\sum_{n=1}^{\infty} \frac{4}{n^2} \cos (mx)
\]
と表すことができました.

さて,上式に$x=0$を代入してみましょう.
すると
\[
	0=\frac{2}{3}\pi ^2-\sum_{n=1}^{\infty} \frac{4}{n^2}
\]
となり,適当に整理すると
\[
	\sum_{n=1}^{\infty} \frac{1}{n^2} = \frac{\pi ^2}{6}
\]
となりました.
これはすなわち,「自然数の二乗の逆数の和が$\pi ^2 /6$になる」ということを意味しています.
左辺は整数に関する基本的な級数になっているわけですが,その値として$\pi$がひょこっと現れるという不思議な公式が出来てしまいました.
この級数は「Basel級数」と呼ばれています.

他にも,Fourier級数展開を使うと導ける級数は色々あります.
関数を色々変えてみて,様々な公式を作ってみるのも興味深いかと思います.

\Section{$\zeta$ 関数,$\Gamma$関数と関数等式}

前章において,自然数の二乗の逆数の和の値に$\pi$が現れることを見ました.
では,「二乗」が「$n$乗」,あるいは実数「$s$乗」と変わった場合の値はどうなるのでしょうか?

そこで,$1$より大きい実数$s$に対して,$\zeta (s)$(ゼータ)という関数を
\[
	\zeta (s) = \sum_{n=1}^{\infty} \frac{1}{n^{s}}
\]
により定義します.
ここで「$1$より大きい実数」と言ったのは,$1$以下の実数では定義式の級数が発散してしまうことを考慮してのことです.
この$\zeta$関数の定義式から,前章のBasel級数は$\zeta (2)$に相当します.
実は,正の偶数$n$に対して,$\zeta (n)$は$\pi ^{n}$と有理数の積の形をしていることが知られています.

「正の偶数」という規則正しい数値を代入すると$\pi$に関係する値を返す$\zeta$関数ですが,この関数まわりで$\pi$が出てくるのは,関数に何かしらの値を代入するときだけではありません.

それを説明するために,$\zeta(s)$とは別の関数として,$s>0$なる実数に対して$\Gamma (s)$(ガンマ)という関数を
\[
	\Gamma (s) = \int_{0}^{\infty} e^{-t}t^{s-1}dt
\]
により定義します.
$\cdots$いきなり積分の式が出てきてしまいましたが,これがどのような関数なのか説明します.
部分積分を行えば,$s>0$なる実数$s$に対して
\begin{eqnarray*}
	\Gamma(s+1) &=& \int_{0}^{\infty} e^{-t}t^{s}dt \\
	&=& \left[ -e^{-t}t^{s} \right]_{t=0}^{\infty} - \int_{0}^{\infty} (-e^{-t})(st^{s-1})dt \\
	&=& 0+s\int_{0}^{\infty} e^{-t}t^{s}dt =s\Gamma(s)
\end{eqnarray*}
となることが分かります.
この$\Gamma(s+1)=s\Gamma(s)$のように,ある種の関数が満たしている等式のことを「関数等式」といいます.
また,とくに
\[
\Gamma(1)=\int_{0}^{\infty} e^{-t}dt=1
\]
となるので,上の関数等式と合わせれば,数学的帰納法により全ての正の整数$n$に対して
\[
	\Gamma(n)=(n-1)!
\]
となることが分かります.
つまり,$\Gamma$関数とは「正の整数でしか定義されなかった階乗の定義域を拡張したもの」ということになります.
ある意味で整数論由来の関数というわけですね.

さて,$\Gamma$関数は階乗の定義域を拡張したものと述べました.
しかし,この関数は更に「ほとんどの複素数」にまで定義域を拡張することができます.
高校までの数学では,関数といえば定義域は実数のものがほとんどだったかと思いますが,大学での数学では定義域を複素数にすることもしばしばあります.

例えば,$y=x^2$という関数は定義域を複素数としても「自然に」定義できますし,$y=1/x$という関数は定義域を「$0$以外の複素数」としてもやはり「自然に」定義できます.
指数関数$e^x$についても,$e^x$をTaylor展開して
\[
	e^x = \sum_{n=0}^{\infty} \frac{1}{n!}x^{n}
\]
という関数として考えれば,やはり複素数全体に定義域を「自然に」拡張することができるでしょう.
これらと同様にして,$\Gamma$関数も「自然な方法により」,定義域を「ほとんどの複素数」とする関数にすることができます.
具体的には,「複素数全体のうち,$0$以下の整数を除いたもの」全体を定義域とすることができます.

$\Gamma$関数についての話が長くなりましたが,ようやく$\zeta$関数の話に戻ります.
$\Gamma$関数同様,$\zeta$関数も「自然な方法で」定義域を拡張することができます.
さらに,この$\zeta$関数の現れる関数等式を得ることもできます.
ここまで出てきた関数を組み合わせて
\[
\Lambda (s)=\pi ^{-s/2} \Gamma\Bigl(\frac{s}{2}\Bigr) \zeta (s)
\]
という関数を定めます.
(ただし,複素数$s$に関して,$\pi ^s = e^{s \log \pi }$と定めると,これは定義域が実数の場合の拡張になっています.)
このとき,この関数について
\[
\Lambda (s)=\Lambda(1-s)
\]
という関数等式を得ることができます.
この等式はすなわち,「$\Lambda (s)$は点$s=1/2$に関して対称な関数であること」を意味する,かなり簡潔かつ綺麗な式になっていることが分かるかと思います.
$\Gamma$関数,$\zeta$関数といういわば「整数論由来の」関数における簡素な関数等式を導くという方向からも,$\pi$が現れてくるわけです.
ちなみに,先ほど定義域を広げることが出来るといった$\zeta$関数ですが,関数等式を用いることにより,負の偶数$n$に対して$\zeta (s)=0$となることが分かります.
それ以外で$\zeta (s)=0$となるような複素数$s$はどのような分布の仕方をしているか?という問いは「Riemann予想」と呼ばれ,最初に提唱されてから150年以上経った今でも未解決です.

\Section{3.Ramanujanと$\pi$}
先の章で,無限和による$\pi$の公式(Basel級数)を1つ見ました.
他にはどのような公式があるのでしょう?そこで,突然ながらこの公式をご覧ください.
\[
\frac{1}{\pi}=\frac{\sqrt{8}}{99^2} \sum_{n=0}^{\infty}\frac{(4n)! (1103+26390n)}{(4^{n}n!)^{4}99^{4n}}
\]
$\cdots$なかなかに面妖な格好をしています.
この公式を見出したのは「インドの魔術師」という異名を持つ,Ramanujanという数学者です.Ramanujan本人はこの公式を証明したわけでなく,実際に証明されたのは提唱されてから50年以上経ったころのことだそうです.
にもかかわらず,このようにかなり複雑な公式をRamanujanはいきなり発見したのですから驚きです.
これ以外にも,Ramanujanは$\pi$に関していくつかの公式を発見しています.

また,Ramanujanは$\pi$に関する公式以外にも様々なことをやっています.
例えば,次の関数をご覧ください.
\[
  \Delta (q)=q(1-q)^{24}(1-q^{2})^{24} \cdots = q\prod_{n=1}^{\infty}(1-q^n)^{24}
\]
これはRamanujanの$\Delta$(デルタ)関数と呼ばれるものです.
$\Delta$関数を$q$に関して「展開」すると,
\[
	\Delta (q)=q-24q^{2}+252q^{3}-1472q^{4}+\cdots
\]
のようになります.
これは$q$に関する多項式のようなもの(正式にはべき級数といいます)になっているわけですが,これの$q^n$における係数を$\tau (n)$とおきます.
すなわち:
\[
	\Delta (q)=q\prod_{n=1}^{\infty}(1-q^n)^{24} = \sum_{n=1}^{\infty} \tau (n) q^{n}
\]
です.
この関数$\tau$を「Ramanujanの$\tau$(タウ)関数」といいます.
また,$\Delta(q)$の定義式の$q$として$q=e^{2\pi iz}$を代入することができ,そうすると
\[
	F(z)=\sum_{n=1}^{\infty} \tau (n) q^{n}=\sum_{n=1}^{\infty} \tau (n) e^{2 \pi inz}
\]
は虚部が正である複素数全体を定義域とする関数となります.
$\Delta$および今作った$F$は,次の性質を満たします.
(3つ目の性質を証明することは若干難しいです.)
\begin{itemize}
	\item $\Delta (q)$は$q$に関するべき級数
	\item $F(z+1)=F(z)$
	\item $F(-1/z)=z^{12} F(z)$
\end{itemize}
この性質により,$F$および$\Delta$は「重さ$12$の保形型式」であるといいます.
さらに,
\begin{itemize}
	\item $\Delta(0)=0$
\end{itemize}
が成り立つことにより,$F$および$\Delta$は「重さ$12$のカスプ型式」であるといいます.
保形型式・カスプ型式は整数論的にも歴史のある関数です.
さて,実際に$\tau (n)$を計算してみると,こんな感じになります.
(計算力に自信のある人は試してみてください.)
\begin{eqnarray*}
\tau(1)=1, \tau(2)=-24, \tau(3)=252, \tau(4)=-1472 \\
\tau (5)=4830, \tau(6)=-6048, \tau(7)=-16744, \tau(8)=84480 \\
\tau(9)=-113643, \tau(10)=-115920, \tau(11)=534612, \tau(12)=-370944
\end{eqnarray*}
これらの数字を見て何か気付くことはあるでしょうか?
もし即答できたら,あなたもRamanujanになれるかもしれない!?

実際のところ,Ramanujanはおおよそ次のようなことに気付きました:
\begin{itemize}
	\item 互いに素な整数$n,m$に対し$\tau(n)\tau(m)=\tau(nm)$
	\item 素数$p$,正の整数$n$に対し$\tau(p^{n+1})=\tau(p) \tau(p^{n})-p^{11}\tau(p^{n-1})$
\end{itemize}
$n,m$を小さめの数字にして確認してみると,
\begin{eqnarray*}
	\tau(4)=-1472=(-24)^2-2^{11} \cdot 1=\tau(2)\tau(2)-2^{11}\tau(1) \\
	\tau(6)=-6048=-24 \cdot 252 = \tau(2) \tau(3)
\end{eqnarray*}
となり,なるほど確かにそうなっているように思えます.
そしてこの考察は的中していたことが後に証明されました.
Ramanujanの洞察力の凄まじさを思い知らされます.

さて,前章でRiemann予想について少しだけ触れましたが,$\zeta$関数を考察する一つの方法として,$\zeta$関数を単体で見るのではなく,何らかの関数のクラスの$1$つであると見てやるというものがあります.
そのような「関数のクラス」として,保形型式から得られる「保形$L$関数」というものがあります.
今回は,$\Delta$関数から保形$L$関数を作り,その中でやはりひょっこり$\pi$が登場することを見たいと思います.

といっても定義自体は簡単で,保形$L$関数$L_{\Delta}(s)$は
\[
	L_{\Delta}(s)=\sum_{n=1}^{\infty}\frac {\tau (n)}{n^s}
\]
により,実部が十分大きい複素数$s$で定義することができます.$\zeta$関数の定義式を少し変えただけですね.
そうなるとやはり,$\zeta$関数と似た性質を持つことが期待されます.

ここで突然ですが
\[
\Lambda (s)=\int_{0}^{\infty}F(iy)y^{s-1}dy
\]
という値を考えてみます.
$F(iy)$が$y \rightarrow \infty$で「非常に速く」$0$に収束すること,および先に挙げた$F$の性質から$F(i/y)=y^{12}F(iy)$となることを使えば,$\Lambda(s)$は全ての複素数$s$で値を持つことが分かります.
また,
\[
	F(iy)=\sum_{n=1}^{\infty}\tau(n) e^{-2\pi yn}
\]
だったことを思い出せば,$\Lambda(s)$は次のように計算できます:
\begin{eqnarray*}
	\Lambda (s) & = & \int_{0}^{\infty}F(iy)y^{s-1}dy \\
	& = & \int_{0}^{\infty}\left(\sum_{n=1}^{\infty}\tau(n) e^{-2\pi yn} \right) y^{s-1}dy \\
	& = & \sum_{n=1}^{\infty}\tau(n) \int_{0}^{\infty}e^{-2\pi yn}y^{s-1}dy
\end{eqnarray*}
ここで
\[
	\int_{0}^{\infty}e^{-2 \pi yn}y^{s-1}dy
\]
において,変数変換$t=2 \pi ny$により,
\begin{eqnarray*}
	\int_{0}^{\infty}e^{-2 \pi yn}y^{s-1}dy & = & (2 \pi n)^{-s}\int_{0}^{\infty}e^t t^{s-1}dy \\
	&=& (2\pi)^{-s} \times \Gamma(s) \times n^{-s}
\end{eqnarray*}
となります.よって,
\begin{eqnarray*}
	\Lambda (s) &=& \sum_{n=1}^{\infty}\tau(n)(2 \pi)^{-s}\Gamma(s)n^{-s} \\
	&=& (2 \pi )^{-s}\Gamma(s)L_{\Delta}(s)
\end{eqnarray*}
となっていることが分かります.
よく分からない積分の式から$L_{\Delta}(s)$が現れ,また再び$\pi$がひょこっと出てきました.
$\Lambda(s)$が全ての複素数に対して定義できていたことから,
\[
	L_{\Delta}(s)=(2 \pi)^s \Lambda(s)/ \Gamma(s)
\]
により$L_{\Delta}(s)$の定義域を複素数全体に拡張することができます.
これが,$\Lambda(s)$なるものを考えた理由の1つです.
$\Lambda(s)$を考えた理由はもう1つあります.
$\Lambda(s)$の定義式にある$y$について,$t=1/y$と積分変換すると,
\begin{eqnarray*}
	\Lambda(s) &=& -\int_{\infty}^{0}F(i/t)t^{-(s-1)}\frac{dt}{t^2} \\
	&=& \int_{0}^{\infty}t^{12}F(it)t^{-(s+1)}dt \\
	&=& \int_{0}^{\infty}F(it)t^{(12-s)-1}dt \\
	&=& \Lambda(12-s)
\end{eqnarray*}
となることが分かります.
これにより,$L_{\Delta}$に関する関数等式
\[
	(2 \pi)^{-s}\Gamma(s)L_{\Delta}(s)=(2 \pi)^{12-s}\Gamma(12-s)L_{\Delta}(12-s)
\]
が導かれました.
$\zeta$関数のとき同様,底を$\pi$の何乗かとする指数関数を掛け合わせることによって,かなり綺麗な形の関数等式を得ることができました.

\Section{おわりに}
今回は$\pi$が「ひょこっと」現れる話ということで,図形的な概念でないところから$\pi$が出現するというものをいくつか挙げてみました.
(後半はかなり解析的整数論の話になってしまいましたが(汗))
今回挙げたもの以外でも$\pi$が現れるところは色々ありますし,また$\pi$でなくても,全く関係がないと思われていた分野に別概念がいきなり現れるということはしばしばあります.
皆さんも,興味があればそういったものを探してみてはいかがでしょうか?
