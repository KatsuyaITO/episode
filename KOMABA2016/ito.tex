%まず最初に使ったプリアンブルをここに書いてください.
%ただしコンパイルの都合上コメントアウトしてください.
%実際に確認する際は,各自の環境でmain.texにこのプリアンブルを追加してください.

%\usepackage{mathrsfs}
%\usepackage[all]{xy}
%\newcommand{\proofend}{\begin{flushright} $\blacksquare$ \end{flushright}}
%\renewcommand{\labelenumi}{(\roman{enumi})}
%\newcommand{\nkgr}{・}
%\theoremstyle{definition}
%\newtheorem{theorem}{定理}
%\renewcommand{\thetheorem}{}
%\newtheorem{defi}{定義}
%\newtheorem{thm}[defi]{定理}
%\newtheorem{lem}[defi]{補題}
%\newtheorem{cor}[defi]{系}
%\newtheorem{prop}[defi]{命題}
%\newtheorem{ex}[defi]{例}


\Chapter{ネイピア数eについて(伊藤)}
\Section{はじめに}
\begin{flushright}
\textit {〜「私のことは嫌いでも、eのことは嫌いにならないでください!」〜}\\
国民的アイドルグループEXP271のとあるセンターの言葉
\end{flushright}

$e^{\pi i}sode$(えぴそーど)という名前の冊子を先輩たちから私達が受け継いでから,はや1年半が経ちましたが,未だに「$e^{\pi i}sode$?これはなんて読むの?」
と聞かれることも多いです.上記のような発言から察するに,$e^{\pi i}$にこめられた意味も理解されていないでしょう.
そこで改めて,{\bf e}という数字について色々とお話をしたいと思います.\par
$e$という数字は{\bf ネイピア数}と呼ばれ,日本語では自然対数の底とも呼ばれたりします.
数なのでちゃんと値があって,$e = 2.718281828459045235360287471352\cdots$という永遠につづく小数として表されます.
ネイピア数という名前は,この数を初めて発表したジョン・ネイピア(John Napier)という数学者にちなんでつけられました.
ネイピアがこの数を考案した1618年というのは,日本で言うと1615年は大阪夏の陣で徳川家康
\footnote{すごくどうでも良い話ですが,徳川家康を知らないと日本では非常識人呼ばわりされますが,ネイピア数eを知らない人に対して非常識だ!という石を投げる人はそこまでいないでしょう.残念です.} と豊臣秀頼が合戦を繰り広げていた年ですし,
ヨーロッパではまだ魔女狩りが勃発していたような時代です.
万有引力の法則を発見したアイザック・ニュートンはまだ生まれてもいませんし,ビブン,セキブンという概念・言葉すら生まれていません.
何が言いたいかと言いますと,「$e$は数学だけの世界に出てくる小難しい数ではなく,古くから人類と共にあった誰でも理解できるような身近なもの」ということです.\par
ということで,出来るだけ皆さんに分かりやすい様に$e$についてお話をしたいと思います.もし本当は身近なはずの$e$がこの冊子を読んで,
また遠くに感じられてしまったらそれは,$e$が悪いのではなく,私が悪いのです.

\Section{身近なe}
\begin{flushright}
\textit {〜「借りた金は忘れるな。貸した金は忘れろ。」〜}\\
田中角栄
\end{flushright}

$e$は身近なところから発生します.とりあえず結論だけ先に述べて解説に移ります.
\begin{center}
連続的に複利で利息を支払うような年利率$1$の銀行が存在するとき,この銀行の実質的な年利率は$e$となる.
\end{center}
という形で$e$は身近に発生します.ここでキーとなる{\bf 金利}という概念です.金利は現代人にとってある意味遠い世界にあるものかもしれないので
一度説明しておきます.例えば,あなたが銀行にいくらかのお金を預けているとします.
\footnote{ここで「預けている」という言葉を使うと,まるで厳重な鍵付きの金庫やロッカーにお金を置いてきたように聞こえますが,そのようなことはありません.銀行も企業とは言え,その銀行の中にいるのは人間ですので,あなたのお友達の山田さんの家のタンスに「お金〜〜万円預けるね.なくしたら承知しないよ.」と言ってお金を置いてきたのと,解釈によっては変わりはないのです.}そうすると,銀行は「私達の所にお金を預けてくれてありがとう」と言って年に何度か少しだけあなたの口座にお金を振り込んでおいてくれます.これを{\bf 利子}と言います.\footnote{今の日本の銀行の通常預金の金利は0.001%ぐらいです.つまり,100万円預けて置くと年に10円ほどもらえる事になります.}ここで{\bf 金利(利率)}という概念が発生します.金利とは,「あなたが銀行に預けているお金」と「利子」の比率を金利と言います.数式で書くと
\[
\mbox{預金額}\times\mbox{金利} = \mbox{利子}
\]
となります.\footnote{ここまでの内容は,できれば読まなくても知っているぐらいのレベルであって欲しいです.}\par
ここからは,簡単のために全て年利率を{\bf 1%}の場合にのみ限定して話をすすめましょう.\\
もし,この世に年率$1$%を謳う銀行がいくつかあったとしましょう.これらは,皆同じに見えますが,実は違う可能性があります.
それは {\bf 年に何回利子が払われるかが分かっていない} ということです.例えば,
\begin{center}
100万円につき,年に一度だけ,1万円の利息がもらえる.
\end{center}
という銀行と,
\begin{center}
100万円につき,年に2度,5千円づつ利息がもらえる
\end{center}
という銀行は1年というスパンで見れば,おなじ年利率$1$%の銀行です.しかし,ここで違ってくるのが{\bf 複利}という考え方です.\par
(ここに複利の図を入れたい)
複利とは,今までもらった利息を預金額に繰り入れて,利息を払ってくれる方式です.例えば,100万円を(利息年1回払いの)年利息1%の銀行に$4$年間預けておくと
\begin{eqnarray*}
100\mbox{万円} \times 1\mbox{%}  + 100\mbox{万円} &=& 101\mbox{万円}\\
101\mbox{万円} \times 1\mbox{%}  + 101\mbox{万円} &=& 102.01\mbox{万円}\\
102.01\mbox{万円} \times 1\mbox{%}  + 102.01\mbox{万円} &=& 103.0301\mbox{万円}\\
103.0301\mbox{万円} \times 1\mbox{%}  + 103.0301\mbox{万円} &=& 104.0604\mbox{万円}
\end{eqnarray*}
というように雪だるま式にお金が増えていきます.最初に考えた1年間の利払回数が違う場合についても同様のことがいえます.
\begin{center}
100万円につき,年に2度,5千円づつ利息がもらえる
\end{center}
は,
\begin{center}
$1$%$\div 2 = 0.5$%づつ半年に1回お金が増える
\end{center}
ということになります.そして,いくらになるかとか言いますと,

\begin{eqnarray*}
100\mbox{万円} \times 0.5\mbox{%}  + 100\mbox{万円} &=& 100.5\mbox{万円(=半年後の預金残高)}\\
100.5\mbox{万円} \times 0.5\mbox{%}  + 100.5\mbox{万円} &=& 101.0025\mbox{万円(=1年後の預金残高)}
\end{eqnarray*}
という風に$101$万円よりも,わずか$0.025$万円だけ増えました.つまり違う利息額となったわけです.\\
では,更に細かくわけて,$1$ヶ月に$1$回.年$12$回の利息が受け取れる銀行があったとしましょう.この場合は
\begin{center}
$1/12 \fallingdotseq 0.083$%づつ1ヶ月に1回お金が増える
\end{center}
ここで,今までのように$12$回計算をしても良いのですが,少し落ち着いて見てみると,\\
例えば$100$万円が$1$%増えるとその後どうなるかと言うのは,
\[
100\mbox{万円} \times 1\mbox{%}  + 100\mbox{万円} = 100\mbox{万円} \times (1 + 0.01) = 101\mbox{万円}\\
\]
という式で計算できるということがわかります.また,これを$2$回払いの式に応用すると
\[
100\mbox{万円} \times (1 + 0.005) \times (1 + 0.005) =  101.0025\mbox{万円}
\]
という風になります.同様に,$12$回払いの場合も
\[
100\mbox{万円} \times (1 + \frac{0.01}{12}) \times  \cdots \times (1 +  \frac{0.01}{12}) = 100\mbox{万円} \times (1 + \frac{0.01}{12})^{12} = 101.004596089
\]
という結果が得られます.\footnote{電卓で計算する場合は $1.00083333 \times\  = \ = \ = \ = \cdots $と$=$ボタンを連打すると計算できます}
また数字に着目すると,利息$12$回払いのときのほうがやはり僅かにお金は多くなっています.\\
では,もっとお金を増やしたい!!ということで,$1$日に$1$回利息が振り込まれるような銀行を考えてみるとどうでしょうか,
\[
100\mbox{万円}\times (1 + \frac{0.01}{365})^{365} = 101.005003 \mbox{万円}
\]
となってやはり,いままでよりも僅かに増えています.では,この{\bf 利払い回数をどんどん増やしていくと億万長者になれるのでは!!!}
\footnote{少し関連した話として,{\bf 72の法則}と言うものが有ります.これは,6%複利でお金を運用すると約12年で2倍になる,8%複利でお金を運用すると約9年で2倍になるというように
複利の%数と二倍になるまでの年数の積がおおよそ72になっているという法則です}
と思ってしまうわけです.例えば,年に$10000$回利払いがされるような銀行があったとしたら
\[
100\mbox{万円}\times (1 + \frac{0.01}{10000})^{10000} = 101.005016 \mbox{万円}
\]
となりますが,よく見てみると,そこまで増えていません.どうやら上限があるようです.そしてこの上限が$e$なのです.
ここで
\[
e^{0.01} = (2.718281828459045235360287471352 \cdots )^{0.01} = 1.01005016708
\]
という数値と見比べて見ましょう\footnote{流石にこれを普通の電卓で計算すわけにはいきませんので,関数電卓で計算するかGoogleで「e\^{}0.01」と検索してみてください}.
そうすると,実は$100$万円$\times e^{0.01} =  101.005016708$万円は今まで出てきた数字に非常に近い数字になっています.\par
つまり,経験的に,
\begin{center}
年利率$1$%の銀行の利払い回数をどんどん大きくしていくと,$1$年トータルでみたときには$e^{0.01}$という利率に近づく.
\end{center}
ということがわかります.これを高校数学の言葉で,{\bf $e^{0.01}$に収束する}と言います.こうして,$e$という数字が簡単な金利計算から出て来るということがわかりました.
この利率は,その瞬間瞬間に利息が発生し,それが預金に繰り入れられているという意味で{\bf 連続複利}と言われます.\par
ここで,一連の議論の結果をまとめて,$e$という数を改めて定義すると
\[
e := \lim_{h \to \infty} \biggl( 1 + \frac{1}{n}\biggl)^n
\]
という風になります.
\begin{table}[hbtp]
  \caption{利率がeに収束していく}
  \label{table:data_type}
  \centering
  \begin{tabular}{c|c}
    \hline
    利払い回数  & 実際の年利率  \\
    \hline \hline
    1回  & 1% \\
    2回  & 1.0025% \\
    12回  & 1.004596089% \\
    365回  &  1.005003%\\
	\vdots & \vdots \\
	無限回 & $e^{0.01}$倍 = 1.005016708%増加\\
    \hline
  \end{tabular}
\end{table}

\Section{まとめ〜eは本当に身近なのか〜}
\begin{flushright}
〜「A bird in the hand is worth two in the bush.」 〜\\
(手に持っている1羽の鳥は、まだ手にしていない茂みの中の2羽の鳥と同じ価値があるということわざ)。
\end{flushright}

利払い回数をどんどん大きくしていくと,やがては$e$に近づくということがわかりましたが,実際にそんなに何度も利払いが行われることなんてあるのだろうかと思う方もいらっしゃると思います.
しかし,1つ言えることは,$e^{利率}$は$1$日複利とほとんど差はないということです.そして,$e$という数字の計算上の便利さから,金融などの分野では$e$を使った利率計算が行われています.
そのことについて触れて,この記事を終えたいと思います.\par
{\bf どうして,金融機関では$e$を使う必要があるのか}というと,まず一つには$e$を使った利率計算が,数学的に相性が良いということがあります.\\
例えば,最初の年に$12$回の利払い$1$%,次の年に$6$回の利払い$2$%,そのまた次の年に$10$回の利払い$3$%の利率でお金を運用したときに,その利息の計算は
\[
(1+\frac{0.01}{12})^{12} \times (1+\frac{0.02}{6})^6 \times (1+\frac{0.03}{10})^{10}
\]
となり,いちいち全てを計算する必要があります.また,数学的に見てもこの式は複雑です.しかし,すべて連続複利であるとみなすと,
\[
e^{0.01} \times e^{0.02} \times e^{0.03} = e^{0.01+0.02+0.03}
\]
という風にシンプルな式にすることが出来ます.\footnote{全く関係がない話ですが,僕がお気に入りの問題として,「$e^\pi$に最も近い整数を電卓なしで計算せよ」という高校数学の問題が有ります.暇な方はやってみてください.もっと関係のない話として,$\pi^e,e^e,\pi^\pi$は全て有理数かすら分かっていませんが,$e^\pi$は超越数であることが分かっています.}またこの記事では述べませんが,この$e$の何々乗という数字はとても微分積分などの数学的な操作と相性が良いです.\par
最後に,そもそも全てのお金を銀行に預けているわけでもないのに,なぜ金融機関がこのようなことをしないといけないのかということについて考えてみましょう.これは,{\bf 今手にもっている100万円と 来年の100万円は当価値ではない}ということに由来しています.なぜならば,今の$100$万円を仮に銀行に預けたとしたら,$100$万円プラス利子がついているからです.
このようにして,金融機関では毎年のようにお金の出入りがありますから,それを一度全て現在の価値に換算して計算する必要があります.\par
「と言っても,今時超低金利だし関係ないんじゃないの」と思う方もいらっしゃるかもしれません.しかし,例えば生命保険や年金は非常に長期のお金の出入りがあります.
よってその積み重ねは大きく,少しでも金利が動いただけで,保険料や年金の掛金に大きな影響を影響をあたえることがあるのです.
\footnote{例えば,前年度のゼロ金利政策が日本銀行によって発表されたとき,多くの積立式の保険商品が販売をやめざるを得なくなったという事実からも分かると思われます}
故に,($e$を用いて)その会社のお金の出入りを管理していくことは非常に重要です.
\footnote{1つお詫びをしなければならないのは,eを用いた現在価値計算が行われるのはどちらかと言うと,数学色の濃い金融派生商品などの分野で,本文中で例として挙げた保険会社や年金などでは,一応「利力」という名前でもって認識はされていますが,少々数学との相性が悪くても,普通の複利計算でゴリ押ししてまうところがあります.}
またこれは,$e$が使われているほんの一例にすぎず,数学が絡む殆どの分野で$e$が使われているということを最後に注意しておきたいと思います.

\Chapter{eと微分方程式の話〜解こう!微分方程式!〜}
\Section{はじめに}
$e$という数は
\[
e := \lim_{h \to \infty} \biggl( 1 + \frac{1}{n}\biggl)^n
\]
という形で定義されるのでした.この$e$についての最も大事な性質は次のものです.
\[
\frac{d}{dx} e^x = e^x
\]
これは微分積分学の言葉ですが,{\bf $e^x$という関数は,微分しても変わらないということ}です.
世の中にはたくさんの関数が有りますが,{\bf 微分して変わらない関数は$e^x$だけ}\footnote{この主張は厳密に言うと正しくないのですが,後で厳密に述べるので許して下さい}です.
つまり,微分積分という概念が生まれる前にできた$e$ですが,それは微分積分学の中心にあるものであり,
とても微分や積分と相性が良いわけです.この記事では,その$e$と微分積分,微分方程式の関わりについて紹介します.
微分方程式というのは「微分」と「方程式」という険しい言葉が2つも並んだものですが,
とても有用であり,世の中の物理現象などの多くは微分方程式で記述されるぐらいに重要な概念です.
この記事では,まず$e$と微分の関係について述べ,その後微分方程式について基本的なことを説明し,最後に幾つか実際に微分方程式を解くということをします.
\Section{微分方程式について}
\Subsection{e再考}
\defb
\[
e = \lim_{n \to \infty} \biggl( 1 + \frac{1}{n}\biggl)^n
\]
として$e \in \R$を定義する.
\defx

\thm
\[
\frac{d}{dx} e^x = e^x
\]
\thmx
この証明は,高校数学の教科書に譲ることとしましょう.\footnote{正しいことを書きたいという数学科生としてのプライドのようなものが私に証明を書かせませんでした.}
\thm
\[
\frac{d}{dx} f(x) = f(x) ,\ f(0) = 1
\]
を満たすような微分可能な関数$f:\R \to \R$はただ一つだけ存在する.\footnote{数学科4年生の大澤くん作詞の歌に「ただひとつ.示すは難し.」という歌詞があります.一意性を示すことの難しさを歌った歌です.詳しくは係まで.}
\thmx
\Subsection{微分方程式入門}
ここで,この$2$つの定理を合わせると,次のような考察ができます.まず,
$e^x$は$\frac{d}{dx}f(x) = f(x)$と$f(0)=1$を満たしています.かつ.このような関数は唯一つしかありません.
よって,$\frac{d}{dx}f(x) = f(x)$と$f(0)=1$を満たせば,直ちにその$f(x) = e^x$ということが導かれます.\\
ここで,{\bf 微分方程式}というものについてすこし紹介します.\\
\[
\frac{d}{dx} f(x) = f(x) ,\ f(0) = 1
\]
という式を見てみると,右には微分が入っています.このように微分演算が入っているような方程式を{\bf 微分方程式}と言います.\\
ある微分方程式を満たすような,関数を見つけることを{\bf 微分方程式を解く}と言います.\\
そして, $f(0) = 1$というのは,最初の条件を指定しているのです.このような条件を{\bf 微分方程式の初期条件}と言います.\\
つまり,我々が今したこと用語でまとめると,
\begin{center}
$\frac{d}{dx} f(x) = f(x) ,\ f(0) = 1$という初期条件の課された微分方程式解くと$f(x) = e^x$が得られた.
\end{center}ということです.
\Subsection{一意性について}
{\bf ある微分方程式の解が一意的に存在する}とは,唯一つしかその微分方程式に対して解が存在しないということです.数式で書くと,$f,g$という関数が微分方程式の解ならば,$f=g$ということです.\\
ここで「そんな一意性なんて考えて何になるんだ」と思う方がいらっしゃるかもしれませんが,一意性は微分方程式にとってとても大事な概念です.次のような例を考えてみましょう.
\ex
物理を習ったことがある人ならばピンとくるかもしれませんが,ある地点$O$からボールを初速度$v_0$で投げると,
ボールは次のような微分方程式で定義される軌道を描いて運動します.
\[
m \frac{d^2 \bm{x} }{dt^2}  = -m \bm{g}, \frac{d  \bm{x}}{dt} = \bm{v}_0 , \bm{x}_0 = 0 
\]
ここで,この微分方程式に一意性がなかったらどうなるでしょうか.\\
つまり異なる解が幾つかあるわけです.つまり,{\bf 全く同じ条件で同じ場所から同じボールを投げても,そのボールがどのような軌道を描くかは予測できない}
という状態が発生するわけです.これに関する未解決問題として.
\begin{eqnarray*}
\frac{\partial}{\partial t}u_i + \sum_{j=1}^n u_j \frac{\partial u_i}{\partial x_j}
&=& \nu \Delta u_i - \frac{\partial p}{\partial x_i} + f_i(x,t)\\
\nabla \cdot u &=& 0\\
u(x,0) &=& u_0(x)  , \ (x\in \R^n, t \ge 0)
\end{eqnarray*}

という微分方程式で定義されるナビエ-ストークス方程式という物があります.
この方程式は気体や液体の中での運動を記述する流体力学の基本的な方程式ですが,この方程式の解の滑らかさや解の存在や一意性については知られていません.そして,この未解決問題を解くと{\bf 100万ドル}がクレイ数学研究所から貰えます.もし,ナビエ・ストークス方程式の解が一意的でなかったり滑らかでなかったりすると,同じ条件で飛行機を飛ばしても同じように飛んでくれなかったり,滑らかに移動してくれないつまりカクカクに移動するという状況が考えられます.
\exx
一方で一意性は,微分方程式を解くという立場においてはとても便利だったりします.なぜならば,もし一意性が保証されていたならば,1つでも解を見つけてしまえば,それ以外に解を探す必要はなくおしまいなわけです.今回の$e^x$という微分方程式でも一意性により,これ以外は無いことがわかりました.
\footnote{高校数学でも少しだけ微分方程式について教えられましたが,この一意性について言及しないがゆえに,とりあえず解けてしまったが本当にこれだけか分からないということに私は陥りました.}
\Subsection{微分方程式を解こう!}
前置きが長くなりましたが,これから幾つかの微分方程式を解いてみましょう.
\ex
\[
\frac{dy}{dx} = y ,\ y(0) = 1
\]
これは解けることを祈っています.$y$という未知の関数を数式できちんと書いてあげれば微分方程式を解けたことになります.\\
そうですね,$y = e^x$となります.
\exx
\ex
\[
\frac{dy}{dx} = y 
\]
今度は初期条件がなくなっています.実はこのときは,微分方程式の解は一意的に存在しません.\\
実際, $y = C e^x ,\ C$は定数とすると,これは微分方程式の解になっています.\\
ここで厳密にではないですが,有用な解き方を書きます.\\
\[
\frac{1}{y} dy = dx
\]
と微分を分数の様に見て移行します.そして両辺を積分します.
\[
\int \frac{1}{y} dy = \int dx
\]
故に,これらの積分を実際に計算して,
\[
\log y = x + c
\]
\[
y = e^{x+c} = C e^x
\]
という風に解を得ることが出来ました.
\exx
\ex
\[
\frac{dy}{dx} = 2y ,\ y(0)=1
\]
今度は$y$に係数があります.ここで,
\[
\frac{d}{dx}f(g(x)) = f'(g(x)) g'(x)
\]
という理系の高校3年生だけが習う公式を紹介しておきます.例えば,
\[
\frac{d}{dx}(5x+2)^2 = 2(5x+2) *5 = 10(5x+2)
\]
ただし,$f(x) = x^2 ,\ g(x) = 5x+2 ,\ f'(x) = 2x ,\ f'(g(x)) = 2(5x+2) , \ g'(x) = 5$でした.
ここで,$y=e^{2x}$を微分してみます.
\[
\frac{dy}{dx}  = \frac{d}{dx}e^{2x} = e^{2x} * 2 = 2 e^{2x} = 2y
\]
ただし,$f(x) = e^x ,\  g(x) =2x ,\  f'(x) = e^x ,\ f'(g(x)) = e^{2x}\ g'(x) =2 $でした.
となり,解になっています.初期条件も満たしています.\\
高校$3$年生の内容を引用しましたが.とりあえず,$a$を定数として,$e^{ax}$を微分すると,
\[
\frac{d}{dx}e^{ax} = ae^{ax}
\]
と$a$だけ前に出てきて,$\frac{dy}{dx} = ay$の解になっていることを覚えておいてください.
\exx
\ex
\[
\frac{d^2 y}{dx^2} - 5 \frac{dy}{dx} + 6y = 0
\]
今度は$2$回微分する用になっています.これは,ここで試しに$y=e^{x}$を入れてみると,
\[
\frac{d^2 y}{dx^2} - 5 \frac{dy}{dx} + 6y =  e^x - 5e^x + 6 e^x = 2e^x
\]
となって$0$にはなってくれませんが,当たらずといえども遠からずという感じですね.今度は$y=e^{2x}$を入れてみます.

\[
\frac{d^2 y}{dx^2} - 5 \frac{dy}{dx} + 6y =  \frac{d}{dx}(2e^{2x})  - 5 \cdot 2e^{2x} + 6 e^{2x} = (4-10+6)e^{2x} =0
\]
といって,$0$になってくれました.また,$y=e^{3x}$を入れてみます.
\[
\frac{d^2 y}{dx^2} - 5 \frac{dy}{dx} + 6y =  9 e^{3x} - 15 e^{3x} + 6e^{3x} = 0
\]
またしても解になってくれました.一方で,$y=e^{2x} + e^{3x}$や$y = 3e^{2x} - 2e^{3x}$なども解になっています.実際,
\[
 \frac{d^2 y}{dx^2} - 5 \frac{dy}{dx} + 6y = (4-10+6)e^{2x} + (9 -15 + 6)e^{3x} = 0
\]
\[
 \frac{d^2 y}{dx^2} - 5 \frac{dy}{dx} + 6y = 3(4-10+6)e^{2x} - 2(9 -15 + 6)e^{3x} = 0
\]
どうやら$ae^{2x} + be^{3x}$は全部解になってくれています.一方で全ての解はこれだけで表されるのでしょうか.少し不安が残ります.
この記事はこの方程式の解が,$ae^{2x} + be^{3x}$で表されることを示して終わります.
\exx
まず事実として,
\thm
\[
\frac{d^2 y}{dx^2} - 5 \frac{dy}{dx} + 6y = 0
\]
は初期値について$2$つ条件を与えると,解が一意的に存在する
\thmx
ということを認めます.\\
\proof
ここで,$z$という関数が微分方程式の解になったとします.この$z$が,$ae^{2x} + be^{3x}$で表されることを証明します\\
$z(0) = z_0 ,z(1) = z_1$とおきます.そして,
\begin{equation*}
\begin{cases}
z_0    = 1\cdot\alpha + 1 \cdot \beta   \\
z_1    = e^2 \cdot\alpha + e^3 \cdot\beta \\
\end{cases}  
\end{equation*}
という連立方程式を解いて$\alpha,\beta$を求めます.そうして,$y = \alpha e^{2x} + \beta e^{3x}$とおくと,
\begin{eqnarray*}
y(0) &=& \alpha e^{2\cdot 0} + \beta e^{3\cdot 0}  = 1\cdot\alpha + 1 \cdot \beta  = z_0\\
y(1) &=& \alpha e^{2\cdot 1} + \beta e^{3\cdot 1}  =e^2 \cdot\alpha + e^3 \cdot\beta = z_1\\
\end{eqnarray*}
となり,これは初期値$y(0) =z_0 , y(1) =z_1$を満たすような微分方程式の解になります.よって,事実として認めた解の一意性から,$y=z$となり全ての解は$y = \alpha e^{2x} + \beta e^{3x}$として表されることが証明されました.
\proofx
\Section{まとめ}
このようにして,$e$は微分積分学の中でも基本的な存在であり,$e$を使うと色々な微分方程式を示すことが出来ました.
微分方程式によって,物体や波や電気や音などの運動を記述することができるので,とても有用です.
そして,高校数学ではあまり触れられませんが{\bf 解の一意性}というのは,数学的にも実際微分方程式を解く上でも重要であることを示しました.
そして最後に出てきた,微分方程式について1つ数学的に触れておきたい事がありますが,
\[
f,g \mbox{が微分方程式の解} \Rightarrow \alpha f + \beta g \mbox{も微分方程式の解.}(\mbox{ただし}\alpha,\beta \in \C)
\]
が成り立っているような,微分方程式を{\bf 線形微分方程式}と言います.この線形という性質はとても重要な性質で数学のどこにでも現れるので覚えておいてください.
最後に,さらなる応用として,微分方程式を形式的に,
\[
dy = 2x dx 
\]
のように書くことが出来ます.これは,$x$が$dx$(少し)だけ増えると,$y$は$dy = 2xdx$だけ増えることを示しています.
これは,最初の状態を決めると解の一意性より,全ての挙動が決まってしまいますが,そこにある程度のランダムさを加えた{\bf 確率微分方程式}という物があります.例えば.
\[
df_t = af_t dt + bf_t dW_t
\]
というのは,ブラック・ショールズ方程式という有名な方程式ですが,これは,$t$が$dt$だけ増えると,$f_t$は確実に,$af_t dt$だけ増え,またさらに$b f_t$の分散を持って
増減します.(つまり,増える可能性もありますし,減る可能性もあります.)このように,ある程度の動きは決定されているが,一方でランダムさをも抱えているようなモデルを表現することができ,
非常に多くの現実世界の現象を記述することができます.\footnote{例えば,この会社の株価はこれから伸びる!といったときに,一直線を描いて伸びていくわけではなく,少しランダムさを含んでギザギザかたちで上昇していくことが想像出来ます}例えば,このブラック・ショールズ方程式を考案した,マイロン・ショールズにはノーベル賞が授与されました.
\footnote{フィッシャー・ブラックはその時には他界していました}例えば,確率微分方程式は,経済の分野においては,株式や金融派生商品や債権などの価格を予想することや,更には物理学や生物学などにランダムさを扱う多くの分野にも応用がされています.
\Chapter{eと微分方程式と半群の話}
\Section{はじめに}
前の$e$と微分方程式の話では,微分方程式を解く際に,$e$が重要であることを述べました.
ここで,大学生向けに,更に微分方程式を一般化した形について考え,そこにも$e$が現れることを紹介し,$e$が普遍的で便利な存在であることを述べたいと思います.
世の中にはたくさんの微分方程式がありますが,それを一般化した形で
\begin{eqnarray*}
&& \frac{du}{dt} = Au\\
&& u(0) = u_0
\end{eqnarray*}
という風に書いてしまいましょう.$u$は我々が求めたい関数,$A$は$u$に何らかの変換(例えば,数を足す,かける,微分するなど)を施すものです.\\
つまり,$2$つの式で定義される微分方程式を解くということ解釈すると,
\begin{center}
よくわからない関数$u$があってそれを解析したい.\\
$u$の最初の値と,$u$が瞬間瞬間にどのように変化していくかは,分かっているので,$u$の全体像を求めて欲しい.
\end{center}
これは,例えば物体や光や音や熱などがどのように動いていくかを調べたい物理ではよくあることで,それぞれの場合に対して微分方程式があります.
ここで数学がしたいことは,問題をすごく一般化したわけですが
\begin{center}
$u$という関数にはどのぐらいの性質を認めてよいのか,$A$という変換にもどのぐらいのの性質を認めてよいのか.
\end{center}
ということになります.より一般的で広い範囲の$u$,$A$を使えるような理論を構築すれば,それだけ多くの問題を同時に解決することが出来ます.
この記事では,$u$をバナッハ空間という空間に属するもの,$A$を線形作用素という変換に限定して構築された{\bf 関数解析}の理論について触れます.
\Section{関数解析と半群}
\Subsection{バナッハ空間}
\defb
$X$がバナッハ空間であるとは,完備なノルム空間であることである.
\defx
いきなり空間に対して$2$つの性質を仮定しましたが,どのようなことなのでしょうか,詳しく見てみましょう.
\defb
$X$が$K$線形空間であるとは,$u,v \in X$に対して足し算$u+v \in X$と,$u \in X, k \in K$に対して,スカラー倍$ku \in X$が定まっていて,\\
$u,v,w \in X$,$k,l \in K$に対して,(ベクトルと同様の)次のような性質を満たしているものである.
\footnote{線形代数を知っている人に対しては冗長であるので,詳しくは説明するべきではないし,線形代数を知らない人に対しても雰囲気だけを知ってもらいたいので厳密に書くことはしません.}\\
$ (u + v ) + w = u + (v + w) , u + v = v + u , u + 0 = u, u + (-u) = 0 ,$\\
$ k(u + v) = ku + kv , (k+l)u= ku + lu ,(kl)u = k(lu), 1u = u$
\defx
ここでたくさんの式が出てきましたが,$K$というのは,実数$\R$や複素数$\C$について考えてもらって構いません.そして,$X$という線形空間は,所謂高校数学のベクトル空間です.
高校数学のベクトルは矢印であり,矢印を足すことやスカラー倍することが許されていました.そして,ベクトルに対しては長さが定まっていたので,それを今から定めます.
\defb
線形空間$X$上で定義された$\R$に値を取る関数$\| , \| : X \to \R$がノルムであるとは次の$3$つの性質が成り立つことである.\\
(1)正値性: 全ての$u \in X$に対して$\| u \| \ge 0$が成り立つ.\\
(2)スカラー倍に対する同次性: 全ての$u \in X$と$k \in K$に対して$\| ku \| = |k| \| u\|$が成り立つ.\\
(3)三角不等式: 全ての$u,v \in X$に対して,$\|u+v \| \le \|u \| + \| v \|$が成り立つ.\\
また,ノルムが1つ指定された線形空間のことを{\bf ノルム空間}という.
\defx
これも長さにとって当然成り立って欲しい性質を述べただけとなりました.そして,バナッハ空間の$1$つ目の性質{\bf ノルム空間}とは,長さが定義された空間ということでした.\\
続いて完備性について,触れてみましょう.
\defb
ノルム空間$X$の元の列,$x_n \ (n=1,2,\cdots)$がコーシー列であるとは次を満たすことである.
\[
\| x_n - x_m \| \to 0 \ (n,m \to \infty)
\]
である.
\defx
つまり,ある点列\footnote{高校数学でいうところの数列であるが,ここで列をなしているものは空間上の点であるので,点列という}の間の距離がどんどん小さくなっているというが,コーシー列であるということの定義です.\footnote{コーシー列の詳しい話については,この$episode$の前田さんの記事「割り算再考」にも書いてありますが,もう一度触れてみます.}ある数列が収束しているとき,これはコーシー列になりますが,
逆に{\bf 一般にコーシー列は収束列ではありません}.例えば,$\Q$という有理数全体の空間を考えて,$3,3.1,3.14,3.141,3.1415,\cdots$という風に$\pi$にどんどん近づいて行くような数列を考えます.
この数列の間はどんどん$0$へと近づいていきますが,その収束先の$\pi$は有理数ではないため,収束先はありません.よって,この数列は収束列ではないのです
\footnote{これは空間を$\Q$で考えたからであり,もちろん$\R$という実数の空間で考えるとこの数列は収束します}.
\defb
ノルム空間$X$が{\bf 完備}であるとは任意のコーシー列が収束する先があるということである.
\defx
つまり,バナッハ空間の$2$つ目の性質はコーシー列のようなちゃんとした数列は,ちゃんと収束する先があるような空間を考えたいということです.\footnote{僕もコーシー列のようにちゃんとした数列なので収束先がほしい}
\begin{thebibliography}{9}
\item Hull, J. C. (2014), Options, Futures, and Other Derivatives, 9th edition (Upper Saddle River, NJ: Prentice Hall).
\end{thebibliography}
