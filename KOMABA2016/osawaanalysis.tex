\Section{解析}
\Subsection{はじめに}
高校生までの解析学では、初等関数と呼ばれる基本的な関数だけを 扱ってきました。$x^n,e^x,\log x,\sin x,\arcsin x,$ etc...や、及びその合成関数を興味の対象として、収束や極限、とくに微分積分を利用して性質を調べてきました。

しかしそもそも関数とは何だったかというと、「ある変数$x$に対して出力$y$を対応させるルール」というだけのルールだったので、例えば入力する変数は実数だけでなく複素数やいくつかの実数の組み合わせも考えられます。

さらに、出力の規則もたとえば
\begin{equation*}
  f(x) = \begin{cases}
    1 & (xは有理数) \\
    0 & (xは無理数)
  \end{cases}
\end{equation*}
という、すべての点で不連続な関数が考えられます。さらに
\[
 w(x) = \sum_{n=0}^\infty a^n \cos(b^n \pi x) \  (0<a<1,bは正の奇数整数)
\]
とおけば、連続関数にもかかわらずほとんどの点で微分不可能であるような関数が出来上がります(ワイエルシュトラス関数)。このような``病的な関数"も解析学の範疇として考えなければなりません。

大学に入学してしばらくは、収束、極限、微分積分に関するをもっと厳密に証明していきます。極限を扱う際に現れるのが、有名な\textbf{$\epsilon - \delta$ 論法}です。東大では昨年、「$\epsilon - \delta$ 論法は人類の常識ですよ??」というコピペが生まれました。なので詳しく知りたい人は、駒場祭を歩いている東大生を捕まえて聞いてみてください。


もうひとつ、数学科に進学する前に学ぶ解析の大きなトピックとしては、微分方程式があります。これは例えば
\[
y'' + 5y' + 6y = x^2
\]
のような式を満たす関数$y = f(x)$を探し出すものです。必修というわけではありませんが、理学部はもちろん工学部でも頻繁に使うので理系学生の多くが履修します。微分方程式を個別に対処していく方法ももちろんやりますが、数学科志望の生徒はむしろ、もっと一般的な微分方程式を考えたくなると思います。たとえば$y' = g(x,y)$なる形の微分方程式に対して、
\[
|g(x,y_1)-g(x,y_2)| \leq K|y_1 - y_2|
\]
が、任意の$y_1,y_2$に対してある定数$K$が存在して成立しているとき、\textbf{リプシッツ条件}を満たしているといい、解となる関数の一意性が保障されます。このような話を好むか好まないかで、進学する学部も変わってくるのでしょう。また、この微分方程式が多変数になった偏微分方程式になれば、これは3年の数学科の授業で学ぶ発展的な分野になります。

\Subsection{ルベーグ積分}
先ほど、有理数で1をとり無理数で0をとる関数を考えましたが、こういった関数の積分はどうやって行うのでしょうか。これは、縦に長い短冊をいっぱい作って面積を求めるリーマン積分では求めるのが難しいです。

このようなときに役に立つのが、より多くの関数を積分できる\textbf{ルベーグ積分}です。ですがこのルベーグ積分を扱う際に準備が必要なのが、\textbf{測度}と呼ばれる概念です。数学科のルベーグ積分の講義の半分くらいは、この測度に関する定理の証明になっています。

測度というのは、面積、体積といった大きさに関する概念を一般化したものです。集合$X$の部分集合族$A$があった時に、その大きさを測る関数を考えるのですが、次のような性質を満たしているものとします。

$\emptyset$を空集合、$E_1,E_2,E_3,\dotsc$をどの二つも互いに共通部分を持たない $A$ に属する集合の列としたとき、
\begin{gather*}
\mu (\emptyset) = 0 \\
\mu (\cup_i E_i) = \sum_{i=1} \mu (E_i)
\end{gather*}
を満たす関数$\mu$を測度と呼び、$A$の元を可測集合、測度が定義された空間を測度空間と呼びます。

測度というのはいろんな集合にたくさんの種類定義できるのですが、ここでは実数上に\textbf{ルベーグ測度}と呼ばれるものを考えます。2次元なら面積、3次元なら体積にあたるものとほぼ考えてもらっても構わないのですが、きちんと定義することによって、様々な性質が明らかになります。集合論の知識も合わせて考えると、実数という詰まった集合の中で、たとえば $\{1,2\}$ みたいな有限集合や、整数の集合のようなスカスカの集合、さらに有理数の集合でさえも、測度はすべて0になってしまうのです。

勘のいい人は気づいたのかもしれませんが、ルベーグ積分というのはこのように可測集合上で、可測関数と呼ばれるもの(よほどヤバい定義をしなければ大概の関数は可測です)の積分を考えます。つまりざっくり言うと、関数の値にその値をとる集合の測度を掛け算して積分値を考えます。つまり冒頭に出てきた関数は、値1をとる有理数の集合は測度0であるため、あの関数の計算結果はどう頑張っても0になってしまいます。

このように新しい積分を定義することによって、今まで扱えなかった関数を扱うことができます。そこでこれ以降の解析では、関数を集めた関数空間を考えて、性格のいい関数、悪い関数というのを考えていきます。具体的には、負の無限大から正の無限大まで積分しても発散しない・無限回微分できてなめらか・そもそも関数の値が0以外を取る部分が限られている(コンパクト台)…といったものが``性格のいい関数"と呼ばれています。ちなみに筆者の研究の話になるのですが、『関数空間』というからにはベクトル空間のように正規直交な基底を考えることができるのではないか、と思って、特に先ほどのような性格のいい正規直交基底は作れないか、と考える研究をしています。性格のいい関数をつかえば、その線形結合で関数を再現することができ、さらに係数の計算もかなり楽になります。

\Subsection{複素関数}
解析分野でもうひとつ扱うトピックが、複素関数です。定義域も値も複素数になるような関数を考えるのですが、数学科では必修で1年間かけてその世界に足を踏み入れます。

まず、複素関数の微分を考えます。微分可能な複素関数は\textbf{正則関数}といい、正則関数であれば\textbf{Cauchy-Riemann方程式}という関係式が成り立ちます。$f(x+iy)=u(x,y)+iv(x,y)$とおくとき、
\[
\frac{\partial u}{\partial x} = \frac{\partial v}{\partial y},\quad
\frac{\partial u}{\partial y} = -\frac{\partial v}{\partial x}
\]
がCauchy-Riemann方程式です。

正則関数は1回微分できる関数として定義されているのですが、複素関数の世界では1回微分できれば無限回微分できるという不思議な性質が成り立っています。このように正則関数の世界では、複素関数は非常に美しい定理が多く成り立っています。
複素関数の微分を考えれば、当然積分も考えます。複素平面上の積分は、座標平面上の関数のように線積分を考えるのですが、次の定理が成り立っています。
\[
Dを区分的C^1級境界をもつ有界領域、fをDとその境界を含む開集合上で定義された正則関数とすると、
\]
\[
\int_{\partial D} f(z) dz = 0
\]
\[
が成り立つ。これを\textbf{Cauchyの積分定理}という。
\]
つまりある領域で正則な関数は、周回積分すると計算結果が0になってしまうのです。さらに、正則でないような点を含んでいるような積分を実行する例として、
\[
f(z) = \frac{1}{2 \pi i} \int_{\partial D} {\frac{f(\zeta)}{\zeta - z}} d\zeta
\]
これを\textbf{Cauchyの積分表示}といいます。

さらにこれを発展させた留数定理というのもあります。これらの積分に関する定理は、複素積分のみならず実数上の積分でも活用できます。たとえば、
\[
\int_0^{\infty} \frac{1}{x^4 + 1}dx = \frac{\sqrt{2}\pi}{4}
\]
というものが求まります。

さらにもうひとつ美しい定理として、\textbf{一致の定理}が挙げられます。2つの正則関数$f,g$があって、これらが集積点(孤立していない点)をもつ集合上で一致しているとき、これらの関数はまるまるすべての区間で一致するというものです。つまり、ほんの少しの長さの線、ほんの少しの面積の領域で正則な関数が一致していれば、これらは広い広い複素平面の多くで一致してしまう、というものです。更にそれを発展させて、 $f$ が限られた集合でしか定義されていなくても、先のように集積点をもつ集合上で一致する関数 $g$ をもう一つ考えれば、 $f$ は $g$ と同じくらいまで拡張できるということです。これが\textbf{解析接続}というものです。解析接続の詳しいことは、複素関数の後半の授業で本格的に扱います。

また、逆関数も含めて両方向に正則な\textbf{双正則写像}を考えれば、複素関数の美しい世界はさらに広がります。任意の単連結な領域はその任意の点$z_0$に対し、原点まわり半径1にうつし合う双正則写像 $f$ で
\[
f(z_0) = 0, \quad f'(z_0) > 0
\]
が成り立つものが存在する、という\textbf{Riemannの写像定理}と呼ばれるものもあります。やがてその話は、関数を集めた関数族の話へつながっていきます。複素関数論は難解な理論だと言われていますが、結果は非常に美しいものが多いです。
