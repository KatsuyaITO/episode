大澤メモ.

例えば,整数全体の集合と多項式の集合という2つの集合を見たときに,共通点として,足し算と掛け算はある.さらに割り算を考えることはできないが,商と余りを考えることはできる.という共通点がある.これは,代数学の言葉では,ユークリッド環とよばれるものであり,ユークリッド環について研究するということは,整数全体の集合の理解を深めることでもあり,多項式の集合全体の理解を深めるということでもある.

幾何学者と呼ばれる人の殆どは「多様体」について研究している.ここで,重要なのは多様体のどのような性質にスポットライトを当てて研究するかである.{\bf 位相幾何学}と呼ばれる学問は,連続的に変形する\footnote{例えば,ある線を切ってしまうとそこで線分が連続ではなくなってしまうので連続的な変形ではないが,力を加えて少しだけ曲げることは連続的な変形である.}ことによって変わらない性質について研究する.例えば,目の前にボールと浮き輪があったときに,我々はこの2つが違うことが認識できるが,何が違うかについて考えたり,浮き輪とコップがあったときにこれらについて共通している性質は何かについて考えたりする.こうすることによって,世の中にある数学的対象を分類することができ,分類することはこの世界を理解することに近づく.分類することは理解を深めるということについては,ポアンカレ予想について考えると良い.ポアンカレ予想とは,「単連結な3次元閉多様体は3次元球面に同相である」という予想であるが,2002年ごろにペレルマンによって解決された.これにより,位相幾何学者にとっては,単連結な3次元閉多様体はすべて3次元球面と同じものであると言うことがわかったのである.
{\bf 微分幾何学}は更にここに連続的に加えて,滑らかに変形をすることによって変わらない性質について研究する.例えば,曲率や接線や接平面や測地線といった概念が微分幾何学の古典的でかつ中心的な対象である.更に微分幾何学者はこの微分可能な多様体に対して,幾つかの性質を加えて研究することが多い.例えば群構造を加えたLie群やRiemann計量と呼ばれる長さの概念を入れたリーマン幾何学は現代の幾何学の中心的な対象である.
{\bf 複素幾何学}はさらに多様体に,複素構造と呼ばれる構造を入れて研究する.例えば,数学界のノーベル賞とも呼ばれるフィールズ賞を受賞した小平邦彦先生は,複素幾何学で大きな功績を上げた.


群論はそれ単体でも奥が深い分野であり,その深遠さを表すものの1つに有限単純群の分類と言うものがある.有限単純群という群の中でも基本的な群は,「素数位数の巡回群」と「5次以上の交代群」と
「リー型の16種類の群」と「26種類の散在型単純群」の4種類しか無いことが,2004年になってようやく分かった.これらは,それだけでも面白い結果であるが,更に散在型単純群の中で最も大きな群であるモンスター群(808017424794512875886459904961710757005754368000000000個の要素を持っている)は物理学(共形場理論や弦理論)などとの関わりがあることが分かってきている.


環のモデルとなっているものに,多項式全体の集合(多項式環)というものがある.なぜ多項式環を考えたかったか,ひいては環論を学ぶのかというと,幾何学者たちが「多項式によって定義される図形」
($y=ax$で直線や,$x^2+y^2=1$で円など)を考えてきたということがある.つまり,環を考えることと多項式によって定まる図形を考えることは同じなのである.多項式環と幾何学のつながりを示した定理として,「ヒルベルトの零点定理」と呼ばれるものがある.これは,大雑把に言って「(アフィン)空間上の点全体」と「多項式環の極大イデアル」が一対一対応しているという定理であり,この定理によって,多項式環を考えることと幾何を考えることがおなじになった.これにより,多項式を考えることと多項式によって定義される幾何的対象を考えることの対応がつき,代数幾何学という分野が始まった.


体論を学ぶ1つのモチベーションとして,「ガロア理論」と呼ばれるものがある.
ガロア理論の1つ大きな結果として,「5次以上の方程式は一般には係数の四則演算と根号の組合せで解く事ができない」というものがある.
ガロアの考え方は以下のようである.$x^2 = 2$という方程式は,有理数の中では解けないが,有理数に$\sqrt[]{2}$という数を足してあげれば解ける.
つまり,考えている体を少しだけ大きくしてあげれば解ける.この体を少しだけ大きくする操作を拡大と言った.
ガロアの注目すべきところは,「体の拡大」と「群」には密接な関係があることを見破ったことである.
そして,「ある方程式$f$が係数の四則演算と根号の組合せで解く事ができる」という体の方程式の言葉を「$f$のガロア群が可解群である」
という群論の言葉に置き換えた.これによって,「5次以上の方程式のガロア群が一般には可解群ではない」という群論の命題を証明すれば良いこととなり,
ガロア理論は成功したのである.