%まず最初に使ったプリアンブルをここに書いてください.
%ただしコンパイルの都合上コメントアウトしてください.
%実際に確認する際は,各自の環境でmain.texにこのプリアンブルを追加してください.

%\usepackage{mathrsfs}
%\usepackage[all]{xy}
%\newcommand{\proofend}{\begin{flushright} $\blacksquare$ \end{flushright}}
%\renewcommand{\labelenumi}{(\roman{enumi})}
%\newcommand{\nkgr}{・}
%\theoremstyle{definition}
%\newtheorem{theorem}{定理}
%\renewcommand{\thetheorem}{}
%\newtheorem{defi}{定義}
%\newtheorem{thm}[defi]{定理}
%\newtheorem{lem}[defi]{補題}
%\newtheorem{cor}[defi]{系}
%\newtheorem{prop}[defi]{命題}
%\newtheorem{ex}[defi]{例}

\Chapter{Haar測度をつくろう(山本)}
% タイトル(名前)でお願いします.
% セクションは \Section \Subsection \Subsubsection で分けてください.
% 詳しくはMAY2015を参考にしてください.
\Section{\S 0. はじめに}
この項の目的を一言でいうと,「局所コンパクト位相群」の上で「良い振る舞い」をする「Haar測度」という測度を構成することです.
\Section{\S 1.もろもろの定義,お話}
有限群の表現論では,群$G$の元$s$を動かした際の総和を取って関数や内積を「平均化する」という手法があります.これをより一般の無限群でやろうとするわけですが,その際の「総和を取る」という操作は「積分」という形に置き換えればよさそうだということになります.積分を定めるためにはまず$G$上に測度を入れる必要がありますが,そのときに入れると便利な測度が,今回構成する「Haar測度」です.

Haar測度はそもそもとしてどのような測度なのかというと,数学的には次で定義されます.
\begin{defi}[Haar測度]
局所コンパクト位相群$G$のBorel集合族$\mathscr{B}(G)$上の測度$\mu$が左不変Haar測度であるとは,次を満たすこと:
・$G$のコンパクト集合$K$に対し,$\mu(K) \le \infty$
・$G$の開集合$O$に対し,$\mu(O)=sup \{ \mu(K) | K \subset O \colon compact \}$
・$G$のBorel集合$B$に対し,$\mu(B)=inf \{ \mu(O) | O \subset G \colon open \}$
・Borel集合$B$および$s \in G$に対し,$\mu (B) = \mu (sB)$
$mu$が右不変Haar測度であるとは,最後の条件を$\mu (B) = \mu (Bs)$に変えたものを満たすこと.
両側不変haar測度とは,左かつ右不変Haar測度であること.
\end{defi}
Haar測度の良い点としては次の二つです:
・$\mu$の定義域がBorel集合族$\mathscr{B}(G)$であるので,$G$の位相構造に対する振る舞いが良い.(正則性)
・可測集合を左(右)移動しても,測度が変わらない.(左(右,両側)不変性)
これらの点において,非常に扱いやすい測度となっています.

Haar測度の例としては,
・$\mathbb{R}^n$上のLebesgue測度
・群に離散位相を入れたときの数え上げ測度
などがあります.2つ目において,特に有限群$G$に離散位相を入れたとき数え上げ測度を考えると,これによる$G$上の関数の積分は先に述べた「平均化」と一致することが分かります.

というわけで,今回示したいのは次の命題です.
\begin{prop}[Haar測度の存在と一意性]
局所コンパクト群$G$上には,左不変Haar測度が正の定数倍を除いて一意的に存在する.
\end{prop}
この命題はFactとして用いられることが多いのですが,折角なので証明を追ってみようというのが本稿の目的です.

\Section{\S 2.色々な準備}
この章では,各種理論から今回の証明に必要なものを(あまり証明はせずに)紹介します.

\Subsection{局所コンパクトHausdorff空間論からの準備}
まず,一般のコンパクト空間に関する主張です.

\begin{thm}(Tychonoffの定理)
$\{ K_i \} _{i \in I}$をコンパクト集合の族とするとき,$\prod_{i \in I} K_i$はコンパクト.
\end{thm}
この証明は長いため省略します.例えば,次の参考文献を参照してください.

\begin{lem}
$J$をコンパクト集合,$\{ F_i \}_{i \in I}$を$J$の閉集合の族とし,$\bigcap_{i \in I}F_i = \emptyset$とする.このとき,有限個の$F_{i_1}, \ldots , F_{i_n}$を選んで,$\bigcap_{j=1}^{n}F_{i_j} = \emptyset$とできる.
\end{lem}
\begin{proof}
$\bigcap_{i \in I}F_i = \emptyset$の両辺の補集合を取ると,$\bigcup_{i \in I}(F_i)^{c} = J$ すなわち,${(F_i)^{c}}_{i \in I}$は$J$の開被覆になります.$J$のコンパクト性から,有限個の$i_1, \ldots , i_n \in I$を選んで$\bigcup_{j=1}^{n}(F_{i_j})^{c} = J$とできます.再び両辺の補集合を取れば,$\bigcap_{j=1}^{n}F_{i_j} = \emptyset$となります.
\end{proof}
続いて,局所コンパクトHausdorff空間について次のような主張があります.これらの証明も長いため,参考文献に譲ります.
\begin{thm}
$X$を局所コンパクトHausdorff空間,$U \subset X \colon open$,$K \in U$をコンパクト集合とする.このとき,$V \in X :compact$であって,$K \subset \int(V) \subset V \subset U$となるものが存在する.ただし,$int(V)$は$V$の$X$における内部とする.
\end{thm}
\begin{thm}[urysohnの補題]
$X$を局所コンパクトHausdorff空間,$U \subset X \colon open$,$K \subset U \colon compact$とする.このとき,次を満たす関数$f \colon X \to [0, 1]$が存在する:
(1)$K$上$f \equiv 1$
(2)$X \setminus V$上$f \equiv 0$
\end{thm}
\begin{thm}[1の分割]
$X$を局所コンパクトHausdorff空間,$K \subset X \colon compact$,$\{ U_i \}_{i=1, \ldots , n}$を$K$の開被覆とする.このとき,次を満たす関数$f_i \colon X \to [0,1] (i=1, \ldots , n)$が存在する:
(1)$K$上$\sum_{i=1}^{n}h_i \equiv 1$
(2)$supp f_i \in U_i (i=1, \ldots n)$
\end{thm}
続いて,今回の構成で決定的な役割をするRieszの表現定理について触れます.
\begin{defi}
(1)$X$:局所コンパクトHausdorff空間に対し,
$L_{+}(X)= \{ f \colon X \to \mathbb{R} \| f \ge 0, supp f \in X \colon compact\}$ と定める.
(2)$I \colon L_{+}(X) \to mathbb{R}$が$L_{+}(X)$上の正の線形汎関数であるとは,次を満たすこと;
・すべての$f \in L_{+}(X)$に対して,$I(f) \ge 0$
・すべての$f,g \in L_{+}(X)$に対して,$I(f+g)=I(f)+I(g)$
・すべての$f \in L_{+}(X), c \ge 0$に対して,$I(cf)=cI(f)$
\end{defi}
\begin{thm}[Rieszの表現定理]
$X$を局所コンパクトHausdorff空間,$I \colon X \to \mathbb{R}$を$L_{+}(X)$上の正の線形汎関数とする.このとき,$X$のBorel集合族$\mathscr{B}(X)$上の測度$\mu$であって,次を満たすものが存在する:
(a)すべての$f \in L_{+}(X)$に対して,$I(f)=\int_{X}fd\mu$
(b)すべての$K \in X \colon compact$に対して,$\mu(K)<\infty$
(c)すべての$B \in \mathscr{B}(X)$に対して,$\mu(B)=inf \{ \mu(O) \| O \subset X \colon open \}$
(d)すべての$O \in X \colon open$に対して,$\mu(O)=sup \{ \mu(K) \| K \subset O \colon compact \}$
\end{thm}
すなわち,$L_{+}(X)$上の正の線形汎関数に対して,$\mathscr{B}(X)$を定義域とする,いくつかの良い性質を持った測度が作れるということです.特に(b)(c)(d)を見ると,先の定義1に出ていた性質そのものです.このことから,位相群$G$上のHaar測度を構成するためには,$L_{+}(X)$上の正の線形汎関数$I$であって,「左不変なもの」を構成してしまえばよさそうだということになります.実際,次のセクションからはその方向でHaar測度を構成していきます.

その際,「左不変性」を実際に証明するために必要な命題があります.

\begin{lem}
定理9の状況で,$K$を$X$のコンパクト部分集合,$\chi_{K}$を$K$の特性関数とし,また$\varepsilon >0$とする.このとき,ある$g \in L_{+}(G)$が存在して,次を満たす:
・$\mu ( \{ x \in X \| \chi_{K}(x) \neq g(x) \} ) < \varepsilon $
・$g \le 1$
・$K$上$g \equiv 1$
\end{lem}
\begin{proof}
定理5の$K$を$K$,$U$を$X$として適用すれば,$K \subset int(V) \subset V subset X$となる$X$のコンパクト部分集合$V$が存在します.このとき,定理9(c)を用いると,$n=1,2,\ldots$に対し,$K \subset V_n \subset V$なる$V_n \subset X \colon open$で,$\mu(V_n) < \mu(K) + 2^{-n}$となるものが存在します.定理9(b)から$\mu(K)<\infty$だったので,$\mu(V_n \setminus K) < 2^{-n}\varepsilon$となります.定理6から,$f_n \in L_{+}(X)$で,$f_n(X) \subset [0,1]$,$K$上$f_n \equiv 1$,$X \setminus V_n$上$f_n \equiv 0$なる$f_n$が存在します.$g(x)=\sum_{n=1}^{\infty}2^{-n}f_n(x)$とおくと,これは一様収束なので$g$は連続です.$f_n$は$K$上1なので,$g$も$K$上常に1です.$f_n \le 1$より$g \le 1$も分かります.最後に,$f_n$は$K$および$(V_n)^{c}$上では$\chi_{K}$と一致しているので,$\{ x \in X \| \chi_{K}(x) \neq g(x) \} \subset \bigcup_{n=1}^{\infty}V_n \setminus K$です.したがって,$\mu( \{ x \in X \| \chi_{K}(x) \neq g(x) \} ) \le \sum_{n=1}^{\infty} \mu(V_n \setminus K) < \sum_{n=1}^{\infty}2^{-n}=\varepsilon$となります.
\end{proof}
\begin{prop}
定理9の状況で,$\mu(K)= \inf \{ I(f) | f \in \L_{+}(X), f|_{K} \equiv 1 \}$
\end{prop}
\begin{proof}
$K$上$f \equiv 1$なる$f \in L_{+}(G)$に対し,$ \mu(K)=\int_{X}\chi_K d\mu \le \int_{X}f d\mu=I(f)$より,$\mu(K) \le \inf \{ I(f) | f \in \L_{+}(X), f|_{K} \equiv 1 \}$となります.逆の不等号を示すため,$\varepsilon>0$とします.補題10で,$\varepsilon$を$\varepsilon /2$として得られる$g$について,$U=\{ x \in X | \chi_{K}(x) \neq g(x) \}$とおくと$\| \chi_{K}-g\| _{L^1(X)} = \int_{X}|\chi_K-g|d\mu = \int_{U}(\chi_K-g_n)d\mu \le int_{U}2 \cdot d\mu=2\mu(U)<2 \cdot \varepsilon / 2=\varepsilon$です.ゆえに$\| g\| _{L^1(X)} < \| \chi_{K} \| _{L^1(X)} $ですが,これは$\chi_K, g \ge 0$から結局 $\int_{X}gd\mu < \int_{X}\chi_{K}d\mu +\varepsilon$,すなわち$I(g)<\mu (K)+\varepsilon$となります.$g$は$K$上で1でしたから,$\inf \{ I(f) | f \in \L_{+}(X), f|_{K} \equiv 1 \} \le I(g) <\mu (K)+\varepsilon$となり,ここで$\varepsilon>0$は任意だったので,$\varepsilon \to 0$として $\inf \{ I(f) | f \in \L_{+}(X), f|_{K} \equiv 1 \} \le \mu (K)$となります.
\end{proof}
\Subsection{位相群論からの準備}

続いて,今回Haar測度を構成する舞台となる,位相群に関する準備です.

\begin{defi}[位相群]
群演算の入ったHausdorff空間$G$が位相群であるとは,群の積演算:$G \times G \to G$および逆元を取る操作:$G \to G$がGの位相およびその積位相に関して連続であること.
\end{defi}
位相群の性質をいくつか見ておきます.
\begin{prop}[位相群の性質]
(1)$G$の元$s$に関して,$s$を左(右)から掛ける操作:$G \to G$は同相写像である.
(2)$G$の単位元$e$の近傍$U$に対し,ある$e$の近傍$V$が存在して,$VV \subset U$となる.ただし,$V,W \subset G$に対し,$VW= \{ vw | v \in V, w \in W \}$である.
\end{prop}
\begin{proof}
(1)同様なので,左から掛ける操作のときのみ示します.$s \in G$とすると,命題の写像は$G \to G \times G \to G ; g \mapsto (s,g) \mapsto sg$と,連続写像の合成として書けるので連続です.また,これには左から$s^{-1}$を掛けるという逆写像も存在し,これも連続なので結局与えられた写像は連続です.
(2)$U$を$e$の近傍とします.積位相の定義と積演算の連続性から,$e$の近傍$V_1, V_2$で,$v_{1} \in V_1, v_{2} \in V_2$なら$v_{1}v_{2} \in U$となるものが存在します.よって特に$V=V_1 \cap V_2$とすれば,これが欲しかった$e$の近傍になります.
\end{proof}
後々の証明に必要なものを補題としていくつか作っておきます.

\begin{lem}
$G$のコンパクト部分集合$K$に対し,$K$上$f \equiv 1$となる$f \in L_{+}(G)$が取れる.
\end{lem}
\begin{proof}
Thm.5の$K$を$K$,$U$を$G$として適用すれば,$K \subset int(V) \subset V \subset G$ となる$G$のコンパクト部分集合$V$が存在します.このとき,Thm.6の$K$を$K$,$V$を$int(V)$として適用すれば,$K$上$f \equiv 1$および,$X \setminus int(V)$上$f \equiv 0$となる$f \colon X \to [0, 1]$が存在します.この$f$に対し,$x \in G$が$f(x)>0$を満たすなら$x \in int(V) \subset V$となりますから,$supp f \in V \colon compact$となり,$f \in L_{+}(G)$となります.
\end{proof}
\begin{lem}
$V$を$G$の単位元$e$の近傍とするとき,次を満たす$g \in L_{+}(G)$が取れる:
・$g(x)>0$
・$G \setminus V$上$g(x)=0$
・すべての$x \in G$に対して,$g(x)=g(x^{-1})$
\end{lem}
\begin{proof}
逆元を取る操作の連続性から,$V^{-1}=\{ x^{-1} | x \in V \}$も$e$の近傍です.ゆえに$V \cap V^{-1}$も$e$の近傍です.Thm.5の$K$を$\{e\}$に,$U$を$int(V \cap V^{-1}$として適用すると,$\{e\} \subset \int(F) \subset F \subset V \cap V^{-1}$となる$G$のコンパクト部分集合$F$が存在します.このとき,$F^{-1}$も$e$のコンパクト近傍なので,$F \cap F^{-1}$も$e$のコンパクト近傍になります.Thm.6を適用すると,$f(e)=1$,$supp f \in F \cap F^{-1}$となる$f \in L_{+}(G)$が存在します.$g(x)=f(x)+f(x^{-1})$なる$f' \in L_{+}(G)$を考えると,$g(e)=1+1=2>0$.また,すべての$x \in G$に対して,$g(x)=f(x)+f(x^{-1})=g(x^{-1})$です.そして,$supp g \subset F \cap F^{-1} \subset V \cap V^{-1} \in V$となるので,$G \setminus V$上$g(x)=0$となります. 
\end{proof}

\begin{prop}
$G$:局所コンパクト位相群とするとき,$L_{+}(G)$の元は一様連続である.すなわち,任意の$\varepsilon>0$に対し,$e$の近傍$V_1$であって,$xy^{-1} \in V_1$なる$x,y \in G$に対し,$|f(x)-f(s)|<\varepsilon$となるようなものが取れる.
\end{prop}
\begin{proof}
$\varepsilon>0$とします.また,$supp f=K \colon compact$とおきます.

まず,$e$の近傍$V_1$であって,$xy^{-1} \in V_1$なる$x \in G, y \in K$に対し,$|f(x)-f(s)|<\varepsilon$となるようなものが取れることを示します.$K$の元$a$に対し,$f$が連続であることから,ある$a$の近傍$V_a$が存在して,$f(V_a) \subset (f(a)-(\varepsilon /2), f(a)+(\varepsilon /2) )$となります.このとき,$V_{a}a^{-1}$は$e$の近傍なので,命題13(2)により,$W_{a}W_{a} \subset V_{a}a^{-1}$となる$e$の近傍が取れます.このとき,$\bigcup_{a \in K}(W_a)a$は$K$の開被覆なので,$K$のコンパクト性から,有限個の$a_1, \ldots, a_n$を選んできて,$( W_{a_i}a_{i} )_{i=1,\ldots,n}$を$K$の部分被覆とできます.$V_1=\bigcap_{i=1}^{n}W_{a_i}$は$e$の近傍の有限個の共通部分なのでやはり$e$の近傍です.\\

この$V_1$が先に述べた条件を満たすことを示します.$x \in G, y \in K, xy^{-1} V_1$とします.$y \in K$より,ある$k \in \{1, \ldots , n \}$であって,$y \in W_{a_k}a_{k}$,すなわち$ya_{k}^{-1} \in W_{a_k} \subset V_{a_k}a_{k}^{-1}$となるものが存在します.このとき$y \in V_{a_k}$より$f(y) \in (f(a_k)-(\varepsilon/2), f(a_k)+(\varepsilon /2) )$ すなわち $|f(y)-f(a_k)|<(\varepsilon /2)$となります.さらに,$x(a_k)^{-1}=x(y^{-1})y(a_k)^{-1} \in VW_{a_k} \in W_{a_k}W_{a_k} \in V_{a_k}a_{k}^{-1}$から,$|f(x)-f(a_k)|<(\varepsilon /2)$となります.よって,$|f(x)-f(y)|<\varepsilon$となることが分かりました.\\

同様にして,$e$の近傍$V_2$であって,$xy^{-1} \in V_1$なる$x \in K, y \in G$に対し,$|f(x)-f(s)|<\varepsilon$となるようなものが取れます.ここで,$V=V_1 \cap V_2$とおきます.これが命題の条件を満たす近傍になることを示します.$xy^{-1} \in V$とします.$x,y$の一方が$K$の元である場合は$xy^{-1} \in V_1 \cap V_2$より,先の場合に帰着できます.$x,y$の双方が$K=supp f$の元でないときは,$f(x)=f(y)=0$となりますから,やはり$|f(x)-f(y)|=0<\varepsilon$が成り立ちます. 
\end{proof}

これで,一通りの準備は終わりました.いよいよHaar測度を構成していきます.

\Section{\S 3.Haar測度の存在}
以降,$G$を局所コンパクト位相群とします.まず,構成だけでなく一意性の証明にも必要な道具を用意します.

\begin{defi}
(1)$s \in G, f \in L_{+}(G)$に対し,$Sf(x)=f(s^{-1}x)$として$Sf \in L_{+}(G)$を定める.
(2)$(\cdot \colon \cdot ) \colon L_{+}(G) \times L_{+}(G) \to [0, \infty)+\{ \infty \}$を次で定める:$f, g \in L_{+}(G)$とする.$G$の有限個の元$s_i$と,0以上の数$c_i$の組であって,すべての$x \in G$に対し$f(x) \le \sum_{i}c_{i} g(s_{i}^{-1}x)$となるものを考える.そのような組が存在するとき,$\sum_{i}c_i$の下限を$(f \colon g)$と定める.そのような組が存在しないとき,$(f \colon g)=\infty$と定める.
\end{defi}
今定義した写像の基本的な性質を見ておきます.
\begin{prop}
$f, g, h \in L_{+}(G)$とする.
(1)すべての'$s \in G$に対し,$(Sf \colon g)=(f \colon g)$
(2)すべての$c \ge 0$に対し,$(cf \colon g)=c(f \colon g)$
(3)すべての$x \in G$に対し$f(x) \le g(x)$なら,$(f \colon h) \le (g \colon h)$
(4)$(f+g \colon h) \le (f \colon h)+(g \colon h)$
(5)$f \ne 0$なら,$( f \colon g ) > 0$
(6)$g \ne 0$なら,$( f \colon g ) < \infty $
(7)$h \ne 0$なら,$( f \colon g ) \le (f \colon g ) (g \colon h )$
\end{prop}
\begin{proof}
(1)〜(4)は定義に戻って考えれば分かります.
(5)$f \ne 0, f(x) \le \sum_{i}c_{i} g(s_{i}^{-1}x) (\forall x \in G)$とします.$g \in L_{+}(G)$より$g$は有界です.ゆえに,$f(x) \le \sum_{i}c_{i} g(s_{i}^{-1}x) \le \sum_{i}c_{i} \sup g$となり,$\sup f \le \sup g ( \sum_{i}c_{i} )$,$0<\sup f \le \sup g (f \colon g )$となることから分かります.
(6)$g(s_{0}) > 0$なる$s \in G$を選び,$(1/2)g(s_{0})=m$とおきます.また,$C=supp f \colon compact$とおきます.$\Omega = g^{-1}( (m, \infty) )$とおくと$s_{0} \in Omega \subset G \colon open$で,$s \in G$に対して$s \in s s_{0}^{-1} \Omega \subset G \colon open$です.族$\{ s s_{0}^{-1}\Omega \}_{s \in G}$は$G$の開被覆であり,特に$C$の開被覆です.ゆえに,有限個の$ s_i(i=1,2, \ldots , n$を選んで$\{ s_{i} s_{0}^{-1}\Omega \}_{i}$が$C$の開被覆になるようにできます.$\sup f=M$とおくと,適当な場合分けにより$f(x) \le (M/m)\sum_{i} g( (s_{i} s^{-1}) ^{-1}x)$が分かります.ゆえに$(f \colon g ) \le nM/m \infty $となります.
(7)は,$f \le \sum_{i} c_{i}S_{i}g, g \le \sum_{j} d_{j}T_{j}h$なら$f \le \sum_{i,j} c_{i}d_{j}S_iT_{j}h$となることが分かるので,$(f \colon h ) \le \sum_{i,j} c_{i}d_{j} = ( \sum_{i} c_{i} ) (\sum_{j} d_{j} )$から$( f \colon g ) \le (f \colon g ) (g \colon h )$となります. 
\end{proof}

ここから,$L_{+}(G)$上の線形汎関数$I$を構成しにかかります.まず,値の「基準」となる$f_{0} \in L_{+}(G) \setminus \{ 0 \}$を一つ固定します.$G$の単位元$e$の近傍$V$全体を$\mathscr{V}$とおき,各$V \in \mathscr{V}$に対し,$G_{V}=\{ g \in L_{+}(G) | g(e)>0 かつ x \not\in V ならg(x)=0 \}$と定めます.各$G_V$は補題15により空ではありません.

\begin{prop}
$V \in \mathscr{V}, f \in L_{+}(G), g \in G_{V}$なら,$0 \le 1/( f_0 \colon f ) \le ( f \colon g ) / ( f_0 \colon g ) \le ( f \colon f_0 ) < \infty$
ただし,$f=0$のとき$1/( f_0 \colon f ) = 0$と考える.
\end{prop}
\begin{proof}
$f=0$のとき,$( f \colon g ) = ( f \colon f_0 ) = 0$となるので特に命題は成り立ちます.
$f \ne 0$のとき,$g \neq 0$および命題18の(7)から$(f_0 \colon g) \le (f_0 \colon f) (f \colon g )$,$ (f \colon g) \le (f \colon f_0 ) (f_0 \colon g )$が成り立ちます.$f, f_0 \neq 0$より,$(f_0 \colon g), (f_0 \colon f) \in (0, \infty)$が分かりますので,この不等式から主張が従います. 
\end{proof}

各$f \in L_{+}(G)$に対し,$J_{f}=[1/ ( f_0 \colon f ) , ( f \colon f_{0} ]$と定めます.これは$\mathbb{R}$の有界閉区間なのでコンパクト集合です.そして,$J=\prod_{f \in L_{+}(G)} J_f$と定めます.これは定理3によりコンパクト集合です.

また,$A \colon \bigcup_{V \in \mathscr{V}}G_V \to J$を,$A(g)=( ( f \colon g ) / ( f_0 \colon g ) )_{f \in L_{+}(G)}$により定めます.$A$による$G_V$の像$A(G_V)$の$J$における閉包を$F_V$とおきます.
\begin{prop}
$\bigcap_{V \in \mathscr{V}}F_V \neq \emptyset$.
\end{prop}
\begin{proof}
背理法によります.左辺が空であるとすると,$J$がコンパクトだったので,補題4から$V_1, \ldots , V_n \in \mathscr{V}$で$\bigcap_{i=1}^{n}F_{V_i}=\emptyset$なるものが存在します.$V_0=\bigcap_{i=1}^{n}V_i$とおくと,これは$e$の近傍であり,$F_{V_0} \neq \emptyset$です.この元を1つとって$g$とおくと,$i=1, \ldots , n$に対し$g \in V_i$となるので,$A(g) \in \bigcap_{i=1}^{n}F_{V_i}=\emptyset$となり矛盾します. 
\end{proof}

ゆえに,$I \in \bigcap_{V \in \mathscr{V}}F_V$なる$I$をとることができます.各$f \in L_{+}(G)$に対し,$I$の$J_f$成分を対応させる写像を$I \colon L_{+}(G) \to \mathbb{R}$と定めます.以降,これが線形汎関数となっていることを示しにいきます.
\begin{prop}
すべての$V \in \mathscr{V}$および有限個の$f_1, \ldots , f_n \in L_{+}(G)$,そして任意の$\varepsilon > 0$に対し,ある$g \in G_V$であって,$i=1, \ldots , n$に対し$|(f_i \colon g ) / ( f_0 \colon g ) -I(f_i) | < \varepsilon$となるものが存在する.
\end{prop}
\begin{proof}
$i=1, \ldots , n$に対し,$U_i=J_{f_i} \cap ( I(f_i)- \varepsilon, I(f_i)+\varepsilon )$とおくと,$U_i$は$J_{f_i}$で開です.$J$の積位相を考えれば,$\prod_{f \neq f_1, \ldots , f_n}J_f \times U_1 \times \cdots \times U_n$は$J$で開です.この集合を$U$とおきます.また,$I(f_i) \in U_i$より,$I \in U$です.$I$は$F_V$,すなわち$A(G_V)$の閉包の元ですから,$A(G_V) \cap U \neq \emptyset$となります.$I' \in A(G_V) \cap U$を一つ取ると,ある$g \in G_V$が存在して,$I'=A(g)$です.このとき,$I' \in U$の$J_{f_i}$成分を見ると,$(f_i \colon g ) / ( f_0 \colon g ) \in U_i = ( I(f_i)- \varepsilon, I(f_i)+\varepsilon )$となります.よって$|(f_i \colon g ) / ( f_0 \colon g ) -I(f_i) | < \varepsilon$が成り立ちます.
\end{proof}
命題21を,より使いやすい形にしておきます.

\begin{cor}
$V_m \in \mathscr{V}(m \in \mathbb{N})$および有限個の$f_1, \ldots , f_n \in L_{+}(G)$に対し,$L_{+}(G)$の元の列$(g_m)_{m \in \mathbb{N}}$であって,各$m$に対して$g_m \in G_{V_m}$かつ,各$i=1, \ldots , n$に対して $\lim_{m \to \infty} (f_i \colon g_m ) / ( f_0 \colon g_m ) =I(f_i)$となるものが存在する.
\end{cor}
\begin{proof}
$m=1,2, \ldots$に対し,$g_m$を命題21で$\varepsilon = 1/m$としたときの$g$として取ると,$|(f_i \colon g_m ) / ( f_0 \colon g_m ) -I(f_i) | < 1/m$となります.両辺の$m \to \infty$による極限を取れば, $\lim_{m \to \infty} (f_i \colon g_m ) / ( f_0 \colon g_m ) =I(f_i)$となります. 
\end{proof}

\begin{prop}
$I$は$L_{+}(G)$上の正の線形汎関数である.すなわち,次が成り立つ.
(1)すべての$f \in L_{+}(X)$に対して,$I(f) \ge 0$
(2)すべての$f,g \in L_{+}(X)$に対して,$I(f+g)=I(f)+I(g)$
(3)すべての$f \in L_{+}(X), c \ge 0$に対して,$I(cf)=cI(f)$
\end{prop}
\begin{proof}
(1)は$I$の定義の仕方から明らかです.
(3)$V$を$e$の適当な近傍とします.系22において,$m=1,2,\ldots$に対し$V_m=V$,$n=2$,$f_1=f$,$f_2=cf$としたときの$(g_m)_{m \in \mathbb{N}}$をとります.命題18(2)より$(cf \colon g_m)=c(f \colon g_m)$で,両辺を$(f_0 \colon g_m)$で割って$m \to \infty$の極限をとれば,$I(cf)=cI(f)$となります.
(2)$V$を$e$の適当な近傍とし,系22において,$m=1,2,\ldots$に対し$V_m=V$,$n=3$,$f_3=f_1+f_2$としたときの$(g_m)_{m \in \mathbb{N}}$をとります.命題18(4)より$(f_{1}+f_{2} \colon g_m ) \le (f_1 \colon g_m )+(f_1 \colon g_m)$となります.両辺を$(f_0 \colon g_m)$で割って$m \to \infty$の極限をとれば,$I(f_1+f_2) \le I(f_1)+I(f_2)$となります.
逆の不等号を示すため,次の補題を用意します.

\begin{lem}
$f, h, h'\in L_{+}(G)$とし,さらにすべての$x \in G$に対して$h(x)+h'(x) \le 1$とする.このとき,$I(fh)+I(fh') \le I(f)$.
\end{lem}
\begin{proof}
$m$を自然数とします.$h,h'$は命題16により一様連続なので,ある$V_m \in \mathscr{V}$で,$s^{-1}x \in V \Rightarrow |h(x)-h(s)|, |h'(x)-h'(s)| \le 1/m$となるものが取れます.$V_m$を上の通りとし,$n=3, f_1=fh, f_2=fh', f_3=f$として系22のような$(g_m)_{m \in \mathbb{N}}$をとります.一旦$m$を一つに固定することとします.$f \le \sum_{i}c_i S_{i}g_m$となるような$c_i>0, s_i \in G$を考えます.$x \in G$に対し,$s_{i}^{-1}x \not\in V_m$なら$S_{i}g_{m}(x)=g_{m}(s_{i}^{-1}x)=0$です.ゆえに,$f(x) \le \sum_{s_{i}^{-1}x \in V_m} c_{i}S_{i}g_{m}(x)$です.$s_{i}^{-1}x \in V_m$であるとき,$|h(x)-h(s_i)| \le 1/m$より,$h(x) \le h(s_i)+(1/m)$で,$f(x)h(x) \le \sum_{s_{i}^{-1}x \in V_m} c_{i}S_{i}g_{m}(x)h(x) \le \sum_{s_{i}^{-1}x \in V_m} c_{i}S_{i}g_{m}(x)(h(s_i)+(1/m)) \le \sum_{i}c_{i}( h(s_i)+(1/m) )S_{i}g_{m}(x)$ となります.よって,$( fh \colon g_m ) \le \sum_{i}c_{i} ( h(s_i)+(1/m) )$ です.$h'$も同様に,$( fh' \colon g_m ) \le \sum_{i}c_{i} ( h'(s_i)+(1/m) )$ です.これら2式を足して,$( fh \colon g_m )+( fh' \colon g_m ) \le \sum_{i}c_{i} ( h(s_i)+h'(s_i)+(2/m) ) \le \sum_{i}c_{i} ( 1+(2/m) )$となります.$\sum_{i}c_{i}$ に関する下限をとって,$( fh \colon g_m )+( fh' \colon g_m ) \le ( f \colon g_m ) ( 1+(2/m) )$です.両辺を$(f_0 \colon g_m)$で割って$m \to \infty$の極限をとれば,$I(fh)+I(fh') \le I(f)$となります. 
\end{proof}
命題23の証明に戻ります.$supp f_{1}+f_{2}=C$とおきます.$f_1, f_2 \ge 0$より,$supp f_1, suppf_2 \subset C$です.補題14により,$C$上常に1となる$f' \in \L_{+}(G)$がとれます.$\varepsilon>0$とし,$F_{\varepsilon}=f_{1}+f_{2}+\varepsilon f'$とおきます.$h,h' \in L_{+}(G)$を次で定めます:$x \in C$なら$h(x)=f_{1}(x)/F_{\varepsilon}(x), h'(x)=f_{2}(x)/F_{\varepsilon}(x)$.$x \not\in C$なら$h(x)=h'(x)=0$とします.$x \in C$かどうかで場合分けすれば,$f_{1}=hF_{\varepsilon}, f_{2}=h'F_{varepsilon}, h+h' \le 1$が分かります.よって,補題24から$I(f_1)+I(f_2) = I(hF_{\varepsilon})+I(h'F_{\varepsilon}) \le I(F_{\varepsilon})=I ( (f_{1}+f_{2})+\varepsilon f' ) \le I(f_{1}+f_{2})+ \varepsilon I(f')$ となります.$\varepsilon$は任意だったので,$\varepsilon \to 0$として,$I(f_1)+I(f_2) \le I(f_{1}+f_{2})$が従います.以上により,$I(f+g)=I(f)+I(g)$が言えました. 
\end{proof}

命題23により,$I$が正の線形汎関数であることが分かりました.したがって定理9により,$I$に対応する測度$\mu$ができます.残すは左不変性のみとなりました.

\begin{prop}
すべての$B \in \mathscr{B}(G), s \in G$に対し,$sB \in \mathscr{B}(G)$であって,$\mu(sB)=\mu(B)$である.すなわち,$\mu$は$G$上のHaar測度である.
\end{prop}
\begin{proof}
まず,命題18(1)を用いれば,命題23(3)と同じ方法により$I(f)=I(SF)$が従います.これを踏まえ,3つのステップに分けて証明を行います.

(Step1)$K \in G \colon compact$のとき.$sK \subset G \colon closed$より,$sK \in  \mathscr{B}(G)$です.$\mu(K)= \inf \{ I(f) | f \in \L_{+}(X), f|_{K} \equiv 1 \}=\inf \{ I(f) | f \in \L_{+}(X), f|_{sK} \equiv 1 \}=\mu(sK)$となります.ただし,1つ目と3つ目の等号は命題11,2つ目の等号は,$\{ f \in \L_{+}(X) | f|_{sK} \equiv 1 \}$と$\{ f \in \L_{+}(X) | f|_{sK} \equiv 1 \}$の間に$f \mapsto Sf$という1対1対応があり,この対応に関して$I(f)=I(Sf)$が成り立っていたことから分かります

(Step2)$O \in G \colon open$のとき.$sO \subset G \colon open$より,$sO \in \mathscr{B}(G)$です.$\mu(O)= \sup \{ \mu(K) | K \subset O \colon compact \}=\sup \{ \mu(K) | K \subset sO \colon compact \} = \mu (O)$となります.ただし,1つ目と3つ目の等号は定理9の性質(d),2つ目の等号は,$\{ K \subset O | compact \}$と$\{ K \subset sO | compact \}$との間に$K \mapsto sK$という1対1対応があり,この対応に関して(Step1)から$\mu(K)=\mu(sK)$が成り立っていたことから分かります.

(Step3)$B \in \mathscr{B}(G)$のとき.$\mathscr{M}=\{A \in \mathscr{B}(G) | \forall s \in G , sB \in \mathscr{B}(G) \}$とおくと,これは単調族であって,Step.2から$G$の開集合を全て元に持ちます.よって,$\mathscr{M}$は$G$の全ての開集合を元に持つ最小の完全加法族,すなわちBorel集合族を含みます.これは$B \in \mathscr{B}(G)$なら全ての$s \in G$に対し$ sB \in \mathscr{B}(G)$が成り立つことを意味しています.そして,$\mu(B)= \inf \{ \mu(O) | B \subset O \in G \colon open\}= \inf \{ \mu(O) | sB \subset O \subset G \colon open\} = \mu(sB)$となります.ただし,1つ目と3つ目の等号は定理9の性質(c),2つ目の等号は,$\{ O \subset G | open, B \subset O \}$と$\{ O \subset G | open, sB \subset O \}$との間に$O \mapsto sO$という1対1対応があり,この対応に関して(Step2)から$\mu(O)=\mu(sO)$が成り立っていたことから分かります.

以上により,Borel集合族の元である集合の測度は左移動で不変であることが分かり,$\mu$が左不変Haar測度であることが示されました. 
\end{proof}
\section{\S 4.Haar測度の一意性}
先のセクションで,(左不変)Haar測度の存在が示されました.このセクションではさらに,Haar測度の一意性を示していきます.

\begin{prop}
$G$を局所コンパクト位相群とし,$\mu_1, \mu_2$を$G$上の左不変Haar測度とする.このとき,ある$c>0$が存在して,すべての$B \in \mathscr{B}(G)$に対して,$\mu_1(B)=c\mu_2(B)$が成り立つ.すなわち,左不変Haar測度は正の定数倍を除いて一意である.
\end{prop}
この命題を示すためには,次の命題を示せば十分です.
\begin{prop}
$\mu$を局所コンパクト位相群$G$上の左不変Haar測度とし,$I \colon L_{+}(G) \to \mathbb{R}$を,$I(f)=\int_{G}f d\mu$により定める.$f_0 \in L_{+}(G)$を固定すると,$I(f)/I(f_0)$の値は左不変Haar測度の値によらない.
\end{prop}
まず,命題27が示されたとして,命題26が成立することを示します.$mu_1, mu_2$を$G$の左不変Haar測度とし,対応する積分を$I_1, I_2$とおくと,命題27から,すべての$f \in L_{+}(G)$に対して$I_{1}(f)/I_{1}(f_0)=I_{2}(f)/I_{2}(f_0)$となるので,$I_{1}(f)= \left( I_{1}(f_0)/I_{2}(f_{0}) \right) I_{2}(f)$が成り立ちます.そこで$c=\left( I_{1}(f_0)/I_{2}(f_{0}) \right)(>0)$とおくと,すべての$f \in L_{+}(G)$に対して$I_{1}(f)= c I_{2}(f)$が成り立ちます.あとは命題25と同じようにやれば命題26が従います. \proofend

それでは,命題27の証明に入ります.$f \in L_{+}(G)$とします.$f=0$のときは$I(f)=0$となるので,$f \neq 0$としてよいことが分かります.$C=supp f \cap supp f_{0}$とおきます.この$C$は定め方からコンパクトです.また,後で使うのですが,$C$上$f' \equiv 1$となるような$f' \in L_{
+}(G)$を1つ取ります.これは補題14から取れることが分かっています.$f, f' \neq 0$より,$\left( f' \colon f\right)$は0でも$\infty$でもありません.そこで,$m=1,2,\ldots$に対し,$\varepsilon_{m}=1/(m+1)\left( f' \colon f \right)>0$と置きます.$f$は命題16から一様連続なので,$m=1,2,\ldots$に対してある$V_m \in \mathscr{V}$が存在して,$s^{-1}x \in V_m$なら$|f(x)-f(s)|<\varepsilon _{m}$となるようにできます.各$V_m$に対し,補題15のような$g_m$を1つずつ取ります.以降しばらく,$m$を1つ固定して考えます.

積分 $\int_{G}f(s)g_{m}(s^{-1}x)d\mu(s)$を考えます.$s^{-1}x \not\in V_m$なら$g_{m}(s^{-1}x)=0$,$s^{-1}x \in V_m$なら$f(s) \ge f(x) -\varepsilon_{m}$であることから,$\int_{G}f(s)g_{m}(s^{-1}x)\mu(s)=\int_{Vx^{-1}}f(s)g_{m}(s^{-1}x)\mu(s) \ge (f(x)-\varepsilon_{m})\int_{Vx^{-1}}g_{m}(s^{-1}x)\mu(s)=(f(x)-\varepsilon_{m})\int_{Vx^{-1}}g_{m}(s^{-1}x)\mu(s)$となりますが,$g_{m}$の取り方から$g(s^{-1}x)=g\left( (s^{-1}x)^{-1} \right)=g(x^{-1}s)$となるので,先ほどの式は$\int_{G}f(s)g_{m}(s^{-1}x)d\mu(s) \ge (f(x)-\varepsilon_{m})\int_{Vx^{-1}}g_{m}(s^{-1}x)\mu(s) = (f(x)-\varepsilon_{m})\int_{Vx^{-1}}g_{m}(x^{-1}s)\mu(s)=(f(x)-\varepsilon_{m}) I(g_{m})$ となります.ゆえに,$(f(x)-\varepsilon_{m}) \le \left( 1/I(g) \right)\int_{G}f(s)g_{m}(s^{-1}x)d\mu(s)$ です.

ここで,$g_{m} \in L_{+}(G)$より,再び命題16から,任意の$\eta >0$に対し,ある$W \in \mathscr{V}$であって,$yz^{-1} \in W$なら$|g_{m}(y)-g_{m}(z)|<\eta$となるものがあります.

さらに,$G$の有限個の元$s_{i}$および関数$h_i \in L_{+}(G)$で,$C$上$\sum_{i}h_i \equiv 1$となり,さらに各$h_i$が$G \setminus s_{i}W$上で0になるようなものを取ることができます(ディユドネ分割):実際,$(sW)_{s \in G}$は$G$の開被覆なので特に$C$の開被覆で,$C$のコンパクト性より有限個の$s_i \in G$を取って,$(s_{i}W)_{i}$が有限部分被覆となるようにできます.これに対応する1の分割(定理7)により,この分割の存在が言えます.

このとき,$\int_{G}f(s)g_{m}(s^{-1}x)d\mu(s)=\int_{C}1 \cdot f(s)g_{m}(s^{-1}x)d\mu(s)=\int_{C} \left( \sum_{i}h_{i}(s) \right) f(s)g_{m}(s^{-1}x)d\mu(s)=\sum_{i} \int_{C}h_{i}(s)f(s)g_{m}(s^{-1}x)d\mu(s)=\sum_{i} \int_{G}h_{i}(s)f(s)g_{m}(s^{-1}x)d\mu(s)$です.$s \not\in s_{i}W$なら$h_{i}(s)=0$,$s \in s_{i}W$なら$s_{i}^{-1}x(s^{-1}x)^{-1}=s_{i}^{-1}s \in W$であることから$g_{m}(s^{-1}x) \le g_{m}(s_{i}^{-1}x) \le \eta$となります.よって, $\int_{G}h_{i}(s)f(s)g_{m}(s^{-1}x)d\mu(s)=\int_{s_{i}W}h_{i}(s)f(s)g_{m}(s^{-1}x)d\mu(s) \le \int_{s_{i}W}h_{i}(s)f(s)\left( g_{m}(s_{i}^{-1}x) + \eta \right) d\mu(s)=\left( g_{m}(s_{i}^{-1}x) + \eta \right) \int_{G}h_{i}(s)f(s)d\mu(s)=\left( g_{m}(s_{i}^{-1}x) + \eta \right)I(fh_{i})$となります.ゆえに,先の等式と合わせて$\int_{G}f(s)g_{m}(s^{-1}x)d\mu(s) \le \sum_{i}\left( g_{m}(s_{i}^{-1}x) + \eta \right)I(fh_{i})$ が分かります.

$c_i=I(fh_i)/I(g_{m})$とおくと,$\sum_{i}c_{i}=\sum_{i}s_{i}\left( 1/I(g_{m}) \right) \int_{G}f(x)h_{i}(x)d\mu=\left( 1/I(g_{m}) \right) \int_{C}f(x) \left( \sum_{i}s_{i}h_{i}(x) \right) d\mu=\left( 1/I(g_{m}) \right) \int_{C}f(x) d\mu=\left( 1/I(g_{m}) \right) \int_{G}f(x) d\mu=I(f)/I(g_{m})$であって,上の不等式たちから $(f(x)-\varepsilon_{m}) \le \left( 1/I(g_{m}) \right)\int_{G}f(s)g_{m}(s^{-1}x)d\mu(s) \le \left( 1/I(g_{m}) \right) \sum_{i}\left( g_{m}(s_{i}^{-1}x) + \eta \right)I(fh_{i})$が分かるので,これを整理して$f(x) \le \varepsilon_{m} + \eta \sum_{i}c_{i} + \sum_{i}c_{i}S_{i}g_{m}(x)$が分かります.ここでさらに,$f(x) \le \varepsilon_{m} + \left( \eta \sum_{i}c_{i} \right)f'(x) + \sum_{i}c_{i}S_{i}g_{m}(x)$…(i)とできます.($x \in C$かどうかで場合分けすれば分かります.) また,$\sum_{i}c_{i}S_{i}g_{m}(x) \le \sum_{i}c_{i}S_{i}g_{m}(x)$より, $\left( \sum_{i}c_{i}S_{i}g(x) \colon g_{m} \right) \le \sum_{i}c_{i}=I(f)/I(g)$です.以上を踏まえて,(i)の両辺を$\left( \cdot \colon g_{m} \right) $の$\cdot$に代入すると,命題18(2)(3)(4)から $\left( f \colon g_{m} \right) \le \left( \varepsilon_{m} + \eta \sum_{i}c_{i} \right)f' + \left( \sum_{i}c_{i}S_{i}g_{m} \colon g_{m} \right) \le \left( \varepsilon_{m} + \eta \sum_{i}c_{i} \right) \left( f' \colon g_{m} \right) + \left( \sum_{i}c_{i}S_{i}g_{m}(x) \colon g \right) \le \left( \varepsilon_{m} + \eta \sum_{i}c_{i} \right) \left( f' \colon g_{m} \right) + I(f)/I(g_{m})$ となります.$\eta>0$は任意だったので,$\eta \to +0$として,$\left( f \colon g_{m} \right) \le \varepsilon_{m} \left( f' \colon g_{m} \right) + I(f)/I(g_{m})$ …(ii)が分かります.

この(ii)に関して,さらに命題18(7)から$\left( f \colon g_{m} \right) \le \varepsilon_{m} \left( f' \colon f \right) \left( f \colon g_{m} \right) + I(f)/I(g_{m})$です.ゆえに$\varepsilon_{m}$の定義を思い出しつつ整理すると $\left( m/(m+1) \right) \left( f \colon g_{m} \right)I(g_{m}) \le I(f)$,$\left( f \colon g_{m} \right)I(g_{m}) \le mI(f)/(m+1) \le 2I(f)$となります.よって,$\left( f' \colon g_{m} \right)I(g_{m}) \le 
\left( f' \colon f \right) \left( f \colon g_{m} \right) I(g_{m}) \le 2I(f)\left( f' \colon f \right)$となります.ここまでをまとめると$0<I(f)/I(g_{m}) \le \left( f \colon g_{m} \right) \le \varepsilon_{m} \left( f' \colon f \right) \left( f \colon g_{m} \right) + I(f)/I(g_{m})$ …(iii)です.さて,ここまでの議論は$f_0$に対しても全く同様に行えます.すなわち,$0<I(f_0)/I(g_{m}) \le \left( f_0 \colon g_{m} \right) \le \varepsilon_{m} \left( f' \colon f_0 \right) \left( f_0 \colon g_{m} \right) + I(f_0)/I(g_{m})$ …(iv)が成り立ちます.(iv)式の辺々の逆数を取り,(iii)に辺々掛けて整理すると$I(f)/ \left( \varepsilon_{m} \left( f' \colon g_{m} \right)I(g_{m}) + I(f_0) \right) \le \left( f \colon g_{m} \right) / \left( f_0 \colon g_{m} \right) \le \left( \left( \varepsilon_{m} f' \colon g_{m} \right)I(g_{m}) + I(f) \right) / I(f_0)$…(v)となります.

$lim_{m \to \infty}\varepsilon_{m} =0, sup_{m \in \mathbb{N}}\left( f' \colon g_{m} \right)I(g_{m}) \le 2I(f)\left( f' \colon f \right) < \infty$より,$\lim_{m \to \infty}\varepsilon_{m} \left( f' \colon g_{m} \right)I(g_{m})=0$となります.これを踏まえて(v)式の辺々の$m \to \infty$における極限を取って,$I(f)/I(f_{0}) \le lim_{m \to \infty} \left( f \colon g_{m} \right) / \left( f_{0} \colon g_{m} \right) \le I(f)/I(f_{0})$,すなわち $I(f)/I(f_{0})=lim_{m \to \infty} \left( f \colon g_{m} \right) / \left( f_{0} \colon g_{m} \right)$となります.この式の右辺において,$g_{m}$の構成は$f, f_{0}$のみに依っていたので,Haar測度によらないことが分かり,したがって左辺もHaar測度によらないことが分かります. 
\proofend
