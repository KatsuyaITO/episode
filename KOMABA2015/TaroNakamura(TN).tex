\Chapter{非可換幾何の呼び声(TN)}
\Section{はじめに}
いきなりで申し訳ないのですが,本稿で非可換幾何学の理論を展開することはありません.タイトル詐欺もいいところですが,幾何と作用素環の間の対応を見て,そこから非可換幾何への着想を紹介いたします.

数学において,調べたい対象を別の何かと対応付け,対応付けたものを調べることで元の調べたい対象の性質がわかるようになる,ということはしばしばあります.例えば,それなりに良い性質を持つ幾何的な対象である局所コンパクトハウスドルフ空間というものを考えます.例えば,我々の住む空間である(と思われる)3次元実空間$\mathbb{R}^3$などがその例です.数直線や2次元平面も(通常の位相を考えれば)局所コンパクトハウスドルフ空間とみなせます.さて,この局所コンパクトハウスドルフ空間の上で連続関数を定義することができます.そのような連続関数すべてを集めてくると,その集合には関数の和と積,それから複素数倍を考えることができ,数学で「多元環」,「代数」などという構造が入ります.これを連続関数環といいますが,この代数的な「環」(正確には多元環)を調べることにより,元の空間の性質がよくわかる,ということが知られています.

この連続関数環は実は$C^*$-環というものの1つです.$C^*$-環はHilbert空間上の有界線型作用素のなす多元環ですが,その中でも可換(積の順序が交換可能)なもので,実は可換な$C^*$-環はある局所コンパクト空間上の連続関数環になります.従って一方を調べることはもう一方を調べることになります.

このような対応がありますが,$C^*$-環は可換なものだけでなく非可換なものがたくさん(それはもうたくさんです)あります.では,その非可換な$C^*$-環が可換なものと同じように幾何学的な何かに対応しているとしたら?対応している何かは「非可換空間」と呼ぶのがふさわしいでしょう.

今回の記事ではこのようなお話について紹介させていただきます.なお,話題の紹介を優先したため,また紙面と執筆者の気力の都合もあり証明は省略しました.(興味を持っていただけた方は参考文献を参照してください)
\Section{$C^*$-環}
以下では少々式を用いて作用素環論の基礎を述べる.よくわからないという方は流し読みで雰囲気だけでも感じてもらえると幸いである.
\Subsection{Banach空間,Hilbert空間}
線形空間$A$にノルム$\lVert x\rVert$(まぁ,絶対値みたいなものです)が定義されていて,そのノルムに関して完備である,すなわち
\begin{center}
Cauchy列 $\{ x_n\}_{n \in \mathbb{N}},\lim_{m,n \rightarrow \infty}\lVert x_m-x_n\rVert =0$に対し\\
その極限が存在する: $\lim_{n\rightarrow \infty}\lVert x_n-x\rVert=0$
\end{center}
をみたすとき,$A$を$\textgt{Banach空間}$という.

同様に線形空間に内積が定義され(pre-Hilbert空間,計量線形空間などという),内積について完備であるとき,Hilbert空間という.

$ex)$ 2次元実ベクトル全体の空間
\[\left\{ \left(
\begin{array}{c}
a \\
b \\
\end{array}
\right)\mathrel{}\middle|\mathrel{}a,b\in \mathbb{R}\right\}\]
に和と実数倍を通常のベクトルの和と実数倍で定め,ノルムを通常のベクトルの長さで定めると,これはBanach空間である.
\Subsection{$C^*$-環の定義}
$A$をBanach空間とする.$A$に積$A \times A \ni \left(x,y\right) \mapsto xy \in A$,写像$A\ni a \mapsto a^*\in A$が定義され,以下の条件を満たすとき,$A$を\textgt{$C^*$-環}という.
\begin{itemize}
\item $A$は和と積について$\mathbb C$上の多元環である.つまり,積について結合法則と分配法則が成り立つ.
\item $\lVert xy\rVert \leq \lVert x\rVert \lVert y\rVert$
\item $\left(x^*\right)^*=x$
\item $\left(x+y\right)=x^*+y^*$
\item $\left(xy\right)^*=y^*a^*$
\item $\left(\alpha x\right)^*=\overline{\alpha} x^*$
\item $\lVert x^*\rVert=\lVert x\rVert$
\item $\lVert xx^*\rVert=\lVert x\rVert \lVert x^*\rVert$
\end{itemize}
写像$*:a \mapsto a^*$を\textgt{対合}という.$C^*$-環$A$の積が順序によらないとき,すなわち任意の$a,b\in A$に対し,$ab=ba$であるとき,$A$は\textgt{可換}であるという.また,$A$が積について単位元をもつとき,$A$は\textgt{unital}であるという.
\Subsection{*-準同型写像}
$A,B$を$C^*$-環とする.写像$\pi:A \rightarrow B$が
\begin{gather*}
\pi(x+y)=\pi(x)+\pi(y), \pi(xy)=\pi(x)\pi(y) \\
\pi(\lambda x)=\lambda \pi(x) \lambda \in \mathbb{C} \\
\pi(x^*)=\pi(x)^*
\end{gather*}
をみたすとき,$\pi$は \textgt{*-準同型}であるという.$\pi $が全単射 *-準同型であるとき,$\pi$は同型であるという.$C^*$-環 $A$, $B$ の間に同型$\pi$が存在するとき,$A$, $B$ は\textgt{同型}であるといい,$A\simeq B$と表す.
\Subsection{$C^*$-環の直和}
$\{A_{\lambda}\}_{\lambda \in \Lambda}$を$C^*$-環の族とする.これに対し,集合$A$を
\[A=\{ \left( x_{\lambda}\right)_{\lambda \in \Lambda}|\forall x_{\lambda}\in \Lambda ,
\sup_{\lambda \in \Lambda}<\infty \}\]
で定める.$A$に和と積,スカラー倍,及びノルムを
\begin{itemize}
\item $\left( x_{\lambda}\right)_{\lambda \in \Lambda}+\left( y_{\lambda}\right)_{\lambda \in \Lambda}
=\left( x_{\lambda}+y_{\lambda}\right)_{\lambda \in \Lambda}$
\item $\left( x_{\lambda}\right)_{\lambda \in \Lambda}\left( y_{\lambda}\right)_{\lambda \in \Lambda}
=\left( x_{\lambda}y_{\lambda}\right)_{\lambda \in \Lambda}$
\item $\alpha \left( x_{\lambda}\right)_{\lambda \in \Lambda}=\left( \alpha x_{\lambda}\right)_{\lambda \in \Lambda}$
\item $\lVert \left( x_{\lambda}\right)_{\lambda \in \Lambda}\rVert=\sup_{\lambda \in \Lambda}\lVert x_{\lambda}\rVert$
\end{itemize}
で定めると,これは$C^*$-環である.$A$を$\{A_{\lambda}\}_{\lambda \in \Lambda}$の\textgt{直和}といい,
\[A=\sum_{\lambda \in \Lambda} \oplus A_{\lambda}\]
と表す.
\Subsection{重要な$C^*$-環の例}
$\Omega$を局所コンパクト空間とする.$C_{\infty}\left( \Omega\right)$を無限遠で消える$\Omega$上の連続関数全体の集合とする.これに和・スカラー倍と積,対合,ノルムを
\begin{itemize}
\item $\left(\lambda x+\mu y\right)\left(\omega\right)=\lambda x\left(\omega\right)+\mu y\left(\omega\right)$
\item $\left(xy\right)\left(\omega\right)=x\left(\omega\right)y\left(\omega\right)$
\item $x^*\left(\omega\right)=\overline{x\left(\omega\right)}$
\item $\lVert x\rVert=\sup\{|x\left(\omega\right)||\omega \in \Omega\}$
\end{itemize}
で定めると,$C_{\infty}(\Omega)$は可換$C^*$-環である.$\Omega$がコンパクトであるとき,かつその時に限り$C_{\infty}(\Omega)$はunitalである.

※局所コンパクト空間$\Omega$上の連続関数$x$が\textgt{無限遠で消える}(\textgt{vanishing at infinity}):任意の正数$\epsilon >0$に対し,あるコンパクト集合$K\subset \Omega$が存在して,
\[\forall \omega \in \Omega\setminus K,\lVert x(\omega)\rVert<\epsilon\]
が成り立つ.
\Section{Gelfand表現}
\Subsection{指標}
$A$を可換$C^*$-環とする.写像$\pi:A\rightarrow\mathbb{C}$が0写像でなく,$A$から$\mathbb{C}$への *-準同型であるとき,$\pi$を$A$の\textgt{指標}といい,$A$の指標全体を$\Omega\left(A\right)$で表し,\textgt{指標空間}という.

$\Omega(A)$に弱 * 位相,すなわち,$A$を$A^{**}$の部分空間として考えた時に,各$x\in A$を連続にする$\Omega(A)$上の位相であって,和,積,スカラー倍を連続にするようなもののうち最弱なものとする.すると,この位相に関し,$\Omega(A)$は局所コンパクトHausdorff空間になる.さらに,$A$がunitalならば$\Omega(A)$はコンパクトである.
\Subsection{Gelfand表現}
写像$\mathscr{F}:A \rightarrow C_{\infty}(\Omega(A))$を
\[\mathscr{F}(x)(\pi)=\pi(x)\]
で定める.$\mathscr{F}$を$A$の\textgt{Gelfand表現}という.
\Subsection{Gelfand-Naimarkの定理その1}
\begin{theo}
$A$を可換$C^*$-環とする.$A$の{\rm Gelfand}表現$\mathscr{F}$は$A$から$C_{\infty}(\Omega(A))$への等距離 *-同型である.
\end{theo}
このGelfand-Naimarkの定理により,任意の可換$C^*$-環に対し,ある局所コンパクトハウスドルフ空間上の連続関数環が対応することがわかった.実は局所コンパクトハウスドルフ空間上の連続関数環から元の位相空間を復元することができるので,この定理は可換$C^*$-環から局所コンパクトハウスドルフ空間を構成できることを示している(その逆も然り).すなわち,\textgt{可換$C^*$-環の理論は局所コンパクト(ハウスドルフ)空間の理論と等価}とみなしてよいということである.(余談ではあるが,圏論の言葉を用いれば,局所コンパクトハウスドルフ空間の圏と可換$C^*$-環の圏が圏同値である,ということである)

かくして可換$C^*$-環と局所コンパクト空間という幾何学的な概念が繋がった.この定理はGrothendieckのスキーム論にも影響を与えたと言われている.

では,この定理から可換ではない$C^*$-環も何か幾何学的なものに対応しているのではないか,と考えてみる.それはどんなものであろうか.可換$C^*$-環から局所コンパクトハウスドルフ空間を構成できるならば、同じ手続きによって非可換な$C^*$-環から「空間」を構成できるののではないだろうか.それはもはや「点」や「空間」といった概念が意味を成すのかわからないが,ともかく非可換$C^*$-環により「構成」した「空間」に相当する何かを「非可換空間」ということにしよう.「非可換空間」は$C^*$-環から構成されるので,関数環($C^*$-環)側から微分構造やRiemann計量を導入する方法がわかれば,非可換微分多様体や非可換Riemann多様体が考えられる.「非可換空間」について知るためにはその基となる非可換な$C^*$-環について知らねばなるまい.そこで,もう少し一般の可換とは限らない$C^*$-環について調べていく.
\Section{GNS表現}
\Subsection{$C^*$-環の表現}
$A$を(可換とは限らない)$C^*$-環,$\mathfrak{H}$をHilbert空間とする.*-準同型$\pi:A\rightarrow \mathcal{B}\left(\mathfrak{H}\right)$に対し,対$\left(\pi,\mathfrak{H}\right)$をAの\textgt{表現}という.$\pi$が単射であるとき,表現$\left(\pi,\mathfrak{H}\right)$は\textgt{忠実}であるという.
\Subsection{表現の直和}
$C^*$-環$A$の表現の族$\left(\pi_{\lambda},\mathfrak{H}_{\lambda}\right)_{\lambda \in \Lambda}$を考える.$\mathfrak{H}_{\lambda}$の直和Hilbert空間を,$\mathfrak{H}$;
\[H:=\bigoplus_{\lambda \in \Lambda}\mathfrak{H}_{\lambda}\]
とする.これに対し,表現の\textgt{直和}$\left(\pi,\mathfrak{H}\right)$を
\[\pi\left(x\right)\left(\left(\xi_{\lambda}\right)_{\lambda \in \Lambda}\right):=
\left(\pi_{\lambda}\left(x\right)\xi_{\lambda}\right)_{\lambda \in \Lambda}\]
で定める.
\Subsection{状態}
$A$を(可換とは限らない)$C^*$-環とする.$A$の線型汎函数$\omega:A\rightarrow \mathbb{C}$が
\[\forall x\in A,\omega(x^*x)\geq0\]
をみたすとき,$\omega$を\textgt{正線型汎函数}という.正線型汎函数は有界線型作用素である.$\lVert\omega\rVert$=1をみたす正線型汎函数を\textgt{状態}という.
\Subsection{GNS表現の構成}
$A$を(可換とは限らない)$C^*$-環とする.$A$の正線型汎函数$\omega:A\rightarrow \mathbb{C}$が与えられたとき,これを用いて$A$の表現を構成する.
\[N_{\omega}:=\{x\in A\mid\omega(x^*x)=0 \}\]
とすると,これは$A$の左イデアルであり,かつ閉集合である.$N_{\omega}$を$\omega$の左核という.

$x\in A$に対し,$\eta_{\omega}\left(x\right)$で商空間$A/N_{\omega}$の剰余類$x+N_{\omega}$を表すこととする.複素線形空間$A/N_{\omega}$に内積を
\[\left(\eta_{\omega}(x)|\eta_{\omega}(y)\right)=\omega(y^*x)\]
で定める.この内積に関して$A/N_{\omega}$を完備化して得られるHilbert空間を$\mathfrak{H}_{\omega}$とする.

各$a\in A$に対し線型作用素: $A/N_{\omega}\ni \eta_{\omega}\left(x\right)\mapsto \eta_{\omega}\left(ax\right)\in A/N_{\omega}$を考えると,これはHilbert空間$\mathfrak{H}_{\omega}$上の有界作用素$\pi_{\omega}\left(a\right)$に拡張できる.そこで写像$\pi_{\omega}:A\ni a\mapsto \pi_{\omega}\left(a\right)\in \mathcal{B}\left(\mathfrak{H}_{\omega}\right)$
を考えると,対$\left(\pi_{\omega},\mathfrak{H}_{\omega}\right)$は $A$ の表現である.この表現を$\omega$による\textgt{Gelfand-Naimark-Segal表現},略して\textgt{GNS表現}という.この表現の構成法を\textgt{GNS構成法}という.
\Subsection{普遍表現}
$A$を(可換とは限らない)$C^*$-環とする.$A$上の状態$\omega$全てについての表現の族$\left(\pi_{\omega},\mathfrak{H}_{\omega}\right)$の直和を,$A$の\textgt{普遍表現}という.
\Subsection{Gelfand-Naimarkの定理その2}
\begin{theo}
任意の$C^*$-環$A$の普遍表現は忠実である.特に,$A$の忠実な表現が存在する.したがって,任意の$C^*$-環$A$はある{\rm Hilbert}空間$\mathfrak{H}$上の有界線型作用素のなす$C^*$-環$\mathcal{B}\left(\mathfrak{H}\right)$と等距離*-同型である.
\end{theo}
これで可換とは限らない任意の$C^*$-環があるHilbert空間上の有界線型作用素のなす$C^*$-環と対応付けられることがわかった.実はHilbert空間は量子力学が展開される空間であり,有界線型作用素は量子力学における物理量を表す役割を果たしている.作用素の積は一般に非可換であり,その非可換性が実際に物理学の中で大きな役割を果たしている.したがって,そのような意味で非可換な$C^*$-環は非可換な「量子的」な空間と対応しているとみなせる.なお,状態という言葉は量子力学に由来する.

その他の例として,局所コンパクト空間上の力学系を考えると,図形の空間的な情報と力学系による時間発展の情報の両方を持つ非可換なC*-環が得られる.

このように非可換な$C^*$-環は非可換な空間の情報を持ったものであり,それを調べることにより,「非可換な空間」を得ることができる.
\Section{非可換空間の例}
ここまで非可換空間と非可換な$C^*$-環が対応するのだろう、という話を見てきたが、簡単な例を用いて実際に非可換な空間を考えて、$C^*$-環が現れることを見ることにする.
\Subsection{$n$ 個の点のなす空間}
一般の空間を考えてその上の運動を考えてもよいが、簡単のために $n$ 個の点からなる空間を考えよう.一般の方のイメージする空間からはおよそかけ離れたものであるから,「空間」という言葉に抵抗のある方もいらっしゃるかもしれないが,そういう方は $n$ 箇所の駅からなる路線と各駅の間を電車が直通で結んでいるようなイメージなどをしてもらうといいかもしれない.
\Subsection{点の運動}
各点に便宜上 $1,2,\dots,n$ という名前を付けよう.この空間の中での運動を考える.各点 $i$ から $i$ 自身への運動,つまり運動とは書いているが静止したまま動かないこと,点 $i$ から点 $j$ への運動が考えられる.また,点 $i$ から点 $j$ への運動の「逆の運動」は点 $j$ から点 $i$ への運動である.さらに運動の合成として,点 $i$ から点 $j$ への運動と点 $j$ から点 $k$ への運動の連続試行:点 $i$ から点 $k$ への運動を考えよう.また,このような運動の中で禁止されているものはないとしよう.点 $i$ から点 $j$ への運動を$e_{ji}$と表すことにして,これらのことを式で表すと
\[e_{ji}^*=e_{ij}, e_{kj}e_{ji}=e_{ki}\]
となる.これは $n$ 次の行列単位($\left(i,j\right)$-成分のみが1で他の成分はすべて0であるような $n$ 次正方行列)に他ならない.こうした $n$ 次行列単位全てを含むような$\mathbb{C}$上の最小の体系は複素数を成分とする $n$ 次正方行列の全体$M\left(n;\mathbb{C}\right)$である.従って,このような $n$ 個の点からなる空間での点から点への運動の体系を記述する情報は$M\left(n;\mathbb{C}\right)$内に記録されているはずであろう.そして,これより小さなものでは情報すべてを記録できない.この$M\left(n;\mathbb{C}\right)$は行列の和と積,複素数倍によりBanach環であり,対合として随伴行列,つまり$A\in M\left(n;\mathbb{C}\right)$の対合$A^*$を複素共役の転置行列$A^*={}^{t}\overline{A}$と定めると,$M\left(n;\mathbb{C}\right)$は非可換な$C^*$-環である.$M\left(n;\mathbb{C}\right)$はHilbert空間$\mathbb{C}^n$上の有界線型作用素のなす$C^*$-環である(Gelfand-Naimarkの定理その2.まぁ,使うまでもなくご存知の結果かもしれませんが…).
\Subsection{2点空間上の関数}
では,この $n$ 点空間の非可換性を見ていきたい.ここでは簡単のために$n=2$として2点空間$\{a,b\}$を考える.この空間の上で定義された関数を考えよう.とはいえ,2点しかない空間なので $a,b$ それぞれに対して関数の値を決めればいい.2点空間$\{a,b\}$上の関数$f$を
\[f(a)=\alpha, f(b)=\beta\]
で定める.ここで定数関数${\bf 1}:a,b\mapsto 1$と,$a$で$1$,$b$で$0$という値を取る関数$e$を考えれば,2点空間上の任意の関数は
\[f=\alpha e+\beta \left({\bf 1}-e\right)\]
と表せる.このまま積を考えても所詮は2点空間なので可換であるが,仮に「微分」を定義できたとすると,2点空間は非可換になってしまう.
\Subsection{非可換微分構造}
2点空間上の関数に「微分」$D$が定義できたとしよう.「微分」$D$は微分であるから,次の規則を満たさなければならないだろう.
\begin{itemize}
\item$D{\bf 1}={\bf 0}$
\item$D(fg)=\left(Df\right)g+f\left(Dg\right)$
\end{itemize}
このような規則をみたす「微分」が定義できると,上で定めた関数$f=\alpha e+\beta \left({\bf 1}-e\right)$の「微分」は
\[df=\left(\alpha-\beta\right)de\]
となる.

さて,ここで$f$として$e^2$を考えよう.定義から明らかに$e^2=e$である.この両辺を「微分」すると
\[\left(de\right)e+ede=de\]
この式の両辺から$2eDe$を引けば
\[\left(de\right)e-ede=\left({\bf 1}-2e\right)de\]
を得る.$de$が0でないとすると右辺が$0$でないから,関数 $e$ とその導関数 $de$ の積が交換可能でないことを意味している.普通の空間の上で微分を考えれば,これは交換可能なはずである! なんてこった! 2点空間は普通の空間じゃなかったんだよ!(な、なんだってー)これと類似の結果が一般の $n$ 点空間でも成り立つ(気力のある方はやってみると計算の練習になる,かも?).
\Subsection{微分について}
上で述べた「微分」について,詳しく立ち入るつもりはないが,そのような関数から関数への写像があること,すなわち上の規則をみたす「微分」の存在について述べておく.通常の多様体上の微分形式や微分とDirac作用素との交換関係の類似性をご存知の方はそれを思い出されると,納得しやすいかと思う.

$C^*$-環がGelfand-Naimarkの定理その2により対応するHilbert空間を$\mathfrak{H}$を考え,一般に$C^*$-環の元と非可換な$\mathfrak{H}$上の有界線型作用素$D$を一つ取り(存在の議論は省略する),$C^*$-環の元$f$の微分を
\[df:=Df-fD\]
により定める.これは先の「微分」がもつべき性質を満たしている.

先の2点空間上の関数$f=\alpha e+\beta \left({\bf 1}-e\right)$は$M\left(2;\mathbb{C}\right)$の元の中の対角成分が$\alpha,\beta$である2次対角行列に対応している.これに対して適当に対角行列でない行列をとり,それを$D$とすれば, $df$ は先の定義で確かに微分になる(興味のある方は計算してみてください).
\Subsection{非可換トーラス}
先の $n$ 点空間は微分を考えれば非可換になったが,関数の積自体は可換で,点も具体的に考えられた($C^*$-環から構成せずに直接空間を定義したので当然ではあるが…)そこでもう少し複雑な例を見ることにしよう.通常のトーラスをもとにした非可換トーラスを考えることができる.簡単のため2次元での「お話」のみを紹介する.トーラス上の関数は周期関数であるから,2次元トーラス上の関数は座標を$\left(x,y\right)$として
\[f(x,y)=\sum_{k,l} a_{kl}e^{ikx}e^{ily}\]
と書くことができる.この$e^{ix},e^{iy}$に対し,非可換積を
\[e^{ix}*e^{iy}=e^{i\theta}e^{iy}*e^{ix}\]
などで定めて,$e^{ix},e^{iy}$により生成される代数は非可換である.これに対応する非可換空間を非可換トーラスと定める.
\Section{あとがき}
ここまで$C^*$-環から始まって.非可換空間の紹介までしてきましたが,そんなもの何に使うの?という話だけを最期に少しさせていただきます.先に述べた通り,まず物理の量子力学とかかわりがあるのでそちらへの応用が期待されます.また,先の $N$ 点空間上の話は格子空間(これは$N^4$点空間です)について,その上で微分を考えると非可換性が現れてしまうことを意味しています.さらには超弦理論も非可換幾何学の性質を持つと言われています.こうした物理学やその背後によって導かれたその先に,非可換幾何学の世界が待っている…のかもしれません.

あまり「非可換幾何学」及びその入り口に具体的に触れる,ということはできませんでしたが,作用素環の基礎からはじめて非可換幾何学の着想の元となる定理までを紹介いたしました.少しでもその雰囲気を味わっていただけたなら幸いです.

\begin{thebibliography}{9}
\item Masamichi Takesaki「Theory of Operator Algebras」Springer
\item Shoichiro Sakai「$C^★$-Algebras and $W^★$-Algebras」Springer
\item 梅垣壽春,大矢雅則,日合文雄「復刊 作用素代数入門」共立出版株式会社
\item 生西明夫,中神祥臣「作用素環入門I 函数解析とフォン・ノイマン環」岩波書店
\item 竹崎正道,「作用素環の構造」岩波書店
\item 綿村哲「非可換幾何学と場の理論」日本物理學會誌vol55, No10, 2000, 一般社団法人日本物理学会
\end{thebibliography}
