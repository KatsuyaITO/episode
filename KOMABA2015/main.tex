\documentclass[notombow,a5,openany]{kyouritu}
\usepackage[dvipdfmx]{graphicx}
\usepackage{makeidx,multicol}
\usepackage{amsmath ,amssymb ,amsthm}
%プリアンブル
\topmargin -0.5in
\headheight 0.2in
\headsep 0.3in  
\evensidemargin -0.03in
\oddsidemargin -0.4in
%\textwidth 5.6in
%\textheight 8.4in
\makeatletter
\renewcommand{\thetable}{%
\arabic{table}}
\makeatother

%圏点
\def\kenten#1{%
\ifvmode\leavevmode\else\hskip\kanjiskip\fi
\setbox1=\hbox to \z@{・\hss}%
\ht1=.63zw
\@kenten#1\end}
\def\@kenten#1{%
\ifx#1\end \let\next=\relax \else
\raise.63zw\copy1\nobreak #1\hskip\kanjiskip\relax
\let\next=\@kenten
\fi\next}

%Section等先頭を大文字にすると番号付けしない.
\newcommand{\Chapter}[1]{\chapter*{{\Huge #1}}
\markboth{#1}{#1}
\addcontentsline{toc}{chapter}{#1}}
\newcommand{\Section}[1]{\section*{{\huge #1}}
\addcontentsline{toc}{section}{#1}}
\newcommand{\Subsection}[1]{\subsection*{\underline{#1}}}
\newcommand{\Subsubsection}[1]{\subsubsection*{#1}}
\setcounter{tocdepth}{0} %Chapterのみ表示する
%TN.tex
\usepackage{amssymb}
\usepackage{amsmath}
\usepackage{mathrsfs}
\usepackage{bm}
\newtheorem{theo}{定理}
\renewcommand{\thetheo}{}
\newtheorem{lemm}[theo]{補題}
\newtheorem{prop}[theo]{命題}

%ito.tex
%これは他とぶつかったら変える予定です.
\newtheorem{Thm}{定理}
\renewcommand{\theThm}{}
\newtheorem{Lemma}{補題}
\renewcommand{\theLemma}{}
\newtheorem{Def}{定義}
\renewcommand{\theDef}{}
\newtheorem{Prop}{命題}
\renewcommand{\theProp}{}
\newtheorem{Ex}{例}
\renewcommand{\theEx}{}
\newtheorem{Prob}{問題}
\renewcommand{\theProb}{}
\newtheorem{Rem}{注意}
\renewcommand{\theRem}{}
\def\qedsymbol{$\square$}
\def\proofname{\gt{証明}\;}
\newenvironment{Proof}{\par\noindent{\it\proofname}}{{\unskip\nobreak\hfill{\it\qedsymbol}}\par\vskip 9pt}
\newenvironment{Proof*}{\par\noindent}{{\unskip\nobreak\hfill{\it\qedsymbol}}\par\vskip 9pt}
\ifx\undefined\bysame \newcommand{\bysame}{\leavevmode\hbox to3em{\hrulefill}\,}\fi
\def\C{\mathbb C}
\def\N{\mathbb N}
\def\R{\mathbb R}
\def\Q{\mathbb Q}
\def\Z{\mathbb Z}
\newcommand{\Real}{\mathop{\mathrm{Re}}\nolimits}
\def\thm{\begin{Thm}}
\def\thmx{\end{Thm}}
\def\prop{\begin{Prop}}
\def\propx{\end{Prop}}
\def\defb{\begin{Def}}
\def\defe{\end{Def}}
\def\defx{\end{Def}}
\def\rem{\begin{Rem}}
\def\remx{\end{Rem}}
\def\prob{\begin{Prob}}
\def\probx{\end{Prob}}
\def\lem{\begin{Lemma}}
\def\lemx{\end{Lemma}}
\def\ex{\begin{Ex}}
\def\exx{\end{Ex}}
\def\cor{\begin{Cor}}
\def\corx{\end{Cor}}
\def\proof{\begin{Proof}}
\def\proofx{\end{Proof}}
\def\a{\alpha}
\newcommand{\Image}{\mathop{\mathrm{Im}}\nolimits}
\newcommand{\Ker}{\mathop{\mathrm{Ker}}\nolimits}
\newcommand{\Coker}{\mathop{\mathrm{Coker}}\nolimits}
\newcommand{\Aut}{\mathop{\mathrm{Aut}}\nolimits}
\newcommand{\Ho}{\mathop{\mathrm{H}}\nolimits}
\newcommand{\Res}{\mathop{\mathrm{Res}}\nolimits}

\def\TO{\Rightarrow}
\def\OT{\Leftarrow}

%本文
\begin{document}
\frontmatter
\Chapter{まえがき}
\Chapter{まえがき}
本日は「数学科展示 ますらぼ」にご来場いただき誠にありがとうございます.本企画は今年度を持ちまして5年目となります.私達の上の上のそのまた上の学年から始まり,今回先代から私達数学科2016年度進学(現在数学科4年生)が引き継ぎました.受け継いだ「ますらぼ」「$e^{\pi i}sode$(えぴそーど)」の名前の重さに押しつぶされそうになりながらも,先輩方の多大なるご助力のもと,何とか一つの形にすることができました.数学科や「ますらぼ」の名前に泥を塗るようなことになっていないことを祈るばかりです.

数学科の学生は普段はここ本郷キャンパスではなく,駒場キャンパスという少し離れた別の場所で活動しています.他学部と比べて実験や実習のようなものがほとんどないため,みんな1日の多くの時間を数学に没入しながら,日々数学がわかったり,数学がわからなかったりに一喜一憂しています.

ところが,一人一人がどのような数学をやっているかとなると,これは人によってバラバラです.

「数学」というものはよくひっくるめて一緒くたに扱われますし,「数学は本質的には一つなのだ」という考えはごく自然なもののように思えます.しかし実際にはそんなことは無く,数学の世界にも「畑違い」「よその庭」「人には人の乳酸菌」があります.どんな大数学者も,その時代の数学を全て理解したことはいまだかつてありません.思うに,数学は統一的に意識されながらも,決して統一されることは無さそうです.そしてこれはむしろ嬉しいことのように思えます.というのも,これは数学の多種多様な楽しみ方,それも自分だけの楽しみ方を,そっくりそのまま保証してくれるからです.ただ残念なことに,全ての数学に出会うことは人の短い一生ではどうやら不可能そうです.

今回の$e^{\pi i}sode$には,人生では出会うことがむしろ稀な数学がたくさん詰まっています.これはとても私一人のなせるわざではなく,執筆者になってくれた同期達の深くそれでいて個性的な知識の賜です.本企画・冊子が,数学との新しい出会いのきっかけになっていただけたのならば,これ以上に嬉しいことはありません.是非1冊お手に取ってみてください.
(高木)

\tableofcontents
\mainmatter
%%まず最初に使ったプリアンブルをここに書いてください.
%ただしコンパイルの都合上コメントアウトしてください.
%実際に確認する際は,各自の環境でmain.texにこのプリアンブルを追加してください.

%\usepackage{mathrsfs}
%\usepackage[all]{xy}
%\newcommand{\proofend}{\begin{flushright} $\blacksquare$ \end{flushright}}
%\renewcommand{\labelenumi}{(\roman{enumi})}
%\newcommand{\nkgr}{・}
%\theoremstyle{definition}
%\newtheorem{theorem}{定理}
%\renewcommand{\thetheorem}{}
%\newtheorem{defi}{定義}
%\newtheorem{thm}[defi]{定理}
%\newtheorem{lem}[defi]{補題}
%\newtheorem{cor}[defi]{系}
%\newtheorem{prop}[defi]{命題}
%\newtheorem{ex}[defi]{例}


\Chapter{象の卵は美味しいぞう(伊藤)}
% タイトル(名前)でお願いします.
% セクションは \Section \Subsection \Subsubsection で分けてください.
% 詳しくはMAY2015を参考にしてください.
\Section{\S 0. はじめに}
本研究の真の目的は、一言で言えば、象の卵を見つけるという子供の頃からの夢をかなえることで
ある。
\Section{\S 1.研究目的}
本研究の目的は、象の卵の殻について、生物、化学、物理、工学などの方面から多角的に調べることである。象の卵の殻は、80kg を超える体重の子象と、その栄養源である卵黄の大きな質量を支えるだけではなく、卵を暖める親の象の体重も支える必要がある。このため、象の卵の殻は、体重の軽い鳥類 (図 1) の卵の殻とは本質的に異なる構造を持つ
\Section{\S 2.研究方法}
初年度は、まず世界の動物園を巡り、研究業績 [1] に可能性が示されたように象舍に卵が隠されて
いないか、探す。
2年目はアフリカに行き、空と地上から象の卵を探す。アフリカ象は気性が荒いが、サバンナの方
がジャングルよりも見通しが効くので、インドよりもアフリカを先に探索する。
3年目は、インドとタイに行き、ジャングルに隠されている卵を探す。ジャングルの場合は空から
は探しにくいが、象使いも多く、象の背中に乗って象の視点から探索することができる。さらに、気
だての優しいインド象ならば卵の在処を教えてくれる可能性もある。
ぞうの卵はおいしいぞう。ぞうの卵はおいしいぞう。ぞうの卵はおいしいぞう。ぞうの卵はおいし
いぞう。ぞうの卵はおいしいぞう。ぞうの卵はおいしいぞう。ぞうの卵はおいしいぞう。ぞうの卵は
おいしいぞう。ぞうの卵はおいしいぞう。ぞうの卵はおいしいぞう。ぞうの卵はおいしいぞう。ぞう
の卵はおいしいぞう。ぞうの卵はおいしいぞう。ぞうの卵はおいしいぞう。ぞうの卵はおいしいぞう。
ぞうの卵はおいしいぞう。ぞうの卵はおいしいぞう。ぞうの卵はおいしいぞう。ぞうの卵はおいしい
ぞう。ぞうの卵はおいしいぞう。ぞうの卵はおいしいぞう。ぞうの卵はおいしいぞう。ぞうの卵はおい
しいぞう。ぞうの卵はおいしいぞう。ぞうの卵はおいしいぞう。ぞうの卵はおいしいぞう。ぞうの卵
はおいしいぞう。ぞうの卵はおいしいぞう。ぞうの卵はおいしいぞう。ぞうの卵はおいしいぞう。ぞ
うの卵はおいしいぞう。ぞうの卵はおいしいぞう。ぞうの卵はおいしいぞう。ぞうの卵はおいしいぞ
う。ぞうの卵はおいしいぞう。ぞうの卵はおいしいぞう。ぞうの卵はおいしいぞう。ぞうの卵はおい
しいぞう。ぞうの卵はおいしいぞう。ぞうの卵はおいしいぞう。ぞうの卵はおいしいぞう。ぞうの卵
はおいしいぞう。ぞうの卵はおいしいぞう。ぞうの卵はおいしいぞう。ぞうの卵はおいしいぞう。ぞ
うの卵はおいしいぞう。ぞうの卵はおいしいぞう。ぞうの卵はおいしいぞう。ぞうの卵はおいしいぞ
う。ぞうの卵はおいしいぞう。ぞうの卵はおいしいぞう。ぞうの卵はおいしいぞう。ぞうの卵はおい
しいぞう。ぞうの卵はおいしいぞう。ぞうの卵はおいしいぞう。ぞうの卵はおいしいぞう。ぞうの卵
はおいしいぞう。ぞうの卵はおいしいぞう。ぞうの卵はおいしいぞう。ぞうの卵はおいしいぞう。ぞ
うの卵はおいしいぞう。ぞうの卵はおいしいぞう。ぞうの卵はおいしいぞう。ぞうの卵はおいしいぞ
う。ぞうの卵はおいしいぞう。ぞうの卵はおいしいぞう。ぞうの卵はおいしいぞう。ぞうの卵はおい
しいぞう。ぞうの卵はおいしいぞう。ぞうの卵はおいしいぞう。ぞうの卵はおいしいぞう。ぞうの卵
はおいしいぞう。ぞうの卵はおいしいぞ
\Chapter{非可換幾何の呼び声(TN)}
\Section{はじめに}
 いきなりで申し訳ないのですが,本稿で非可換幾何学の理論を展開することはありません.タイトル詐欺もいいところですが,幾何と作用素環の間の対応を見て,そこから非可換幾何への着想を紹介いたします.\\
 数学において,調べたい対象を別の何かと対応付け,対応付けたものを調べることで元の調べたい対象の性質がわかるようになる,ということはしばしばあります.例えば,それなりに良い性質を持つ幾何的な対象である局所コンパクトハウスドルフ空間というものを考えます.例えば,我々の住む空間である(と思われる)3次元実空間$\mathbb{R}^3$などがその例です.数直線や2次元平面も(通常の位相を考えれば)局所コンパクトハウスドルフ空間とみなせます.さて,この局所コンパクトハウスドルフ空間の上で連続関数を定義することができます.そのような連続関数すべてを集めてくると,その集合には関数の和と積,それから複素数倍を考えることができ,数学で「多元環」,「代数」などという構造が入ります.これを連続関数環といいますが,この代数的な「環」(正確には多元環)を調べることにより,元の空間の性質がよくわかる,ということが知られています.\\
 この連続関数環は実は$C^*$-環というものの1つです.$C^*$-環はHilbert空間上の有界線型作用素のなす多元環ですが,その中でも可換(積の順序が交換可能)なもので,実は可換な$C^*$-環はある局所コンパクト空間上の連続関数環になります.従って一方を調べることはもう一方を調べることになります.\\
 このような対応がありますが,$C^*$-環は可換なものだけでなく非可換なものがたくさん(それはもうたくさんです)あります.では,その非可換な$C^*$-環が対応しているであろう「非可換」な「空間」はどういったものなのでしょうか?\\
 今回の記事ではこのようなお話について紹介させていただきます.なお,話題の紹介を優先したため,また紙面と執筆者の気力の都合もあり証明は省略しました.(興味を持っていただけた方は参考文献を参照してください)
\Section{$C^*$-環}
 以下では少々式を用いて作用素環論の基礎を述べる.よくわからないという方は流し読みで雰囲気だけでも感じてもらえると幸いである.
\Subsection{Banach空間,Hilbert空間}
 線形空間$A$にノルム$\parallel x\parallel$(まぁ,絶対値みたいなものです)が定義されていて,そのノルムに関して完備である,すなわち
\begin{center}
Cauchy列$\{ x_n\}_{n \in \mathbb{N}},\lim_{m,n \rightarrow \infty}\parallel x_m-x_n\parallel =0$に対し\\
その極限が存在する:$\lim_{n\rightarrow \infty}\parallel x_n-x\parallel=0$
\end{center}
をみたすとき,$A$を$\textgt{Banach空間}$という.\\
 同様に線形空間に内積が定義され(pre-Hilbert空間,計量線形空間などという),内積について完備であるとき,Hilbert空間という.
 $ex)$ 2次元実ベクトル全体の空間
\begin{center}
$\Biggl\{ \left(
\begin{array}{c}
a \\
b \\
\end{array}
\right)\Biggl|a.b\in \mathbb{R}\Biggr\}$
\end{center}
に和と実数倍を通常のベクトルの和と実数倍で定め,ノルムを通常のベクトルの長さで定めると,これはBanach空間である.\\
\Subsection{$C^*$-環の定義}
 $A$をBanach空間とする.Aに積$A \times A \ni \left(x,y\right) \mapsto xy \in A$,写像$A\ni a \mapsto a^*\in A$が定義され,以下の条件を満たすとき,$A$を\textgt{$C^*$-環}という.
\begin{itemize}
\item $A$は和と積について$\mathbb C$上の多元環である.つまり,積について結合法則と分配法則が成り立つ.
\item $\parallel xy\parallel \leq \parallel x\parallel \parallel y\parallel$
\item $\left(x^*\right)^*=x$
\item $\left(x+y\right)=x^*+y^*$
\item $\left(xy\right)^*=y^*a^*$
\item $\left(\alpha x\right)^*=\overline{\alpha} x^*$
\item $\parallel x^*\parallel=\parallel x\parallel$
\item $\parallel xx^*\parallel=\parallel x\parallel \parallel x^*\parallel$
\end{itemize}
写像$*:a \mapsto a^*$を\textgt{対合}という.$C^*$-環$A$の積が可換であるとき,$A$は\textgt{可換}であるという.また,$A$が積について単位元をもつとき,$A$は\textgt{unital}であるという.
\Subsection{*-準同型写像}
 $A,B$を$C^*$-環とする.写像$\pi:A \rightarrow B$が
\begin{center}
$\pi \left( x+y\right)=\pi \left( x\right)+\pi \left( y\right), \pi \left( xy\right)=\pi \left( x\right)\pi \left( y\right)$
$\pi \left( \lambda x\right)=\lambda \pi \left( x\right) \lambda \in \mathbb{C}$
$\pi \left(x^*\right)=\pi \left(x\right)^*$
\end{center}
をみたすとき,$\pi$は\textgt{*-準同型}であるという.$\pi $が全単射*-準同型であるとき,$\pi$は同型であるという.$C^*$-環$A,B$の間に同型$\pi$が存在するとき,$A,B$は\textgt{同型}であるといい,$A\simeq B$と表す.
\Subsection{$C^*$-環の直和}
$\{A_{\lambda}\}_{\lambda \in \Lambda}$を$C^*$-環の族とする.これに対し,集合$A$を
\begin{center}
$A=\{ \left( x_{\lambda}\right)_{\lambda \in \Lambda}|\forall x_{\lambda}\in \Lambda ,
\sup_{\lambda \in \Lambda}<\infty \}$
\end{center}
で定める.$A$に和と積,スカラー倍,及びノルムを
\begin{itemize}
\item $\left( x_{\lambda}\right)_{\lambda \in \Lambda}+\left( y_{\lambda}\right)_{\lambda \in \Lambda}
=\left( x_{\lambda}+y_{\lambda}\right)_{\lambda \in \Lambda}$
\item $\left( x_{\lambda}\right)_{\lambda \in \Lambda}\left( y_{\lambda}\right)_{\lambda \in \Lambda}
=\left( x_{\lambda}y_{\lambda}\right)_{\lambda \in \Lambda}$
\item $\alpha \left( x_{\lambda}\right)_{\lambda \in \Lambda}=\left( \alpha x_{\lambda}\right)_{\lambda \in \Lambda}$
\item $\parallel \left( x_{\lambda}\right)_{\lambda \in \Lambda}\parallel=\sup_{\lambda \in \Lambda}\parallel x_{\lambda}\parallel$
\end{itemize}
で定めると,これは$C^:$-環である.$A$を$\{A_{\lambda}\}_{\lambda \in \Lambda}$の\textgt{直和}といい,
\begin{center}
$A=\sum_{\lambda \in \Lambda} \oplus A_{\lambda}$
\end{center}
と表す.
\Subsection{重要な$C^*$-環の例}
 $\Omega$を局所コンパクト空間とする.$C_{\infty}\left( \Omega\right)$を無限遠で消える$\Omega$上の連続関数全体の集合とする.これに和・スカラー倍と積,対合,ノルムを
\begin{itemize}
\item $\left(\lambda x+\mu y\right)\left(\omega\right)=\lambda x\left(\omega\right)+\mu y\left(\omega\right)$
\item $\left(xy\right)\left(\omega\right)=x\left(\omega\right)y\left(\omega\right)$
\item $x^*\left(\omega\right)=\overline{x\left(\omega\right)}$
\item $\parallel x\parallel=\sup\{|x\left(\omega\right)||\omega \in \Omega\}$
\end{itemize}
で定めると,$C_{\infty}\left( \Omega\right)$は可換$C^*$-環である.$\Omega$がコンパクトであるとき,かつその時に限り$C_{\infty}\left( \Omega\right)$はunitalである.\\
\\
 ※局所コンパクト空間$\Omega$上の連続関数$x$が\textgt{無限遠で消える}(\textgt{vanishing at infinity}):任意の正数$\epsilon >0$に対し,あるコンパクト集合$K\subset \Omega$が存在して,
\begin{center}
$\forall \omega \in \Omega\setminus K,\parallel x\left( \omega\right)\parallel<\epsilon$
\end{center}
が成り立つ.
\Section{Gelfand表現}
\Subsection{指標}
 $A$を可換$C^*$-環とする.写像$\pi:A\rightarrow\mathbb{C}$が0写像でなく,$A$から$\mathbb{C}$への*-準同型であるとき,$\pi$を$A$の\textgt{指標}といい,$A$の指標全体を$\Omega\left(A\right)$で表し,\textgt{指標空間}という.
 $\Omega\left(A\right)$に弱*位相,すなわち,$A$を$A^{**}$の部分空間として考えた時に,各$x\in A$を連続にする$\Omega\left(A\right)$上の位相であって,和,積,スカラー倍を連続にするようなもののうち最弱なものとする.すると,この位相に関し,$\Omega\left(A\right)$は局所コンパクトHausdorff空間になる.さらに,$A$がunitalならば$\Omega\left(A\right)$はコンパクトである.
\Subsection{Gelfand表現}
 写像$\mathscr{F}:A \rightarrow C_{\infty}\left(\Omega\left(A\right)\right)$を
\begin{center}
$\mathscr{F}\left(x\right)\left(\pi\right)=\pi\left(x\right)$
\end{center}
で定める.$\mathscr{F}$を$A$の\textgt{Gelfand表現}という.
\Subsection{Gelfand-Naimarkの定理その1}
\begin{theo}
$A$を可換$C^*$-環とする.$A$の{\rm Gelfand}表現$\mathscr{F}$は$A$から$C_{\infty}\left(\Omega\left(A\right)\right)$への等距離*-同型である.
\end{theo}
 このGelfand-Naimarkの定理により,任意の可換$C^*$-環に対し,ある局所コンパクトハウスドルフ空間上の連続関数環が対応することがわかった.実は局所コンパクトハウスドルフ空間上の連続関数環から元の位相空間を復元することができるので,この定理は可換$C^*$-環から局所コンパクトハウスドルフ空間を構成できることを示している(その逆も然り).すなわち\textgt{,可換$C^*$-環の理論は局所コンパクト(ハウスドルフ)空間の理論と等価}とみなしてよいということである.(余談ではあるが,圏論の言葉を用いれば,局所コンパクトハウスドルフ空間の圏と可換$C^*$-環の圏が圏同値である,ということである)\\
 かくして可換$C^*$-環と局所コンパクト空間という幾何学的な概念が繋がった.この定理はGrothendieckのスキーム論にも影響を与えたと言われている.\\
 では,この定理から可換ではない$C^*$-環も何か幾何学的なものに対応しているのではないか,と考えてみる.それはどんなものであろうか.可換$C^*$-環から局所コンパクトハウスドルフ空間を構成できるならば、同じ手続きによって非可換な$C^*$-環から「空間」を構成できるののではないだろうか.それはもはや「点」や「空間」といった概念が意味を成すのかわからないが,ともかく非可換$C^*$-環により「構成」した「空間」に相当する何かを「非可換空間」ということにしよう.「非可換空間」は$C^*$-環から構成されるので,関数環側から微分構造やRiemann計量を導入する方法がわかれば,非可換微分多様体や非可換Riemann多様体が考えられる.「非可換空間」について知るためにはその基となる非可換な$C^*$-環について知らねばなるまい.そこで,もう少し一般の可換とは限らない$C^*$-環について調べていく.
\Section{GNS表現}
\Subsection{$C^*$-環の表現}
 $A$を(可換とは限らない)$C^*$-環,$\mathfrak{H}$をHilbert空間とする.*-準同型$\pi:A\rightarrow \mathcal{B}\left(\mathfrak{H}\right)$に対し,対$\left(\pi,\mathfrak{H}\right)$をAの\textgt{表現}という.$\pi$が単射であるとき,表現$\left(\pi,\mathfrak{H}\right)$は\textgt{忠実}であるという.
\Subsection{表現の直和}
 $C^*$-環$A$の表現の族$\left(\pi_{\lambda},\mathfrak{H}_{\lambda}\right)_{\lambda \in \Lambda}$を考える.$\mathfrak{H}_{\lambda}$の直和Hilbert空間を,$\mathfrak{H}$;
\begin{center}
$H:=\bigoplus_{\lambda \in \Lambda}\mathfrak{H}_{\lambda}$
\end{center}
とする.これに対し,表現の\textgt{直和}$\left(\pi,\mathfrak{H}\right)$を
\begin{center}
$\pi\left(x\right)\left(\left(\xi_{\lambda}\right)_{\lambda \in \Lambda}\right):=
\left(\pi_{\lambda}\left(x\right)\xi_{\lambda}\right)_{\lambda \in \Lambda}$
\end{center}
で定める.
\Subsection{状態}
 $A$を(可換とは限らない)$C^*$-環とする.$A$の線型汎函数$\omega:A\rightarrow \mathbb{C}$が
\begin{center}
$\forall x\in A,\omega \left(x^*x \right)\geq0$
\end{center}
をみたすとき,$\omega$を\textgt{正線型汎函数}という.正線型汎函数は有界線型作用素である.$\parallel\omega\parallel$=1をみたす正線型汎函数を\textgt{状態}という.
\Subsection{GNS表現の構成}
 $A$を(可換とは限らない)$C^*$-環とする.$A$の正線型汎函数$\omega:A\rightarrow \mathbb{C}$が与えられたとき,これを用いて$A$の表現を構成する.
\begin{center}
$N_{\omega}:=\{x\in A|\omega\left(x^*x\right)=0 \}$
\end{center}
とすると,これは$A$の左イデアルであり,かつ閉集合である.$N_{\omega}$を$\omega$の左核という.\\
 $x\in A$に対し,$\eta_{\omega}\left(x\right)$で商空間$A/N_{\omega}$の剰余類$x+N_{\omega}$を表すこととする.複素線形空間$A/N_{\omega}$に内積を
\begin{center}
$\left(\eta_{\omega}\left(x\right)|\eta_{\omega}\left(y\right)\right)=\omega\left(y^*x\right)$
\end{center}
で定める.この内積に関して$A/N_{\omega}$を完備化して得られるHilbert空間を$\mathfrak{H}_{\omega}$とする.\\
 各$a\in A$に対し線型作用素:$A/N_{\omega}\ni \eta_{\omega}\left(x\right)\mapsto \eta_{\omega}\left(ax\right)\in A/N_{\omega}$を考えると,これはHilbert空間$\mathfrak{H}_{\omega}$上の有界作用素$\pi_{\omega}\left(a\right)$に拡張できる.そこで写像$\pi_{\omega}:A\ni a\mapsto \pi_{\omega}\left(a\right)\in \mathcal{B}\left(\mathfrak{H}_{\omega}\right)$
を考えると,対$\left(\pi_{\omega},\mathfrak{H}_{\omega}\right)$はAの表現である.この表現を$\omega$による\textgt{Gelfand-Naimark-Segal表現},略して\textgt{GNS表現}という.この表現の構成法を\textgt{GNS構成法}という.
\Subsection{普遍表現}
 $A$を(可換とは限らない)$C^*$-環とする.$A$上の状態$\omega$全てについての表現の族$\left(\pi_{\omega},\mathfrak{H}_{\omega}\right)$の直和を,$A$の\textgt{普遍表現}という.
\Subsection{Gelfand-Naimarkの定理その2}
\begin{theo}
任意の$C^*$-環$A$の普遍表現は忠実である.特に,$A$の忠実な表現が存在する.したがって,任意の$C^*$-環$A$はある{\rm Hilbert}空間$\mathfrak{H}$上の有界線型作用素のなす$C^*$-環$\mathcal{B}\left(\mathfrak{H}\right)$と等距離*-同型である.
\end{theo}
 これで可換とは限らない任意の$C^*$-環があるHilbert空間上の有界線型作用素のなす$C^*$-環と対応付けられることがわかった.実はHilbert空間は量子力学が展開される空間であり,有界線型作用素は量子力学における物理量を表す役割を果たしている.作用素の積は一般に非可換であり,その非可換性が実際に物理学の中で大きな役割を果たしている.したがって,そのような意味で非可換な$C^*$-環は非可換な量子力学的空間と対応しているとみなせる.なお,状態という言葉は量子力学に由来する.\\
 その他の例として,局所コンパクト空間上の力学系を考えると,図形の空間的な情報と力学系による時間発展の情報の両方を持つ非可換なC*-環が得られる.\\
 このように非可換な$C^*$-環は非可換な空間の情報を持ったものであり,それを調べることにより,「非可換な空間」を得ることができる.
\Section{非可換空間の例}
 ここまで非可換空間と非可換な$C^*$-環が対応するのだろう、という話を見てきたが、簡単な例を用いて実際に非可換な空間を考えて、$C^*$-環が現れることを見ることにする.
\Subsection{n個の点のなす空間}
 一般の空間を考えてその上の運動を考えてもよいが、簡単のためにn個の点からなる空間を考えよう.一般の方のイメージする空間からはおよそかけ離れたものであるから,「空間」という言葉に抵抗のある方もいらっしゃるかもしれないが,そういう方はn箇所の駅からなる路線と各駅の間を電車が直通で結んでいるようなイメージなどをしてもらうといいかもしれない.\\
\Subsection{点の運動}
 各点に便宜上1,2,…nという名前を付けよう.この空間の中での運動を考える.各点iからi自身への運動,つまり運動とは書いているが静止したまま動かないこと,点iから点jへの運動が考えられる.また,点iから点jへの運動の「逆の運動」は点jから点iへの運動である.さらに運動の合成として,点iから点jへの運動と点jから点kへの運動の連続試行:点iから点kへの運動を考えよう.この運動の合成は,もちろん最初の運動の終点と次の運動の始点が同じでなければならないから一般には可換ではない.また,このような運動の中で禁止されているものはないとしよう.点iから点jへの運動を$e_{ji}$と表すことにして,これらのことを式で表すと
\begin{center}
$e_{ji}^*=e_{ij}, e_{kj}e_{ji}=e_{ki}$
\end{center}
となる.これはn次の行列単位($\left(i,j\right)$-成分のみが1で他の成分はすべて0であるようなn次正方行列)に他ならない.こうしたn次行列単位全てを含むような$\mathbb{C}$上の最小の体系は複素数を成分とするn次正方行列の全体$M\left(n;\mathbb{C}\right)$である.従って,このようなn個の点からなる空間での点から点への運動の体系を記述する情報は$M\left(n;\mathbb{C}\right)$内に記録されているはずであろう.そして,$M\left(n;\mathbb{C}\right)$の最小性からこれより小さなものでは情報すべてを記録できない.この$M\left(n;\mathbb{C}\right)$は行列の和と積,複素数倍によりBanach環であり,対合として随伴行列,つまり$A\in M\left(n;\mathbb{C}\right)$の対合$A^*$を複素共役の転置行列$A^*={}^{t}\overline{A}$と定めると,$M\left(n;\mathbb{C}\right)$は非可換な$C^*$-環である.$M\left(n;\mathbb{C}\right)$はHilbert空間$\mathbb{C}^n$上の有界線型作用素のなす$C^*$-環である(Gelfand-Naimarkの定理その2.まぁ,使うまでもなくご存知の結果かもしれませんが…).
\Subsection{2点空間上の関数}
 では,このn点空間の非可換性を見ていきたい.ここでは簡単のために$n=2$として2点空間$\{a,b\}$を考える.この空間の上で定義された関数を考えよう.とはいえ,2点しかない空間なのでa,bそれぞれに対して関数の値を決めればいい.2点空間$\{a,b\}$上の関数$f$を
\begin{center}
$f\left(a\right)=\alpha, f\left(b\right)=\beta$
\end{center}
で定める.ここで定数関数${\bf 1}:a,b\mapsto 1$と$a$で$1$,$b$で$0$という値を取る関数$e$を考えれば,2点空間上の任意の関数は
\begin{center}
$f=\alpha e+\beta \left({\bf 1}-e\right)$
\end{center}
と表せる.このまま積を考えても所詮は2点空間なので可換であるが,仮に「微分」を定義できたとすると,2点空間は非可換になってしまう.
\Subsection{非可換微分構造}
 2点空間上の関数に「微分」$D$が定義できたとしよう.「微分」$D$は微分であるから,次の規則を満たさなければならないだろう.
\begin{itemize}
\item$D{\bf 1}={\bf 0}$
\item$D\left(fg\right)=\left(Df\right)g+f\left(Dg\right)$
\end{itemize}
このような規則をみたす「微分」が定義できると,上で定めた関数$f=\alpha e+\beta \left({\bf 1}-e\right)$の「微分」は
\begin{center}
$df=\left(\alpha-\beta\right)$
\end{center}
となる.\\
 さて,ここで$f$として$e^2$を考えよう.定義から明らかに$e^2=e$である.この両辺を「微分」すると
\begin{center}
$\left(de\right)e+ede=de$
\end{center}
この式の両辺から$2eDe$を引けば
\begin{center}
$\left(de\right)e-ede=\left({\bf 1}-2e\right)de$
\end{center}
を得る.$de$が0でないとすると右辺が$0$でないから,関数eとその導関数deの積が交換可能でないことを意味している.普通の空間の上で微分を考えれば,これは交換可能なはずである! なんてこった! 2点空間は普通の空間じゃなかったんだよ!(な、なんだってー)これと類似の結果が一般のn点空間でも成り立つ(気力のある方はやってみると計算の練習になる,かも?).
\Subsection{微分について}
 上で述べた「微分」について,詳しく立ち入るつもりはないが,そのような関数から関数への写像があること,すなわち上の規則をみたす「微分」の存在について述べておく.通常の多様体上の微分形式や微分とDirac作用素との交換関係の類似性をご存知の方はそれを思い出されると,納得しやすいかと思う.\\
 $C^*$-環がGelfand-Naimarkの定理その2により対応するHilbert空間を$\mathfrak{H}$を考え,一般に$C^*$-環の元と非可換な$\mathfrak{H}$上の有界線型作用素$D$を一つ取り(存在の議論は省略する),$C^*$-環の元$f$の微分を
\begin{center}
$df:=Df-fD$
\end{center}
により定める.これは先の「微分」がもつべき性質を満たしている.\\
 先の2点空間上の関数のなす$C^*$-環は$\mathbb{C}^2$上の有界線型作用素のなす$C^*$-環$\mathcal{B}\left(\mathbb{C}^2\right)=M\left(2;\mathbb{C}\right)$と対応している.$f=\alpha e+\beta \left({\bf 1}-e\right)$は$M\left(2;\mathbb{C}\right)$の元の中で対角成分が$\alpha,\beta$である2次対角行列に対応している.これに対して適当に対角行列でない行列をとり,それを$D$とすれば,dfは先の定義で確かに微分になる(興味のある方は計算してみてください).
\Subsection{非可換トーラス}
 先のn点空間は微分を考えれば非可換になったが,関数の積自体は可換で,点も具体的に考えられた($C^*$-環から構成せずに直接空間を定義したので当然ではあるが…)そこでもう少し複雑な例を見ることにしよう.通常のトーラスをもとにした非可換トーラスを考えることができる.簡単のため2次元での「お話」のみを紹介する.トーラス上の関数は周期関数であるから,2次元トーラス上の関数は座標を$\left(x,y\right)$として
\begin{center}
$f=\sum_{k,l} a_{kl}e^{ikx}e^{ily}$
\end{center}
と書くことができる.この$e^{ix},e^{iy}$に対し,非可換積を
\begin{center}
$e^{ix}*e^{iy}=e^{i\theta}e^{iy}*e^{ix}$
\end{center}
などで定めて,$e^{ix},e^{iy}$により生成される代数は非可換である.これに対応する非可換空間を非可換トーラスと定める.
\Section{あとがき}
 ここまで$C^*$-環から始まって.非可換空間の紹介までしてきましたが,そんなもの何に使うの?という話だけを最期に少しさせていただきます.先に述べた通り,まず物理の量子力学とかかわりがあるのでそちらへの応用が期待されます.また,先のN点空間上の話は格子空間(これは$N^4$点空間です)について,その上で微分を考えると非可換が表れてしまうことを意味しています.さらには超弦理論も非可換幾何学の性質を持つと言われています.こうした物理学やその背後によって導かれたその先に,非可換幾何学の世界が待っている…のかもしれません.
 あまり「非可換幾何学」及びその入り口に具体的に触れる,ということはできませんでしたが,作用素環の基礎からはじめて非可換幾何学の着想の元となる定理までを紹介いたしました.少しでもその雰囲気を味わっていただけたなら幸いです.

\Subsubsection{参考文献}
\begin{description}
\item{[1]}Masamichi Takesaki「Theory of Operator Algebras」,Springer
\item{[2]}Shoichiro Sakai「$C^★$-Algebras and $W^★$-Algebras」,Springer
\item{[3]}梅垣壽春,大矢雅則,日合文雄「復刊 作用素代数入門」,共立出版株式会社
\item{[4]}生西明夫,中神祥臣「作用素環入門I 函数解析とフォン・ノイマン環」,岩波書店
\item{[5]}竹崎正道,「作用素環の構造」,岩波書店
\item{[6]}綿村哲「非可換幾何学と場の理論」日本物理学会誌vol55,No10,2000
\end{description}

%まず最初に使ったプリアンブルをここに書いてください.
%ただしコンパイルの都合上コメントアウトしてください.
%実際に確認する際は,各自の環境でmain.texにこのプリアンブルを追加してください.

%\usepackage{mathrsfs}
%\usepackage[all]{xy}
%\newcommand{\proofend}{\begin{flushright} $\blacksquare$ \end{flushright}}
%\renewcommand{\labelenumi}{(\roman{enumi})}
%\newcommand{\nkgr}{・}
%\theoremstyle{definition}
%\newtheorem{theorem}{定理}
%\renewcommand{\thetheorem}{}
%\newtheorem{defi}{定義}
%\newtheorem{thm}[defi]{定理}
%\newtheorem{lem}[defi]{補題}
%\newtheorem{cor}[defi]{系}
%\newtheorem{prop}[defi]{命題}
%\newtheorem{ex}[defi]{例}


\Chapter{代数学の基本定理でみる数学の世界(伊藤)}
% タイトル(名前)でお願いします.
% セクションは \Section \Subsection \Subsubsection で分けてください.
% 詳しくはMAY2015を参考にしてください.
\Section{はじめに}
数学科展示ますらぼにお越しいただきましてありがとうございます.
数学科とはどのようなことをやっている学科なのか一般の人に説明するのはなかなか難しく,
一般の人に端的な説明を求められるとなかなか四苦八苦してしまうところがあります.
この記事では数学科がどのようなことを勉強しているのかについて,
\textbf{代数学の基本定理}という定理を題材に出来るだけわかりやすく説明したいと思います.
この記事は第7回関西すうがく徒のつどいにおける拙講演「代数学の基本定理でみる数学の世界」を
更に詳しくして紙面化したものですので,講演に関してまとめたウェブ上の記事 \url{http://togetter.com/li/878845} も参考にしていただければと思います.
\Section{代数学の基本定理とは}
代数学の基本定理とは
\thm
次数が$1$以上の複素係数一変数方程式には複素根が存在する
\thmx
という定理です.具体的にはどういうことを言っているのでしょうか.例を見てみましょう.
\ex
$2x-4=0$というのは$1$次方程式ですが$x=2$という解を持ちます.
\exx
\ex
$ax^2+bx+c=0$というのは$2$次方程式ですが,この方程式の解の公式も中学校で習ったことでしょう.
\exx
\ex
$3$次多項式と$4$次方程式にも,極めて難解ですが解の公式というものが知られています.
これについては``カルダーノの公式''や``フェラーリの公式''で調べてください.
\exx
これらの解の公式とは\underline{具体的にバッチリと解のありかを求める}公式です.
一方で代数学の基本定理とは複素根が存在すると言っているだけなので,\underline{どこにあるかは分からないけど}\\\underline{とりあえず存在はするよ}という定理なんです.
しかし,これはどんな方程式にも解があるということを言っているのでそれは強い主張であるともいえます.
この定理は$1600$年ごろに様々な数学者によって予想され,$1800$年ごろにガウスによって証明がされました.
代数学の基本定理は高校生でも証明できるような定理なのですが,その基本的さ故に様々な証明があり,
大学$3,4$年生で習うようなことを使っても証明することができます.
この代数学の基本定理と共に大学の数学とはどのようなものなのかを見てみましょう.
\Section{大学1年生}
大学$1$年生で習う数学とは``解析学入門''と``線形代数学''の$2$つです.
どちらも数学の基礎であるとともに理系の多くの学科でも使われるものです.
\Subsection{解析学入門}
解析学入門は東大では``数学$1$''という科目名で開講されています.
微分と積分について現代数学的に学び直そうという科目です.
意識高く大学で勉強をしようと思っていた東大の$1$年生たちの多くがこの科目に打ちのめされて俗にいう五月病に羅患します.
この解析学のつまづきやすい$2$つのポイントとして,\large{$\varepsilon$-$\delta$論法}と\large{コンパクト}というものがあります.
この$2$つについて見てみましょう.
\Subsubsection{$\varepsilon $- $\delta$ 論法}
高校数学にも極限という概念はあって,$x$が$0$に限りなく近づくとか$n$が$\infty$に発散するとかいう言葉が使われています.これを厳密に定義しようというのが$\varepsilon$ - $\delta$ 論法です.
本格的な$\varepsilon$ - $\delta$ 論法に入る前に幾つか練習をしてみましょう.
\ex
ある$x$という実数の絶対値は全ての正の数$p>0$より小さいとします.これを数式で書くと以下のようになります.
\[
\forall p > 0 : \ |x| < p
\]
$\forall p$で全ての$p$についてということを言っているわけです.
ではこの$x$はどんな数なのでしょうか.$p=0.1$としてみても,$|x|$はこれより小さいです.$p=0.0001$としても$|x|$はこれより小さいです.
$p=0.000\cdots (0が2億個) \cdots 01$よりも$|x|$は小さいです.これはつまり$|x| = 0$ということです
$x$は$0$としたわけではないが,$0$になってしまった.これが現代数学の``限りなく近い''という概念をつかむのに大事な考え方です.
\exx
\ex
ある$x$という数は全ての正の数$p>0$よりも大きいとします.つまり,
\[
\forall p > 0 : \ x > p
\]
です.この$x$も具体的にはどんな数なのでしょうか.$p=1000$としてみても,$x$はこれより大きいです.$p=2億$としてみてもこれより大きいです.
これもやはり$x$は$\infty$であるということを示しているのではないでしょうか.$\infty$というのはきちんと定義されていませんが,
$x$は限りなく大きいとはこのような気分なんだなあということがイメージできます.
\exx
ではここで,$\varepsilon-N$論法というのを見てみましょう.
\defb[$\varepsilon$ - $N$ 論法]
$\lim_{n\to\infty} a_n = \alpha $\\
$\iff$
$\forall \varepsilon > 0,\  \exists N \in \N \quad \textrm{s.t.}\  \forall n \in \N,\  n > N \Rightarrow |a_n - \alpha | < \varepsilon $
\defe
突然数式がたくさん出てきて混乱したかもしれませんが落ち着いて見れば簡単です.\\
$\lim_{n\to\infty} a_n = \alpha $というのを定義しているわけです.\\
$a_n$が$\alpha$に限りなく近づくとはどういうことをなのでしょうか.\\
それは,例で見たとおり,$|a_n - \alpha|$が限りなく小さくなればいいわけです.
それを表すために,$\varepsilon > 0$というとても小さな数を$1$つ取ってきます.
それに対して,ある$N$を取ってきて$N$以降では$|a_n - \alpha | < \varepsilon$が成り立っているよとするわけです.\\
例えば,$1000$項目以降では,$|a_n - \alpha| < 0.001$が,$100000$項以降では,$|a_n - \alpha | < 0.0000000001$が成り立っていたら
どんどん近づいて行くような気がしますよね.これが限りなく近づくよ,ということを言うためにまず$\forall \varepsilon > 0$としているわけです.\\
\ex
$a_n = \frac{n+1}{n}$とすると$n \to \infty $でこれは$1$に収束します.\\
実際,$|a_n - \alpha| = | \frac{n+1}{n} - 1 | = | \frac{1}{n} |$です.\\
例えばこの$|a_n - \alpha|$を$0.01$より小さくしたい!と思えば,$N=100$としてあげれば,
$N$項目以降では$|\frac{1}{n}| < 0.01$が成り立つわけです.\\
ここでも,$|a_n - \alpha| $という差は$n$が大きくなるに連れてどんどん小さくなっていますね.
\exx
このような方法を採用するメリットとして,極限という概念がきっちりと定義されて,例えば,次のような明らかに成り立って欲しい極限の性質も厳密に証明する事ができます.
\prob
$\lim_{n\to\infty} a_n = \alpha , \lim_{n\to\infty} b_n = \beta $とする.\\
$\lim_{n\to\infty} (a_n + b_n) = \alpha + \beta , \lim_{n\to\infty} a_n b_n = \alpha\beta $
を示せ.
\probx
同様にして$\varepsilon-\delta$論法も見てみましょう.
\defb[$\varepsilon - \delta$ 論法による定義]
$\lim_{x \to a}f(x) = b$\\
$\iff$
$\forall \varepsilon > 0,\  \exists \delta > 0\quad \textrm{s.t.}\  \forall x \in \mathbb{R},\  0 < |x-a| < \delta \to |f(x)-b| < \varepsilon$
\defe
これも同様の考え方です.$x\to a\  (xがどんどんaに近づく)$のとき,$f(x) \to b\  (f(x)はどんどんbに近づく)$ということを定義しているわけです.また$|f(x)-b|$を限りなく小さくするために,$|x-a|$の幅を限りなく小さくとっているわけです.\\
またこの$\forall \exists$という並びは,どんな$\varepsilon (とても小さいイメージ)$に対してでも,いちいち$\delta (さらに小さいイメージ)$をとってくるということを表しています.
\prob
$\lim_{x\to 0} x^2 = 0$を証明せよ.$\varepsilon >0$として,$\delta = \sqrt{\varepsilon}$とすれば,$ 0 < | x | < \delta$ならば$ |x^2| < \varepsilon = \delta^2$を示せばよい. 
\probx
\Subsubsection{コンパクト}
次に第二のつまづきポイントであるコンパクトについて触れましょう.
高校数学で次のような定理があったことを思い出しましょう.
\thm[最大値最小値の定理]
$[a,b]$を有界閉区間,$f$を$[a,b]$上の実数値連続関数とする.
このとき$f$は最大値および最小値にそれぞれ少なくとも一点で到達する.
\thmx
これは高校数学では大した有り難みもない定理でしたが現代数学では重要です.
ここで重要なのは$[a,b]$が有界閉区間であるという仮定と,$f$は連続であるという仮定です.実際
\ex[非有界]
$\R$上で連続な関数$f(x)=x$は$\R$で最大値,最小値を持たない.
\exx
\ex[不連続]
$[-1,1]$上の関数.$f(x)=1/x$は最大値,最小値を持たない.
\exx
という例が示すように,有界閉区間または連続という仮定を外すとたちまちこの定理は成り立たなくなります.
この有界閉区間という概念を一般化したのがコンパクトです.
\defb[コンパクト]
$X$空間が$(点列)$コンパクトである\\
$\iff$
$X$内の任意の点列が$X$内に収束する部分列を含む
\defe
これも例を見てみましょう.
\ex
開区間$(0,1)$はコンパクトではない.なぜならば,$\{1/n\}$という数列は$0$に収束するが,この数列の部分列は$(0,1)$内の点に収束しない.
\exx
\ex
実数$\R$はコンパクトではない.なぜならば,$\{n\}$という数列は$\infty$に発散するが,この数列の部分列は$\R$内の点に収束しない.
\exx
\thm
$I \subset \R^n$がコンパクトであることと有界かつ閉であることは同値
\thmx
という風にコンパクトは有界閉区間の拡張になっているわけです.そして,
\thm
$I \subset \R^n$をコンパクト,$f$を$I$上の実数値連続関数とする.
このとき$f$は最大値および最小値にそれぞれ少なくとも一点で到達する.
\thmx
という定理が成り立ちます.こうして,$2$のポイントをおさらいしたところでその応用として代数学の基本定理を証明してみましょう.
\thm[代数学の基本定理]
次数が$1$以上の任意の複素係数一変数多項式$p(z)=a_0+a_1 z+\cdots + a_nz^n$には複素根が存在する.
\thmx
\proof[初等解析による証明]
これは杉浦光夫「解析入門1」に載っている証明です.
証明のポイントは3つ.\\
$(1)\ \lim_{|z|\to\infty}|p(z)| = \infty$\\
$(2)\ |p(z)|$はコンパクト集合上で最小値を取る.\\
$(3)\ |p(a)|>0 \Rightarrow \exists b \in \C \ s.t. \ |p(b)| < |p(a)|$(下には下がいる)
です.
\[
\lim_{|z|\to\infty}|p(z)| = \infty
\]
という意味をもう一度解釈してみましょう.
\[
\forall M \in \R \ \exists R>0 \ s.t. \  |z| > R \ \Rightarrow\  |p(z)|>M
\]
ということでした.そして$M$は任意ですから,$M=|p(0)|$として,それに対して$R>0$を一つ取り,
$K=\{ z\in\C |\ |z|\le R\}$とおけば,$K$の外では$|p(z)|>M$が成り立ちます.
つまりこの$K$の中で最小値を探せばいいいわけです.ところで$K$はコンパクトであるので
\begin{center}
$|p(z)|$は$K$上で最小値を取る
\end{center}
が言えます.最後に,
\[
|p(a)|>0 \Rightarrow \exists b \in \C \ s.t. \ |p(b)| < |p(a)|
\]
が言えて$(この証明は杉浦に譲ります)$証明終了.
\proofx
代数学の基本定理の証明は$1$年生の解析の大事な部分を使って得られるのでした.
\begin{itembox}[l]{解析学入門のまとめ}
$\varepsilon$-$\delta$論法は極限の概念を厳密化するもの.コンパクトは有界閉区間を一般化したもの.
この$2$つの概念を使って代数学の基本定理は証明できる.
\end{itembox}
\Section{大学2年生}
\Subsection{解析学続論}
大学$1$生では他に線形代数という科目を勉強しますが,この記事には関係ないので割愛します.
大学$2$年生では$1$年生で習った解析学と線形代数学の発展について学びます.
解析学では多変数の解析について学びます.ここでは線積分というものについて触れましょう.
今まで積分といえば,$\int_a^b$と言ったように区間$[a\ b]\subset \R$上での積分を考えてきましたが,
例えば,円周$\{(x,y)\in\R^2 | x^2+y^2=1\}$にそってある関数を積分したいということは数学だけでなく
多くの理系分野でよくあることです.まず曲線とは何かについて考えてみましょう.
\defb
$I\subset\R$を区間とします.$\phi:I \to \R^n$が空間曲線であるとは,一対一の連続写像であるこということである.
\defx
一対一というのは,$a\neq b \Rightarrow \phi(a) \neq \phi(b)$であるということで,つまりは自己交差をしないということです.
確かに自己交差をしなくてちゃんと繋がっていなくては曲線とはいえませんね.
\ex
$\phi:[0,1] \to \R^2$を$\phi(t)=(t,t)$で定める.これは$(0,0)$と$(1,1)$を結ぶ直線であり,空間曲線である.
\exx
\ex
$\phi:[0,2\pi) \to \R^2$を$\phi(t)=(\cos t ,\sin t)$で定める.これは単位円周です.
\exx
それではこれらの曲線にそった積分というのを次で定めます.
\defb
$f:\R^n \to \R$を関数,$\phi:I\to\R^n$を滑らかな曲線として,この曲線の像を$C$で表す.曲線$C$に沿った$f$の線積分を以下で定義する.
\[
\int_C f(x)ds := \lim_{d(\Delta)\to 0} \sum_{i=1}^N f(\phi(\xi_i)) |\phi(t_i) - \phi(t_{i-1})|
\]
ただし,ここでの$\Delta$とは区間$I$の分割$t_0 < t_1 < \cdots < t_{N-1} < t_N$を考えており$d(\Delta)$はその分割の最も大きい幅です.
\defe
これはリーマン積分の考え方を使った積分の定義であり,詳しくは$e\pi isode\ vol.3$の``積分の歩み''を参照していただきたいのですが,
基本的には高校でならった区分求積法の考え方と同じで,区分求積法はある区間を同じ幅で分割していましたが,それを好きな幅で分割して良いようにしたという話です.またこの積分は収束して以下のようにも表されます.
\prop
$f,\phi,C$を上の定義と同様とし,$I=[a,b]$となるときに次が成り立つ.
\[
\int_C f(x) ds = \int_a^b f(\phi(x))|\phi'(t)|dt
\]
\propx
\ex
$f:\R^2\to\R$を$f(x)=1$という定数関数にして,$\phi:[0,1] \to \R^2$を$\phi(t)=(t,t)$で定める.\\
このとき
\[
\int_C f(x) ds = \int_0^1 |(1,1)| dt = \int_0^1 \sqrt{2}  dt  = \sqrt{2}
\]
です.この積分は$(0,0)$と$(1,1)$を結ぶ直線の長さ$\sqrt{2}$を求めています.
\exx
\Subsection{複素解析}
(複素解析については,この$e\pi isode$にある荒田さんの記事が大変参考になります.)
代数学の基本定理の証明方法に複素解析的な方法を使ったものが有名です.
複素解析は今まで実数関数でやってきたことを複素数の範囲に拡張することによって色々な美しい結果が得られる学問です.
複素解析の主な研究対象には正則関数というものがあります.まずそれを定義しましょう.
\defb[正則関数]
$f:\C \to \C$が正則であるとは各点で微分係数を持つということである.つまり,
\[
f'(z) = \lim_{h\to 0} \frac{f(z+h) - f(z)}{h}
\]
が各$z\in\C$で収束するということである.
\defx
\rem
これだけでは普通の実数の微分可能関数と変わらないではないかと思うかもしれませんが次のようことが成り立つことに注意しなければなりません.
つまり,$h\to 0$としていますがこの$h$は複素数なので色々な$0$への近づき方をするということです.
$h=x+yi$とおいて実部と虚部に分けたとき,$y=0,x\to 0$として$0$に近づけたときこれは偏微分$\frac{\partial f}{\partial x}$になります.
一方で$x=0,y\to 0$として$0$に近づけたとき,
\[
f'(z) = \lim_{y\to 0} \frac{f(z+yi)-f(z)}{yi} = -i\frac{\partial f}{\partial y}
\]
であり,
\[
f'(z) = \frac{\partial f}{\partial x} = -i\frac{\partial f}{\partial y}
\]
が成り立つ必要があります.この関係をコーシー・リーマンの関係式といいます.
\remx
次に$\C$の部分集合として重要な単連結領域というのを定義しますが,これは``便利な領域''として考えて頂いて構いません.
\defb[単連結領域]
$D \subset \C$が単連結領域であるとは,連結な開集合であって$D$内の任意の閉曲線は$1$点にホモトピックであるようなものである.
\defx
$1$点とホモトピックであるとはこの記事の後半を参照していただきたいのですが,単連結領域とは穴がない領域をイメージしてください.
そうすると以下のように重要な定理が成り立ちます.
\thm[コーシーの積分定理と積分公式]
$D$を単連結領域とし、$f(z)$ は $D$ 上で正則である複素関数とするとき、$C$ を $D$ 内にある長さを持つ単純閉曲線とする.
\[
 \oint_C f(z) \, dz\ = 0
\]
$a$をまた$C$によって囲まれる領域に属する点とする.
\[
 f(a) = \frac{1}{2 \pi i}\int_C \frac{f(z)}{z-a}dz
\]
\[
 f^{(n)}(a) = \frac{n!}{2 \pi i}\int_C \frac{f(z)}{(z-a)^{n+1}}dz
\]
\thmx
この定理の意味とは$f(z)$が正則であれば,どんな閉曲線上で積分してもその値は$0$になるということと,
逆に$a$という一点でだけ正則でないような$\frac{f(z)}{z-a}$という関数を積分するときは$f(a)$の値のみを考えればいいよという意味です.
実際に例を見てみましょう.
\ex[コーシーの積分公式の例]
$f(z) = 1 , C = \{z\in\C | |z-a|=r\} $ のときコーシーの積分公式.\\
\[
1^{(n)}(a) = \frac{n!}{2 \pi i}\int_{|z-a|=r} \frac{1}{(z-a)^{n+1}} dz 
\]
となりますが
\[
 \int_{|z-a|=r} \frac{1}{(z-a)^{n+1}} dz =
  \begin{cases}
   \  2\pi i  \ \ (n=0) \\
   \  0 \ \ (n \ge 1) \\
  \end{cases}
\]
に他ならなりません.
\exx
正則関数がどれだけ関数に強い条件を課しているかというのは次の定理でわかります.
\thm[リュービルの定理]
複素平面全体で正則かつ有界な関数は定数関数のみ.
\thmx
\proof
\leavevmode\\
この証明は,藤本坦孝「複素解析」に載っている証明です.
証明のポイントは以下の$3$つです.\\
$(1)\ f有界つまり\forall z \in \C :\ |f(z)|\le M $かつ$f$正則を仮定する.\\
$(2)\  $仮定を満たす関数は正則より$f(z)=\sum_{n=0}^\infty c_n z^n $とべき級数展開可能\\
$(3)\ c_n$は$\forall R>0 : \ |c_n|\le \frac{M}{R^n}$をみたす.(ここでコーシーの積分公式が使われている)
\proofx
このリュービルの定理を用いて代数学の基本定理を証明する事ができます.
\proof[リュービルの定理を用いた代数学の基本定理の証明]
\leavevmode\\
この証明はLars Valerian Ahlfors「Complex Analysis」に載っている証明です.
証明のポイント:\\
$(1)\ p(z)=a_n z^n + \cdots + a_1 z+ a_0$が零点を持たないと仮定する(背理法)\\
$(2)\ g(z) = \frac{1}{p(z)}$は$\C$上で正則となる\\
$(3)\  \lim_{|z|\to\infty} |g(z)| = 0$となる.\\
$(4)\ $上から$g:$有界であることが言え,Liouvileより定数となり矛盾.
\proofx
複素解析の一つの目標として留数計算というものがあります.コーシーの積分公式では分数型の$1$点のみで正則でない関数の積分を考えましたが,
今度は他の形の正則でない点が複数ある場合でも積分計算をしてみようというというものです.
\defb[留数]
$f$が環状領域$\Delta(a,r,R) = \{z\in\C | \  r<|z-a|<R\}$で正則とする.このとき\\
\[
f(z) = \sum_{n=-\infty}^{\infty} a_n (z-a)^n
\]
という風に展開できて,これを$f$のローラン展開という.\\
特に$\Delta(a,0,R)$で正則$(a$のみ孤立して正則でない$)$とき,\\
$a_{-1}$のことを$f$の$a$での留数といい$\Res_{a} f$とかく.\\
\defx
\ex
$\frac{1}{z-c}$という関数を$|z|>|c|$でローラン展開すると.
\[
\frac{1}{z-c} = \frac{1}{z} + \frac{c}{z^2} + \frac{c^2}{z^3} + \cdots
\]
\exx
\thm[留数定理]
$D:$区分的$C^1$境界を持つ領域.$f:\overline{D}\setminus\{p_1,\cdots,p_n\}$で正則とする.
\[
\frac{1}{2 \pi i}\int_{\partial D} f(z)dz = \sum_{i=1}^n Res_{p_i} f
\]
\thmx
この留数定理とは,$f$という関数を積分する際は,$p_i$という点での留数のみを考えればいいよと言っているわけです.
留数を計算するのに便利な次の公式を紹介します.
\prop
$(1)\ z=a$に於いて$\lim_{z\to a} (z-a)f(z)$が有限確定値を持つとき,
\[
\Res_{a} f = \lim_{z\to a} (z-a)f(z)
\]
$(2)\ z=a$に於いて$\lim_{z\to a} (z-a)^m f(z)$が有限確定値を持つとき,
\[
\Res_{a} f = \frac{1}{(m-1)!} \lim_{z\to a} \frac{d^{m-1}}{dz^{m-1}} ((z-a)^mf(z))
\]
$(3)\ g,h$を正則関数として,$g(a)\neq 0,h(a)=0,h'(a)\neq 0$ならば
\[
\Res_{a} \frac{g}{h} = \frac{g(a)}{h'(a)}
\]
\propx
$\lim_{z\to a} (z-a)^m f(z)$が有限確定値を持つとき,$a$は$m$位の極であるといいますが,
$m$位の極の留数を計算するときは$(z-a)^m f(z)$という正則関数のテイラー展開を考えてあげればいいという話です.
正則関数の零点に関して次のような定理が成り立っています.
\thm[偏角の原理]
$D:$今までと同様.$f:$正則とする.
\[
\frac{1}{2 \pi i} \int_{\partial D} \frac{f'(z)}{f(z)}dz = (f\mbox{の}D\mbox{内の重複度込みの零点の個数})
\]
\thmx
\thm[ルーシェの定理]
$D:$区分的に$C^1$な境界を持つ有界領域\\
$f,g:D$とその境界上で定義された正則関数.\\
$\forall z \in \partial D :\ |f(z)-g(z)|<|f(z)|+|g(z)|$が成り立つとする.\\
このとき,$f$と$g$の零点の個数は等しい.
\thmx
ルーシェの定理は$f$と$g$の零点の個数を見たいときにその境界上のみで$f,g$の様子を考えて上げればいいという定理です.
\proof
\leavevmode\\
定理・証明ともに平地健吾先生に教えて頂きました.
証明のポイント\\
$(1)\ F_t(z) = (1-t) f(z) + t g(z)$は$0$にならない\\
$(2)\ N_t=\int_{\partial D} \frac{F_t'(z)}{F_t(z)}dz$は偏角の原理より$F_t$の零点の個数だがこれは$t$について連続.\\
$(3)\ (f$の零点の個数$)=N_0=N_1=(g$の零点の個数$)$
\proofx
%誰か良い例ください
この定理を用いて代数学の基本定理を証明する事ができます.
\proof[ルーシェの定理を用いた代数学の基本定理の証明]
\leavevmode\\

$f(z)=a_n z^n + \cdots + a_1 z+ a_0$と$g(z)=a_n z^n$とおく.\\
$|f(z)-g(z)|$は$n-1$次式,$|f(z)|+|g(z)|$は$n$次式より,\\
十分大きな円周上では$|f(z)-g(z)|<|f(z)|+|g(z)|$が成り立つ.\\
よって$f$の零点の個数は$n$個
\proofx
\prob
実はルーシェの定理まで行かなくても偏角の原理のみで代数学の基本定理を証明する事ができます.各自考えて見てください.
\probx
\begin{itembox}[l]{複素解析のまとめ}
複素解析は正則関数という性質のよいものを扱う学問.複素関数の積分は正則でない点の留数のみを見ることによってできる.
リュービルの定理は正則関数の条件の強さを表し,偏角の原理やルーシェの定理は正則関数の零点の個数を調べるものである.
これらを使って代数学の基本定理は証明できる.
\end{itembox}
\Subsection{集合と位相}
ここでは位相空間論というものについて触れましょう.今までは$\R^n$のみで連続や収束という概念を考えて来ましたが,これを任意の集合に対して
扱えるようにするのが位相空間論の一つの目標です.

\defb[位相空間]
$X$を集合とする.$X$の部分集合からなる集合$\mathcal{O}$が$X$の開集合系である.\\
$\iff (1) (U_i)_{i\in I}$が $\mathcal{O}$の族ならば,$\cup_{i\in I} U_i \in \mathcal{O}$\\
$(2)(U_i)_{i\in I}$が $\mathcal{O}$の有限族ならば,$\cap_{i\in I} U_i \in \mathcal{O}$\\
また$\mathcal{O}$に属する元を$X$の開集合といい,$(X,\mathcal{O})$を位相空間という.
また閉集合とは開集合の補集合になっているものと定義します.
\defx
このようにして任意の集合に対して好きな開集合だけを集めてきて空間の構造を与えられることができるわけです.
\ex
自然数の集合$\N$に対して次のような位相を与えることができる.\\
$(1)$\ $\mathcal{O} = \{\emptyset , \N\}$\\
$(2)$\ $\mathcal{O} = \{U\subset\N | Uは\N の部分集合\}$\\
$(3)$\ $\mathcal{O} = \{ \N \setminus I | I\subset\N は有限部分集合\} \cup \{ \emptyset \}$\\
これらの例は全て開集合系の定義を満たしていますので,これらにより$\N$を位相空間とできます.\\
$(1)$を密着位相,$(2)$離散位相,$(3)$を補有限位相と言います.
\exx
ここで収束を位相空間の言葉で書いてみましょう
\defb
位相空間$X$の点列$\{x_n\}$が$x$に収束する
\[
\iff\ \forall U :xを含む開集合 \ \exists N \in \N \ s.t.\ n \ge N \Rightarrow x_n \in U
\]
\defx
こう見てみると$U$とは$\varepsilon-\delta$のように$\varepsilon$とっていることがわかります.
つまり,位相空間とは開集合によって,集合に近いという考え方を与えているわけです.
こう考えてみると,$(1)$の密着位相は全ての点が同じ開集合に属しており,近いところにいるという意味で``密着''しています.\\
$(2)$の離散位相は,全ての$2$点は別々の開集合に入れることができるので``離散''しています.\\
ここで,集合に$(1)(3)$の場合の収束について見て見ましょう.
\ex
$x_n = n$という$\N$内の点列を考える.\\
このとき,$(1)(3)$の位相構造において$x_n$は任意の点に収束する.\\
$(1)$の場合.例えば,$1$に収束することを示してみましょう.$1$を含む開集合は$\N$だけですから,
$n \ge 1 \Rightarrow x_n \in \N$が成り立ちます.よって,$x_n$は$1$に収束します.\\
$(3)$の場合.$U$を$1$を含む開集合とします.これは$\{a_1,\cdots,a_n\}$という有限集合の補集合になっています.\\
よってこれらの最大値を$N$とおくと,$n \ge N+1 \Rightarrow  x_n \notin \{a_1,\cdots,a_n\}\ (最大値より大きいので)$が成り立つので,
\[
\iff\ \forall U :1を含む開集合 \ \exists N \in \N \ s.t.\ n \ge N \Rightarrow x_n \in U
\]
が示せました.同様にして,$x_n$は任意の点に収束することがわかります.一方で$(2)$の場合は$U=\{1\}$という開集合に対して$N$が取ってこれないので任意の点に収束しません.
\exx
次のような扱いやすい空間が定義されます.
\defb
$X:$位相空間,$A\subset X$ がコンパクトである.$\iff$
\[
A \subset \cup_{i\in I} U_i \Rightarrow \exists \{i_1,...,i_n\} \subset I s.t. \ A \subset U_{i_1} \cup \cdots \cup U_{i_n}
\]
\defx
\ex
$(1)(3)$はコンパクトである.$(2)$はコンパクトではない.
\exx

位相空間がコンパクトであるという概念については斎藤毅「はじまりはコンパクト」や斎藤毅「集合と位相」に詳しく解説されています.
また,最初に定義した点列コンパクトとコンパクトという概念は一致します.
\defb
$X$がハウスドルフ空間である.$\iff$\\
$x,y \in X , x\neq y \Rightarrow \exists U:x\mbox{の開近傍} ,\exists V:y\mbox{の開近傍} \ s.t. \ U \cap V = \emptyset$
\defx
\ex
$(1)(3)$はハウスドルフではない.$(2)$はハウスドルフである.
\exx
また連結という概念も位相空間の言葉を使って定式化することができます.
\defb
$X$が連結空間である.$\iff$\\
$X$の部分集合で開集合かつ閉集合であるようなものは,$\emptyset$と$X$のみ.
\defx
\ex
$(0,1) \cup (2,3)\subset\R$は連結空間ではない.\\
実際,区間$(0,1)$は開集合でかつ,$(2,3)$という開集合の補集合になっているので閉集合です.\\
これはこの区間が繋がっていないことによって起こる結果です.
\exx
ここで連続写像の概念も極めてシンプルに一般化されます
\defb
$f:X\to Y$が連続写像である.$\iff$\\
$U \subset Y$が開集合ならば$f^{-1}(U) \subset X$は開集合.
\defx
次の定理は,連続写像の性質を表すとともに,最大値最小値の定理と中間値の定理を一般化したものとも言えます.
\thm
$X,Y$を位相空間として,$f:X\to Y$を連続写像とする.\\
$(1)$$A\subset X$がコンパクトならば$f(A)\subset Y$もコンパクトである.\\
$(2)$$A\subset X$が連結ならば$f(A)\subset Y$も連結である.\\
\thmx
\rem
一方で\\
$(3)A\subset X$が開集合ならば$f(A)\subset Y$も開集合である.\\
$(4)A\subset X$が閉集合ならば$f(A)\subset Y$も閉集合である.\\
はどちらも一般には成り立ちません.これらが成り立つ写像をそれぞれ.開写像,閉写像といいます.
\remx
\proof[位相空間論における代数学の基本定理の証明]
この証明は斎藤毅「集合と位相」に載っている証明です.\\
$f:\C \to \C$を多項式が定める写像とすると,\\
正則関数の一般論から$f$は開写像であることが言える.\\
また$\C$の一点コンパクト化である$\C P^1$を考えることにより,$f$は閉写像であることがわかります.\\
よって,$\C$は連結空間で$f(\C)\subset\C$は開かつ閉であり,空集合でないので$f(\C)=\C$がいえます.\\
つまり,この多項式には$0$点が存在します.
\proofx
\begin{itembox}[l]{集合と位相のまとめ}
位相空間とは,集合に開集合とはなんであるかを定めることにより元の遠近感を定めたもの.これにより連続関数を定義できる.
また性質の良いコンパクト空間やハウスドルフ空間や連結空間というものがあり,これらの性質により代数学の基本定理は証明できる.
\end{itembox}
\Section{大学3年生}
\Subsection{多様体}
多様体とは位相空間の中でも特に重要なもので,幾何学の主な研究対象です.
\defb[多様体]
$M:$が$n$次元の$(C^\infty 級)$多様体であるとは$M$がハウスドルフ空間であり,次のような開近傍$U_i$と同相写像$\phi_i : U_i \to \phi_i(U_i) \subset \R^n$が存在することである.\\
\[ \bigcup_i U_i = M \]
$U_i \cap U_j \neq \emptyset$のとき,次の座標変換がが$C^\infty$級である.
\[
\phi_i \circ \phi_j^{-1} | _{\phi_j(U_i\cap U_j)} :\phi_j(U_i\cap U_j) \to \phi_i(U_i\cap U_j)
\]
\defx
突然仰々しい定義が出てきましたが,多様体の$2$目の性質は局所ユークリッド的と言われるものです.
つまり,位相空間で連続写像については考えることが出来ましたが,微分を考えることはまだ出来ません.
そこで,位相空間のある一部を見てあげればそれはユークリッド空間であるとみなせるものを多様体としたのです.
つまり,多様体は好きなところに座標を入れることができ,かつ好きな用にいれた座標はちゃんとうまく合わさっているよというのがこの定義です.
幾何学の主な対称と言いましたが,例えば有名なポアンカレ予想は多様体に関する次のような定理です.
\thm[ポアンカレ予想]
\leavevmode\\
$M$多様体,$M$と$S^n$がホモトピー同値ならば$M$は$S^n$と同相である\\
$n=2$については$2$次元多様体の分類は$20$世紀はじめに知られており,
$n=3$コンパクトかつ単連結ならば$3$次元多様体は$S^3$と同相である.ということがポアンカレによって予想されました.\\
$n\ge 5$のときはスメールが解決,$n = 4$はフリードマンが解決,$n=3$のときはペレルマンが解決しました
\thmx
このように,数学の一つの目標にある概念を\underline{分類}するというものがあります.
分類することによって,今まで雑然と広がっていた世界が綺麗に掃除され見通しよくなるというのが数学の$1$つの仕事です.
では多様体での微分について定義しましょう.
\defb
多様体 $M_1, M_2$ を考える.写像 $F : M_1 \to M_2$ が $C^\infty$級であるとは、$F(x) \in M_2$ のまわりの座標近傍 $(V,\psi), F^{-1}(V)$ に含まれる$x\in M_1$ のまわりの座標近傍 $(U, \phi)$ に対して、
$\psi \circ F \circ \phi^{-1} : \phi(U) \to \psi(V )$が $C^\infty$ 級となることである。
\defx
ユークリッド空間内の多様体のみを考えて$f:M\to N$,$M\subset\R^k,N\subset\R^l$の微分を考える.\\
\[
df_x = (\frac{\partial f_i}{\partial x_j}):\R^k\to\R^l
\]
という行列で定める.\\
\defb
$x\in M$が正則点 $\iff df_x$が全射.\\
$x\in M$が特異点 $\iff df_x$が全射でない.\\
$y\in N$が正則値 $\iff f^{-1}(y)$が全て正則点.\\
$y\in N$が臨界値 $\iff f^{-1}(y)$が特異点を元として含む.\\
\defx

\proof[多様体論を用いた代数学の基本定理の証明]
\leavevmode\\
この証明はJohn Willard Milnor ``Topology from the Differentiable Viewpoint''に載っている証明です.
証明のポイント\\
$(1)\ p(z):\C\to\C$を今までどおりの多項式とする.\\
$(2)\ h:S^2\to\C$というステレオグラフィック射影によって,\\
$(3)\ h^{-1}\circ p \circ h : S^2 \to S^2$を定める.\\
$(4)\ f$が$C^\infty$級写像であることを示す.\\
$(5)\ f$は有限個の臨界点しか持たない.\\
$(6)\ f$の球から正則値の集合は有限を除いたものなので連結.\\
$(7)\ \# f^{-1}(y)$は開近傍上で定数であるので常に0でない.\\
\proofx
\begin{itembox}[l]{多様体のまとめ}
多様体とは位相空間の中でも好きなところに座標を入れてユークリッド空間のように扱えるものである.これにより
多様体の構造が入っている集合には連続写像だけでなく写像の微分を定義することが出来た.
多様体とは幾何学の基本的な研究対象である.また多項式をコンパクトな多様体$S^2(球面)$間の写像であるとしてその臨界値を見ることにより代数学の基本定理は証明できる.

\end{itembox}
\Subsection{群論}
代数学が扱う対称として基本的でシンプルなものが群です.群は集合に掛け算のみが入ったものを考えており,
例えば今まで習ってきた,整数や実数や群といったものは全て群です.群の定義は以下の様なものです.
\defb
集合$G$が群であるとは,$G$上の二項演算が$x,y,z\in G$に対して以下を満たすことである.\\
$\rm (i)$ $(xy)z=x(yz)$\\
$\rm (ii)$ $\exists e \in  G \ s.t. \ xe=ex=x$\\
$\rm (iii)$ $\forall x\in G \ \exists x^{-1} \in G \ s.t. \ x x^{-1} = x^{-1} x= e$\\
さらに,$xy=yx$を満たすとき$G$をアーベル群という.\\
\defe
また群の部分集合にも同じ演算の構造が入る場合に部分群といいます.つまり次のような定義です.
\defb
部分集合$H\subset G$が\\
$\forall x,y \in H: \ xy\in H$,\ \ $\forall x\in H:\ x^{-1} \in H$,\ \ $e\in H$\\
を満たすとき$H$を$G$の部分群という.
\defe
また部分群の中で扱いやすいものを定義します.
\defb
$N\subset G:$部分が正規部分群である\\
$\iff \ \forall g\in G : \ gNg^{-1} \subset N$
\defe
例えばアーベル群では全ての部分群は正規部分群です.
\defb
群$G$,$G'$があったとして,$f:G\to G' \  \forall x,y \in G \ f(xy)=f(x)f(y)$をみたす$f$を群準同型写像という.\\
特に$f$が準同型で全単射であるとき,同型写像であるという.
\defe
ここで群論で重要なシローの定理について触れておきましょう.
\defb
$G:$有限群,$p$:素数とする.$G$が$p$群$\iff\ |G|$が$p$のべき乗個.\\
$G$の部分群でかつ$p$群であるものを$p$部分群.\\
$|G|=p^e m, m,p$は互いに素なとき,\\
元の個数が$p^e$個の群をシロー$p$群という.
\defe

\thm[シローの定理]
$G:$有限群,$p$:素数とする.\\
$\rm (I)$シロー$p$部分群が存在する.\\
$\rm (II)$$(p\ $シロー部分群の個数$)\equiv\ 1 \bmod p$\\
$\rm (III)$$\forall P:シローp$群,$\forall Q:p$部分群,$\exists x\in G\  s.t. \ Q\subset xPx^{-1}$
\thmx
これは有限群を分類する際にとても便利な定理です.
\Subsection{ガロア理論}
ガロア論とは,体という四則演算が入った集合の拡大を考える理論で,体の拡大とそのガロア群を対応させることができるという理論です.
\defb
$K \subset L$に体の構造であるとき,$L$は$K$の拡大であるという.
\defx
\defb
$L$の同型で,かつ$K$上では恒等写像なもの全体を$\Aut_K(L)$とかく.
$\# \Aut_K(L)=[L:K]$のとき$L$は$K$のガロア拡大であるといい,$\Aut_K(L)$をガロア群という.
\defx
\ex
\[
[\C:\R]=2,\Aut_\R(\C) = \{\sigma_1,\sigma_2\}
\]
ただし,$\sigma_1$は恒等写像,$\sigma_2$は複素共役写像である.
\exx
\thm
$L/K$ を有限次ガロア拡大,$G$ をそのガロア群とする.
$(1)L/K$ の中間体 $M$ と $G$ の部分群 $H$の間に次の一対一対応がある:
\[
M \rightarrow G(L/M) , F(H) \leftarrow H
\]
$(2) M$ と $H $が対応するとき,$L/M$ はガロア拡大で $H$はそのガロア群である.
そして
\[
[L : M] = |H|, [M : K] = |G : H|
\]
$(3) M$ と $H$ が対応するとき,$M/K$がガロア拡大であることと$H$ が$G$ の正
規部分群であることは同値である.さらにこのとき
\[
G(M/K) \cong G/H. 
\]
\thmx
このガロア理論を用いても代数学の基本定理を証明する事ができます.
\proof[Galois理論による証明のポイント]
\leavevmode\\
雪江明彦「代数学2 環と体とガロア理論」に同様の証明が載っています.\\
$(1)\ K/\C$を有限次拡大とする.(背理法)\\
$(2)\ K/\R$は正規拡大であるとして一般性を失わない.\\
$(3)\ K/\R$はGalois拡大になる.\\
$(4)\ G:=Gal(K/R)$とする,$H\subset G$を2-Sylow群とすると,これに対応する拡大は奇数次拡大で,$\R$そのものしかない.\\
$(5)\ [K:\C]>1$で$2^n$であるとすると,$Gal(K/\C)$が2-群になって,$\C$が二次拡大をもつことになる.\\
\proofx
\begin{itembox}[l]{代数学のまとめ}
群論とは集合に積の構造のみを入れて考えるシンプルな学問である.
ガロア理論は体という四則演算のできる集合の拡大とこの群論を結びつける理論である.
ガロア理論により代数学の基本定理は証明できる.
\end{itembox}
\Subsection{代数トポロジー}
多様体論では微分同相という同値関係で位相空間を分類しましたが,代数トポロジーはホモトピー同値という関係でもって位相空間を分類します.例えば,コップの表面とドーナツの表面はホモトピー同値というように,我々の考える``同じ''よりも更にゆるい考えかたでものを分類します.一方で,球面とドーナツの表面はホモトピー同値ではありません.これは``穴の数''に起因しています.この穴という概念を高次元化してみるのがホモロジー群です.
\defb[ホモトピー]
$f,g:X\to Y$という連続写像ががホモトピックであるとは,\\
連続写像 $F : X \times [0, 1] \to Y$であって,すべての $x \in X$ について$ F(x, 0) = f(x), F(x, 1) = g(x) $をみたすものが存在することをいう.つまり,$F_t :X\to Y$という$t$によってパラメータ付された関数が存在して,
$t=0$では$f$と一致し,$t=1$では$g$と一致するようにできるという意味である.これを$f\simeq g$と表す.
\defx
難しいような定義が出てきましたが,$f,g$がホモトピックであるとは,$f$をじわじわと動かしていくと$g$になるというイメージです.
\defb[ホモトピー同値]
$X,Y$という位相空間がホモトピー同値であるとは,\\
$f:X\to Y$と$g:Y\to X$という連続写像が存在して,$g \circ f \simeq 1_X$,$f \circ g \simeq 1_Y$とできることである.
\defx
ここで,空間が等しいということを各学問がどのように見ているかという事を見てみましょう.
\defb
$X,Y$という集合が,集合として同等であるとは,\\
$f:X\to Y$と$g:Y\to X$という写像が存在して,$g \circ f = 1_X$,$f \circ g = 1_Y$とできることである.
\defx
\defb
$X,Y \subset \C$が双正則であるとは,\\
$f:X\to Y$と$g:Y\to X$という正則写像が存在して,$g \circ f = 1_X$,$f \circ g = 1_Y$とできることである.
\defx
\defb
$X,Y$という群が群として同型であるとは,\\
$f:X\to Y$と$g:Y\to X$という群準同型が存在して,$g \circ f = 1_X$,$f \circ g = 1_Y$とできることである.
\defx
\defb
$X,Y$という位相空間が,同相であるとは,\\
$f:X\to Y$と$g:Y\to X$という連続写像が存在して,$g \circ f = 1_X$,$f \circ g = 1_Y$とできることである.
\defx
\defb
$X,Y$という多様体が微分同相であるとは,\\
$f:X\to Y$と$g:Y\to X$という$C^\infty$級写像が存在して,$g \circ f = 1_X$,$f \circ g = 1_Y$とできることである.
\defx
このようにして,数学という学問はある集合がまず等しいという概念を定義します.そうして,等しい物を集めて,違うものは違うものに分類します. その分類に必要になるのが\underline{不変量}というものです.不変量とは,
\[
X と Y が等しい \Rightarrow XとYの不変量が等しい
\]
が成り立つものです.このとき対偶として
\[
 XとYの不変量が等しくない \Rightarrow X と Y が等しくない
\]
が成り立ちます.こうやって,不変量をみることによって,ある$X$と$Y$が等しくないかどうかということが判別できます.
話が大きくそれましたが,代数トポロジーにおける不変量としてホモロジー群というものがあります.
\defb
空間$X$を$q$次ホモロジー群$H_q(X)$に,連続写像$f:X\to Y$をホモロジー群間の準同型$f_* : H_q(X) \to H_q(Y)$に対応させる.
このとき,次のような性質を満たすようにできる.\\
$(1):$空間$X$での恒等写像は,ホモロジー群$H_q(X)$での恒等写像に対応する.\\
$(2):$$f:X\to Y,g:Y\to Z$という連続写像があるとき,$g\circ f$は$(g\circ f)_* = g_* \circ f_* :H_q(X) \to H_q(Z)$が成り立つ.
つまり位相空間での関係を保ちます.かつホモロジー群は不変量となっている.つまり,
$(3):$$X,Y$がホモトピー同値ならば,$H_q(X) \cong H_q(Y)$という同型成り立つ.
\defx
このホモロジー群が実際にあることを証明するのは少々手間がかかりますが,このような性質を満たすようなものが存在するとして
色々な結論を導くことが出来ます.例えば,$S^n$のホモロジー群と言うのは,幾つかの公理を加えるだけで簡単に証明できます.
\thm
$S^n=\{x\in\R^{n+1} | |x|=1\}$とすると,$\Ho_k S^n = \Z\  (k=0,n)$,$\Ho_k S^n = 0\  (k\neq 0,n)$
\thmx
\thm
$f:S^n \to S^n$を写像とすると,これは$f_*:\Ho_n(S^n)\to \Ho_n(S^n)$を引き起こす.\\
これは$\Z$間の準同型であるので,$f_*(x) = kx$とおけて,この$k$を$\deg(f)$と置いて,写像度という.
\thmx
\prop
写像度は次のような性質を満たす.\\
$\rm (i)$$\deg(id)=1$\\
$\rm (ii)$$\deg(f\circ g) = \deg(f)*\deg(g)$\\
$\rm (iii)$$f \simeq g \Rightarrow \deg(f) = \deg(g)$\\
\propx
ところで,この写像度というのは,回転数とも呼ばれ,$f$の定義域が単位円を一周するとき,$f$の像は単位円を何周するかということを表しています.
\proof[代数トポロジー的な代数学の基本定理の証明]
\leavevmode\\
この証明は,Albrecht Dold「Lectures on Algebraic Topology」に載っている証明です.
証明のポイント\\
$(1) \ \hat{p}:S^1\to S^1 ; z \to \frac{p(z)}{|p(z)|}$により定める.\\
$(2)\ |z|\le 1$ で 零点を持たないならば$\deg \hat{p} = 0$をしめす\\
$(3)\ p_t(z) =  \frac{p(tz)}{|p(tz)|}$による.\\
$(4)\ |z|\ge 1$ で 零点を持たないならば$\deg \hat{p} = n$.\\
$(5)\ p_t(z) =  \frac{t^kp(z/t)}{|t^kp(z/t)|}$による\\
\proofx
ここまでできるだけ丁寧に説明することを心がけてきましたが,紙面の都合上これ以降は概略をのべるのみにします.
\Subsection{微分幾何学}
\defb
$\{g_p\}_{p\in M}$がRiemann計量である.\\
$\iff \ (1)g_p : T_p M \times T_p M \to \R $は内積.\\
$(2)s_1,s_2:U\to TM:$切断とすると,$g_p(s_1(p),s_2(p))$は$C^\infty$級・
\defe
\thm
$M$:境界つきコンパクトRiemann多様体. $K$を$M$のガウス曲率. $k_g:\partial M$の測地曲率.
\[
\int_M K\;dA+\int_{\partial M}k_g\;ds=2\pi\chi(M)
\]
\thmx
\proof[微分幾何学による代数学の基本定理の証明]
\leavevmode\\
この証明は \url{http://arxiv.org/pdf/1106.0924.pdf} に載っている証明です.
証明のポイント\\
$p(z)$が零点を持たないと仮定する.\\
$p^{*}(z) = z_np(1/z) = a_0z_n + a_1z{n-1} + \cdots + a_n$として, $f(z)=p(z)p^*(z)$とする.\\
$f(\tfrac{1}{w}) = p(\tfrac{1}{w})p^*(\tfrac{1}{w}) = w^{-2n}p^*(w)p(w) = w^{-2n}f(w)$\\
$w\in \C :\ g=\frac{1}{|f(w)|^{\frac{2}{n}}}\,|dw|^2$\\
$w\in \hat{\C} \setminus \{0\}:\  g=\frac{1}{|f(1/w)|^{\frac{2}{n}}}\,|d(1/w)|^2$ とする\\
$\frac{1}{|f(w)|^{\frac{1}{n}}}\,K_g=\frac{1}{n}\Delta \log|f(w)|=\frac{1}{n}\Delta \text{Re}(\log f(w))=0$
$\int_{\mathbf{S}^2}K_g=4\pi$により矛盾.
\proofx
\Subsection{確率論}
\proof[確率論による代数学の基本定理の証明]
この証明はL. C. G. Rogers, David Williams「Diffusions, Markov Processes, and Martingales: Volume 1, Foundations」に載っている証明です.
証明のポイントは.
$(1)\ (B_t:t \ge 0)$をブラウン運動とする.\\
$(2)\ f(z) = 1/p(z)$とおくと,これは正則で,$z\to\infty$で$0$に収束する.\\
$(3)\ \alpha < \beta$として$\{\Real f \le \alpha\}$と$\{\Real f \ge \beta\}$は開集合を含む.\\
$(4)\ f(B_t)$はマルチンゲールの収束から,$f(B_t)\to f(B_\infty)$に$L^1$収束する\\
$(5)\ $一方でこれはブラウン運動の再帰性に矛盾する.\\
\proofx
\Section{参考文献}
これまで様々な分野を紹介してきましたが,それぞれの分野についてもし興味が涌いたならば是非書店や図書館に行って,本物の数学に触れてみてください.各証明に乗せている本はどれも其の分野の面白い本なので参考になると思います.


%\Chapter{編集後記}
%\thispagestyle{empty}
\vspace*{10zw}
\vfill

\parindent=0pt
\begin{picture}(110,1)
\setlength{\unitlength}{1truemm}
\put(5,2){\Large\textbf{$e^{\pi i}sode$ Vol.4 !!!volを変える!!! }} 
\thicklines
\put(0,1){\line(2,0){110}}
\thinlines
\put(0,0){\line(2,0){110}}
\end{picture}

\small{2016年11月25日発行!!!変更する!!!}\\
 \normalsize{著 者・・・・・東京大学理学部数学科有志}\\
 \normalsize{発行人・・・・・!!!名前を書く!!!}\\
\begin{picture}(100,1)
\setlength{\unitlength}{1truemm}
\thinlines
\put(0,1){\line(2,0){110}}
\thicklines
\put(0,0){\line(2,0){110}}
\put(0,-5){\small{\copyright  Students at Department of Mathematics,The University of Tokyo 2016 Printed in Japan}}
\end{picture}


\backmatter
\end{document}
