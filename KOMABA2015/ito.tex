%まず最初に使ったプリアンブルをここに書いてください.
%ただしコンパイルの都合上コメントアウトしてください.
%実際に確認する際は,各自の環境でmain.texにこのプリアンブルを追加してください.

%\usepackage{mathrsfs}
%\usepackage[all]{xy}
%\newcommand{\proofend}{\begin{flushright} $\blacksquare$ \end{flushright}}
%\renewcommand{\labelenumi}{(\roman{enumi})}
%\newcommand{\nkgr}{・}
%\theoremstyle{definition}
%\newtheorem{theorem}{定理}
%\renewcommand{\thetheorem}{}
%\newtheorem{defi}{定義}
%\newtheorem{thm}[defi]{定理}
%\newtheorem{lem}[defi]{補題}
%\newtheorem{cor}[defi]{系}
%\newtheorem{prop}[defi]{命題}
%\newtheorem{ex}[defi]{例}


\Chapter{代数学の基本定理でみる数学の世界(伊藤)}
% タイトル(名前)でお願いします.
% セクションは \Section \Subsection \Subsubsection で分けてください.
% 詳しくはMAY2015を参考にしてください.
\Section{はじめに}
数学科展示ますらぼにお越しいただきましてありがとうございます.
数学科とはどのようなことをやっている学科なのか一般の人に説明するのはなかなか難しく,
一般の人に端的な説明を求められるとなかなか四苦八苦してしまうところがあります.
この記事では数学科がどのようなことを勉強しているのかについて,
\textbf{代数学の基本定理}という定理を題材に出来るだけわかりやすく説明したいと思います.
この記事は第7回関西すうがく徒のつどいにおける拙講演「代数学の基本定理でみる数学の世界」を
更に詳しくして紙面化したものですので,講演に関してまとめたウェブ上の記事 http://togetter.com/li/878845 も参考にしていただければと思います.
\Section{代数学の基本定理とは}
代数学の基本定理とは
\thm
次数が$1$以上の複素係数一変数方程式には複素根が存在する
\thmx
という定理です.具体的にはどういうことを言っているのでしょうか.例を見てみましょう.
\ex
$2x-4=0$というのは$1$次方程式ですが$x=2$という解を持ちます.
\exx
\ex
$ax^2+bx+c=0$というのは$2$次方程式ですが,この方程式の解の公式も中学校で習ったことでしょう.
\exx
\ex
$3$次多項式と$4$次方程式にも,極めて難解ですが解の公式というものが知られています.
これについては"カルダーノの公式"や"フェラーリの公式"で調べてください.
\exx
これらの解の公式とは\underline{具体的にバッチリと解のありかを求める}公式です.
一方で代数学の基本定理とは複素根が存在すると言っているだけなので,\underline{どこにあるかは分からないけどとりあえず存在はするよ}という定理なんです.
しかし,これはどんな方程式にも解があるということを言っているのでそれは強い主張であるともいえます.
この定理は$1600$年ごろに様々な数学者によって予想され,$1800$年ごろにガウスによって証明がされました.
代数学の基本定理は高校生でも証明できるような定理なのですが,その基本的さ故に様々な証明があり,
大学$3,4$年生で習うようなことを使っても証明することができます.
この代数学の基本定理と共に大学の数学とはどのようなものなのかを見てみましょう.
\Section{大学1年生}
大学$1$年生で習う数学とは"解析学入門"と"線形代数学"の$2$つです.
どちらも数学の基礎であるとともに理系の多くの学科でも使われるものです.
\Subsection{解析学入門}
解析学入門は東大では"数学$1$"という科目名で開講されています.
微分と積分について現代数学的に学び直そうという科目です.
意識高く大学で勉強をしようと思っていた東大の$1$年生たちの多くがこの科目に打ちのめされて俗にいう五月病に羅患します.
この解析学のつまづきやすい$2$つのポイントとして,\large{$\epsilon-\delta$論法}と\large{コンパクト}というものがあります.
この$2$つについて見てみましょう.
\Subsubsection{$\epsilon - \delta$ 論法}
高校数学にも極限という概念はあって,$x$が$0$に限りなく近づくとか$n$が$\infty$に発散するとかいう言葉が使われています.これを厳密に定義しようというのが$\epsilon - \delta$ 論法です.
本格的な$\epsilon - \delta$ 論法に入る前に幾つか練習をしてみましょう.
\ex
ある$x$という実数の絶対値は全ての正の数$p>0$より小さいとします.これを数式で書くと以下のようになります.
\[
\forall p > 0 : \ |x| < p
\]
$\forall p$で全ての$p$についてということを言っているわけです.
ではこの$x$はどんな数なのでしょうか.$p=0.1$としてみても,$|x|$はこれより小さいです.$p=0.0001$としても$|x|$はこれより小さいです.
$p=0.000\cdots (0が2億個) \cdots 01$よりも$|x|$は小さいです.これはつまり$|x| = 0$ということです
$x$は$0$としたわけではないが,$0$になってしまった.これが現代数学の"限りなく近い"という概念をつかむのに大事な考え方です.
\exx
\ex
ある$x$という数は全ての正の数$p>0$よりも大きいとします.つまり,
\[
\forall p > 0 : \ x > p
\]
です.この$x$も具体的にはどんな数なのでしょうか.$p=1000$としてみても,$x$はこれより大きいです.$p=2億$としてみてもこれより大きいです.
これもやはり$x$は$\infty$であるということを示しているのではないでしょうか.$\infty$というのはきちんと定義されていませんが,
$x$は限りなく大きいとはこのような気分なんだなあということがイメージできます.
\exx
ではここで,$\epsilon-N$論法というのを見てみましょう.
\defb[$\epsilon - N$ 論法]
$\lim_{n\to\infty} a_n = \alpha $\\
$\iff$
$\forall \varepsilon > 0,\  \exists N \in \N \quad \textrm{s.t.}\  \forall n \in \N,\  n > N \Rightarrow |a_n - \alpha | < \varepsilon $
\defe
突然数式がたくさん出てきて混乱したかもしれませんが落ち着いて見れば簡単です.\\
$\lim_{n\to\infty} a_n = \alpha $というのを定義しているわけです.\\
$a_n$が$\alpha$に限りなく近づくとはどういうことをなのでしょうか.\\
それは,例で見たとおり,$|a_n - \alpha|$が限りなく小さくなればいいわけです.
それを表すために,$\varepsilon > 0$というとても小さな数を$1$つ取ってきます.
それに対して,ある$N$を取ってきて$N$以降では$|a_n - \alpha | < \varepsilon$が成り立っているよとするわけです.\\
例えば,$1000$項目以降では,$|a_n - \alpha| < 0.001$が,$100000$項以降では,$|a_n - \alpha | < 0.0000000001$が成り立っていたら
どんどん近づいて行くような気がしますよね.これが限りなく近づくよ,ということを言うためにまず$\forall \epsilon > 0$としているわけです.\\
\ex
$a_n = \frac{n+1}{n}$とすると$n \to \infty $でこれは$1$に収束します.\\
実際,$|a_n - \alpha| = | \frac{n+1}{n} - 1 | = | \frac{1}{n} |$です.\\
例えばこの$|a_n - \alpha|$を$0.01$より小さくしたい!と思えば,$N=100$としてあげれば,
$N$項目以降では$|\frac{1}{n}| < 0.01$が成り立つわけです.\\
ここでも,$|a_n - \alpha| $という差は$n$が大きくなるに連れてどんどん小さくなっていますね.
\exx
このような方法を採用するメリットとして,極限という概念がきっちりと定義されて,例えば,次のような明らかに成り立って欲しい極限の性質も厳密に証明する事ができます.
\prob
$\lim_{n\to\infty} a_n = \alpha , \lim_{n\to\infty} b_n = \beta $とする.\\
$\lim_{n\to\infty} (a_n + b_n) = \alpha + \beta , \lim_{n\to\infty} a_n b_n = \alpha\beta $
を示せ.
\probx
同様にして$\epsilon-\delta$論法も見てみましょう.
\defb[$\epsilon - \delta$ 論法による定義]
$\lim_{x \to a}f(x) = b$\\
$\iff$
$\forall \varepsilon > 0,\  \exists \delta > 0\quad \textrm{s.t.}\  \forall x \in \mathbb{R},\  0 < |x-a| < \delta \to |f(x)-b| < \varepsilon$
\defe
これも同様の考え方です.$x\to a\  (xがどんどんaに近づく)$のとき,$f(x) \to b\  (f(x)はどんどんbに近づく)$ということを定義しているわけです.また$|f(x)-b|$を限りなく小さくするために,$|x-a|$の幅を限りなく小さくとっているわけです.\\
またこの$\forall \exists$という並びは,どんな$\varepsilon (とても小さいイメージ)$に対してでも,いちいち$\delta (さらに小さいイメージ)$をとってくるということを表しています.
\prob
$\lim_{x\to 0} x^2 = 0$を証明せよ.$\varepsilon >0$として,$\delta = \sqrt{\varepsilon}$とすれば,$ 0 < | x | < \delta$ならば$ |x^2| < \varepsilon = \delta^2$を示せばよい. 
\probx
\Subsubsection{コンパクト}
次に第二のつまづきポイントであるコンパクトについて触れましょう.
高校数学で次のような定理があったことを思い出しましょう.
\thm[最大値最小値の定理]
$[a,b]$を有界閉区間,$f$を$[a,b]$上の実数値連続関数とする.
このとき$f$は最大値および最小値にそれぞれ少なくとも一点で到達する.
\thmx
これは高校数学では大した有り難みもない定理でしたが現代数学では重要です.
ここで重要なのは$[a,b]$が有界閉区間であるという仮定と,$f$は連続であるという仮定です.実際
\ex[非有界]
$\R$上で連続な関数$f(x)=x$は$\R$で最大値,最小値を持たない.
\exx
\ex[不連続]
$[-1,1]$上の関数.$f(x)=1/x$は最大値,最小値を持たない.
\exx
という例が示すように,有界閉区間または連続という仮定を外すとたちまちこの定理は成り立たなくなります.
この有界閉区間という概念を一般化したのがコンパクトです.
\defb[コンパクト]
$X$空間が$(点列)$コンパクトである\\
$\iff$
$X$内の任意の点列が$X$内に収束する部分列を含む
\defe
これも例を見てみましょう.
\ex
開区間$(0,1)$はコンパクトではない.なぜならば,$\{1/n\}$という数列は$0$に収束するが,この数列の部分列は$(0,1)$内の点に収束しない.
\exx
\ex
実数$\R$はコンパクトではない.なぜならば,$\{n\}$という数列は$\infty$に発散するが,この数列の部分列は$\R$内の点に収束しない.
\exx
\thm
$I \subset \R^n$がコンパクトであることと有界かつ閉であることは同値
\thmx
という風にコンパクトは有界閉区間の拡張になっているわけです.そして,
\thm
$I \subset \R^n$をコンパクト,$f$を$I$上の実数値連続関数とする.
このとき$f$は最大値および最小値にそれぞれ少なくとも一点で到達する.
\thmx
という定理が成り立ちます.こうして,$2$のポイントをおさらいしたところでその応用として代数学の基本定理を証明してみましょう.
\thm[代数学の基本定理]
次数が$1$以上の任意の複素係数一変数多項式$p(z)=a_0+a_1 z+\cdots + a_nz^n$には複素根が存在する.
\thmx
\proof[初等解析による証明]
これは杉浦光夫「解析入門1」に載っている証明です.
証明のポイントは3つ.\\
$(1)\ \lim_{|z|\to\infty}|p(z)| = \infty$\\
$(2)\ |p(z)|$はコンパクト集合上で最小値を取る.\\
$(3)\ |p(a)|>0 \Rightarrow \exists b \in \C \ s.t. \ |p(b)| < |p(a)|$(下には下がいる)
です.
\[
\lim_{|z|\to\infty}|p(z)| = \infty
\]
という意味をもう一度解釈してみましょう.
\[
\forall M \in \R \ \exists R>0 \ s.t. \  |z| > R \ \Rightarrow\  |p(z)|>M
\]
ということでした.そして$M$は任意ですから,$M=|p(0)|$として,それに対して$R>0$を一つ取り,
$K=\{ z\in\C |\ |z|\le R\}$とおけば,$K$の外では$|p(z)|>M$が成り立ちます.
つまりこの$K$の中で最小値を探せばいいいわけです.ところで$K$はコンパクトであるので
\begin{center}
$|p(z)|$は$K$上で最小値を取る
\end{center}
が言えます.最後に,
\[
|p(a)|>0 \Rightarrow \exists b \in \C \ s.t. \ |p(b)| < |p(a)|
\]
が言えて$(この証明は杉浦に譲ります)$証明終了.
\proofx
代数学の基本定理の証明は$1$年生の解析の大事な部分を使って得られるのでした.
\Section{大学2年生}
\Subsection{解析学続論}
大学$1$生では他に線形代数という科目を勉強しますが,この記事には関係ないので割愛します.
大学$2$年生では$1$年生で習った解析学と線形代数学の発展について学びます.
解析学では多変数の解析について学びます.ここでは線積分というものについて触れましょう.
今まで積分といえば,$\int_a^b$と言ったように区間$[a\ b]\subset \R$上での積分を考えてきましたが,
例えば,円周$\{(x,y)\in\R^2 | x^2+y^2=1\}$にそってある関数を積分したいということは数学だけでなく
多くの理系分野でよくあることです.まず曲線とは何かについて考えてみましょう.
\defb
$I\subset\R$を区間とします.$\phi:I \to \R^n$が空間曲線であるとは,一対一の連続写像であるこということである.
\defx
一対一というのは,$a\neq b \Rightarrow \phi(a) \neq \phi(b)$であるということで,つまりは自己交差をしないということです.
確かに自己交差をしなくてちゃんと繋がっていなくては曲線とはいえませんね.
\ex
$\phi:[0,1] \to \R^2$を$\phi(t)=(t,t)$で定める.これは$(0,0)$と$(1,1)$を結ぶ直線であり,空間曲線である.
\exx
\ex
$\phi:[0,2\pi) \to \R^2$を$\phi(t)=(\cos t ,\sin t)$で定める.これは単位円周です.
\exx
それではこれらの曲線にそった積分というのを次で定めます.
\defb
$f:\R^n \to \R$を関数,$\phi:I\to\R^n$を滑らかな曲線として,この曲線の像を$C$で表す.曲線$C$に沿った$f$の線積分を以下で定義する.,
\[
\int_C f(x)ds := \lim_{d(\Delta)\to 0} \sum_{i=1}^N f(\phi(\xi_i)) |\phi(t_i) - \phi(t_{i-1})|
\]
ただし,ここでの$\Delta$とは区間$I$の分割$t_0 < t_1 < \cdots < t_{N-1} < t_N$を考えており$d(\Delta)$はその分割の最も大きい幅です.
\defe
これは$Riemann$積分の考え方を使った積分の定義であり,詳しくは$e\pi sode\ vol.3$の"積分の歩み"を参照していただきたいのですが,
基本的には高校でならった区分求積法の考え方と同じで,区分求積法はある区間を同じ幅で分割していましたが,それを好きな幅で分割して良いようにしたという話です.またこの積分は収束して以下のようにも表されます.
\prop
$f,\phi,C$を上の定義と同様とし,$I=[a,b]$となるときに次が成り立つ.
\[
\int_C f(x) ds = \int_a^b f(\phi(x))|\phi'(t)|dt
\]
\propx
\ex
$f:\R^2\to\R$を$f(x)=1$という定数関数にして,$\phi:[0,1] \to \R^2$を$\phi(t)=(t,t)$で定める.\\
このとき
\[
\int_C f(x) ds = \int_0^1 |(t,t)| dt = \int_0^1 \sqrt{2} t dt  = \sqrt{2}
\]
です.この積分は$(0,0)$と$(1,1)$を結ぶ直線の長さ$\sqrt{2}$を求めています.
\exx
\Subsection{複素解析}
代数学の基本定理の証明方法に複素解析的な方法を使ったものが有名です.
複素解析は今まで実数関数でやってきたことを複素数の範囲に拡張することによって色々な美しい結果が得られる学問です.
複素解析の主な研究対象には正則関数というものがあります.まずそれを定義しましょう.
\defb[正則関数]
$f:\C \to \C$が正則であるとは各点で微分係数を持つということである.つまり,
\[
f'(z) = \lim_{h\to 0} \frac{f(z+h) - f(z)}{h}
\]
が各$z\in\C$で収束するということである.
\defx
\rem
これだけでは普通の実数の微分可能関数と変わらないではないかと思うかもしれませんが次のようことが成り立つことに注意しなければなりません.
つまり,$h\to 0$としていますがこの$h$は複素数なので色々な$0$への近づき方をするということです.
$h=x+yi$とおいて実部と虚部に分けたとき,$y=0,x\to 0$として$0$に近づけたときこれは偏微分$\frac{\partial f}{\partial x}$になります.
一方で$x=0,y\to 0$として$0$に近づけたとき,
\[
f'(z) = \lim_{y\to 0} \frac{f(z+yi)-f(z)}{yi} = -i\frac{\partial f}{\partial y}
\]
であり,
\[
f'(z) = \frac{\partial f}{\partial x} = -i\frac{\partial f}{\partial y}
\]
が成り立つ必要があります.この関係をコーシー・リーマンの関係式といいます.
\remx
次に$\C$の部分集合として重要な単連結領域というのを定義しますが,これは"便利な領域"として考えて頂いて構いません.
\defb[単連結領域]
$D \subset \C$が単連結領域であるとは,連結な開集合であって$D$内の任意の閉曲線は$1$点にホモトピックであるようなものである.
\defx
$1$点とホモトピックであるとはこの記事の後半を参照していただきたいのですが,単連結領域とは穴がない領域をイメージしてください.
そうすると以下のように重要な定理が成り立ちます.
\thm[コーシーの積分定理と積分公式]
$D$を単連結領域とし、$f(z)$ は $D$ 上で正則である複素関数とするとき、$C$ を $D$ 内にある長さを持つ単純閉曲線とする.
\[
 \oint_C f(z) \, dz\ = 0
\]
$a$をまた$C$によって囲まれる領域に属する点とする.
\[
 f(a) = \frac{1}{2 \pi i}\int_C \frac{f(z)}{z-a}dz
\]
\[
 f^{(n)}(a) = \frac{n!}{2 \pi i}\int_C \frac{f(z)}{(z-a)^{n+1}}dz
\]
\thmx
この定理の意味とは$f(z)$が正則であれば,どんな閉曲線上で積分してもその値は$0$になるということと,
逆に$a$という一点でだけ正則でないような$\frac{f(z)}{z-a}$という関数を積分するときは$f(a)$の値のみを考えればいいよという意味です.
この$e\pi isode$にある荒田さんの記事も参考になります.実際に例を見てみましょう.
\ex[コーシーの積分公式の例]
$f(z) = 1 , C = \{z\in\C | |z-a|=r\} $ のときコーシーの積分公式.\\
\[
1^{(n)}(a) = \frac{n!}{2 \pi i}\int_{|z-a|=r} \frac{1}{(z-a)^{n+1}} dz 
\]
となりますが
\[
 \int_{|z-a|=r} \frac{1}{(z-a)^{n+1}} dz =
  \begin{cases}
   \  2\pi i  \ \ (n=0) \\
   \  0 \ \ (n \ge 1) \\
  \end{cases}
\]
に他ならなりません.
\exx
正則関数がどれだけ関数に強い条件を課しているかというのは次の定理でわかります.
\thm[リュービルの定理]
複素平面全体で正則かつ有界な関数は定数関数のみ.
\thmx
\proof
\leavevmode\\
この証明は,藤本坦孝「複素解析」に載っている証明です.
証明のポイントは以下の$3$つです.\\
$(1)\ f有界つまり\forall z \in \C :\ |f(z)|\le M $かつ$f$正則を仮定する.\\
$(2)\  $仮定を満たす関数は正則より$f(z)=\sum_{n=0}^\infty c_n z^n $とべき級数展開可能\\
$(3)\ c_n$は$\forall R>0 : \ |c_n|\le \frac{M}{R^n}$をみたす.(ここでコーシーの積分公式が使われている)
\proofx
このリュービルの定理を用いて代数学の基本定理を証明する事ができます.
\proof[リュービルの定理を用いた代数学の基本定理の証明]
\leavevmode\\
この証明はLars Valerian Ahlfors「Complex Analysis」に載っている証明です.
証明のポイント:\\
$(1)\ p(z)=a_n z^n + \cdots + a_1 z+ a_0$が零点を持たないと仮定する(背理法)\\
$(2)\ g(z) = \frac{1}{p(z)}$は$\C$上で正則となる\\
$(3)\  \lim_{|z|\to\infty} |g(z)| = 0$となる.\\
$(4)\ $上から$g:$有界であることが言え,Liouvileより定数となり矛盾.
\proofx
複素解析の一つの目標として留数計算というものがあります.コーシーの積分公式では分数型の$1$点のみで正則でない関数の積分を考えましたが,
今度は他の形の正則でない点が複数ある場合でも積分計算をしてみようというというものです.
\defb[留数]
$f$が環状領域$\Delta(a,r,R) = \{z\in\C | \  r<|z-a|<R\}$で正則とする.このとき\\
\[
f(z) = \sum_{n=-\infty}^{\infty} a_n (z-a)^n
\]
という風に展開できて,これを$f$のローラン展開という.\\
特に$\Delta(a,0,R)$で正則$(a$のみ孤立して正則でない$)$とき,\\
$a_{-1}$のことを$f$の$a$での留数といい$\Res_{a} f$とかく.\\
\defx
\ex
$\frac{1}{z-c}$という関数を$|z|>|c|$でローラン展開すると.
\[
\frac{1}{z-c} = \frac{1}{z} + \frac{c}{z^2} + \frac{c^2}{z^3} + \cdots
\]
\exx
\thm[留数定理]
$D:$区分的$C^1$境界を持つ領域.$f:\overline{D}\setminus\{p_1,\cdots,p_n\}$で正則とする.
\[
\frac{1}{2 \pi i}\int_{\partial D} f(z)dz = \sum_{i=1}^n Res_{p_i} f
\]
\thmx
この留数定理とは,$f$という関数を積分する際は,$p_i$という点での留数のみを考えればいいよと言っているわけです.
留数を計算するのに便利な次の公式を紹介します.
\prop
$(1)\ z=a$に於いて$\lim_{z\to a} (z-a)f(z)$が有限確定値を持つとき,
\[
\Res_{a} f = \lim_{z\to a} (z-a)f(z)
\]
$(2)\ z=a$に於いて$\lim_{z\to a} (z-a)^m f(z)$が有限確定値を持つとき,
\[
\Res_{a} f = \frac{1}{(m-1)!} \lim_{z\to a} \frac{d^{m-1}}{dz^{m-1}} ((z-a)^mf(z))
\]
$(3)\ g,h$を正則関数として,$g(a)\neq 0,h(a)=0,h'(a)\neq 0$ならば
\[
\Res_{a} \frac{g}{h} = \frac{g(a)}{h'(a)}
\]
\propx
$\lim_{z\to a} (z-a)^m f(z)$が有限確定値を持つとき,$a$は$m$位の極であるといいますが,
$m$位の極の留数を計算するときは$(z-a)^m f(z)$という正則関数のテイラー展開を考えてあげればいいという話です.
正則関数の零点に関して次のような定理が成り立っています.
\thm[偏角の原理]
$D:$今までと同様.$f:$正則とする.
\[
\frac{1}{2 \pi i} \int_{\partial D} \frac{f'(z)}{f(z)}dz = (f\mbox{の}D\mbox{内の重複度込みの零点の個数})
\]
\thmx
\thm[ルーシェの定理]
$D:$区分的に$C^1$な境界を持つ有界領域\\
$f,g:D$とその境界上で定義された正則関数.\\
$\forall z \in \partial D :\ |f(z)-g(z)|<|f(z)|+|g(z)|$が成り立つとする.\\
このとき,$f$と$g$の零点の個数は等しい.
\thmx
ルーシェの定理は$f$と$g$の零点の個数を見たいときにその境界上のみで$f,g$の様子を考えて上げればいいという定理です.
\proof
\leavevmode\\
定理・証明ともに平地健吾先生に教えて頂きました.
証明のポイント\\
$(1)\ F_t(z) = (1-t) f(z) + t g(z)$は$0$にならない\\
$(2)\ N_t=\int_{\partial D} \frac{F_t'(z)}{F_t(z)}dz$は偏角の原理より$F_t$の零点の個数だがこれは$t$について連続.\\
$(3)\ (f$の零点の個数$)=N_0=N_1=(g$の零点の個数$)$
\proofx
%誰か良い例ください
この定理を用いて代数学の基本定理を証明する事ができます.
\proof[ルーシェの定理を用いた代数学の基本定理の証明]
\leavevmode\\

$f(z)=a_n z^n + \cdots + a_1 z+ a_0$と$g(z)=a_n z^n$とおく.\\
$|f(z)-g(z)|$は$n-1$次式,$|f(z)|+|g(z)|$は$n$次式より,\\
十分大きな円周上では$|f(z)-g(z)|<|f(z)|+|g(z)|$が成り立つ.\\
よって$f$の零点の個数は$n$個
\proofx
\prob
実はルーシェの定理まで行かなくても偏角の原理のみで代数学の基本定理を証明する事ができます.各自考えて見てください.
\probx
\Subsection{集合と位相}
ここでは位相空間論というものについて触れましょう.今までは$\R^n$のみで連続や収束という概念を考えて来ましたが,これを任意の集合に対して
扱えるようにするのが位相空間論の一つの目標です.

\defb[位相空間]
$X$を集合とする.$X$の部分集合からなる集合$\mathcal{O}$が$X$の開集合系である.\\
$\iff (1) (U_i)_{i\in I}$が $\mathcal{O}$の族ならば,$\cup_{i\in I} U_i \in \mathcal{O}$\\
$(2)(U_i)_{i\in I}$が $\mathcal{O}$の有限族ならば,$\cap_{i\in I} U_i \in \mathcal{O}$\\
また$\mathcal{O}$に属する元を$X$の開集合といい,$(X,\mathcal{O})$を位相空間という.
また閉集合とは開集合の補集合になっているものと定義します.
\defx
このようにして任意の集合に対して好きな開集合だけを集めてきて空間の構造を与えられることができるわけです.
\ex
自然数の集合$\N$に対して次のような位相を与えることができる.\\
$(1)$\ $\mathcal{O} = \{\emptyset , \N\}$\\
$(2)$\ $\mathcal{O} = \{U\subset\N | Uは\N の部分集合\}$\\
$(3)$\ $\mathcal{O} = \{ \N \setminus I | I\subset\N は有限部分集合\}$\\
これらの例は全て開集合系の定義を満たしていますので,これらにより$\N$を位相空間とできます.\\
$(1)$を密着位相,$(2)$離散位相.$(3)$を補有限位相と言います.
\exx
ここで収束を位相空間の言葉で書いてみましょう
\defb
位相空間$X$の点列$\{x_n\}$が$x$に収束する
\[
\iff\ \forall U :xを含む開集合 \ \exists N \in \N \ s.t.\ n \ge N \Rightarrow x_n \in U
\]
\defx
こう見てみると$U$とは$\varepsilon-\delta$のように$\varepsilon$とっていることがわかります.
つまり,位相空間とは開集合によって,集合に近いという考え方を与えているわけです.
こう考えてみると,$(1)$の密着位相は全ての点が同じ開集合に属しており,近いところにいるという意味で"密着"しています.\\
$(2)$の離散位相は,全ての$2$点は別々の開集合に入れることができるので"離散"しています.\\
ここで,集合に$(1)(3)$の場合の収束について見て見ましょう.
\ex
$x_n = n$という$\N$内の点列を考える.\\
このとき,$(1)(3)$の位相構造において$x_n$は任意の点に収束する.\\
$(1)$の場合.例えば,$1$に収束することを示してみましょう.$1$を含む開集合は$\N$だけですから,
$n \ge 1 \Rightarrow x_n \in \N$が成り立ちます.よって,$x_n$は$1$に収束します.\\
$(3)$の場合.$U$を$1$を含む開集合とします.これは$\{a_1,\cdots,a_n\}$という有限集合の補集合になっています.\\
よってこれらの最大値を$N$とおくと,$n \ge N+1 \Rightarrow  x_n \notin \{a_1,\cdots,a_n\}\ (最大値より大きいので)$が成り立つので,
\[
\iff\ \forall U :1を含む開集合 \ \exists N \in \N \ s.t.\ n \ge N \Rightarrow x_n \in U
\]
が示せました.同様にして,$x_n$は任意の点に収束することがわかります.一方で$(2)$の場合は$U=\{1\}$という開集合に対して$N$が取ってこれないので任意の点に収束しません.
\exx
次のような扱いやすい空間が定義されます.
\defb
$X:$位相空間,$A\subset X$ がコンパクトである.$\iff$
\[
A \subset \cup_{i\in I} U_i \Rightarrow \exists \{i_1,...,i_n\} \subset I s.t. \ A \subset U_{i_1} \cup \cdots \cup U_{i_n}
\]
\defx
\ex
$(1)(3)$はコンパクトである.$(2)$はコンパクトではない.
\exx

位相空間がコンパクトであるという概念については斎藤毅「はじまりはコンパクト」や斎藤毅「集合と位相」に詳しく解説されています.
また,最初に定義した点列コンパクトとコンパクトという概念は一致します.
\defb
$X$がハウスドルフ空間である.$\iff$\\
$x,y \in X , x\neq y \Rightarrow \exists U:x\mbox{の開近傍} ,\exists V:y\mbox{の開近傍} \ s.t. \ U \cap V = \emptyset$
\defx
\ex
$(1)(3)$はハウスドルフではない.$(2)$はハウスドルフである.
\exx
また連結という概念も位相空間の言葉を使って定式化することができます.
\defb
$X$が連結空間である.$\iff$\\
$X$の部分集合で開集合かつ閉集合であるようなものは,$\emptyset$と$X$のみ.
\defx
\ex
$(0,1) \cup (2,3)\subset\R$は連結空間ではない.\\
実際,区間$(0,1)$は開集合でかつ,$(2,3)$という開集合の補集合になっているので閉集合です.\\
これはこの区間が繋がっていないことによって起こる結果です.
\exx
ここで連続写像の概念も極めてシンプルに一般化されます
\defb
$f:X\to Y$が連続写像である.$\iff$\\
$U \subset Y$が開集合ならば$f^{-1}(U) \subset X$は開集合.
\defx
次の定理は,連続写像の性質を表すとともに,最大値最小値の定理と中間値の定理を一般化したものとも言えます.
\thm
$X,Y$を位相空間として,$f:X\to Y$を連続写像とする.\\
$(1)$$A\subset X$がコンパクトならば$f(A)\subset Y$もコンパクトである.\\
$(2)$$A\subset X$が連結ならば$f(A)\subset Y$も連結である.\\
\thmx
\rem
一方で\\
$(3)A\subset X$が開集合ならば$f(A)\subset Y$も開集合である.\\
$(4)A\subset X$が閉集合ならば$f(A)\subset Y$も閉集合である.\\
はどちらも一般には成り立ちません.これらが成り立つ写像をそれぞれ.開写像,閉写像といいます.
\remx
\proof[位相空間論における代数学の基本定理の証明]
この証明は斎藤毅「集合と位相」に載っている証明です.\\
$f:\C \to \C$を多項式が定める写像とすると,\\
正則関数の一般論から$f$は開写像であることが言える.\\
また$\C$の一点コンパクト化である$\C P^1$を考えることにより,$f$は閉写像であることがわかります.\\
よって,$\C$は連結空間で$f(\C)\subset\C$は開かつ閉であり,空集合でないので$f(\C)=\C$がいえます.\\
つまり,この多項式には$0$点が存在します.
\proofx

