\documentclass[notombow,a4,openany]{kyouritu}
\usepackage{makeidx,multicol}
\usepackage{amsmath ,amssymb ,amsthm}
%プリアンブル
\topmargin -0.5in
\headheight 0.2in
\headsep 0.3in  
%\evensidemargin -0.03in
%\oddsidemargin -0.4in
%\textwidth 5.6in
%\textheight 8.4in
\makeatletter
\renewcommand{\thetable}{%
\arabic{table}}
\makeatother

%圏点
\def\kenten#1{%
\ifvmode\leavevmode\else\hskip\kanjiskip\fi
\setbox1=\hbox to \z@{・\hss}%
\ht1=.63zw
\@kenten#1\end}
\def\@kenten#1{%
\ifx#1\end \let\next=\relax \else
\raise.63zw\copy1\nobreak #1\hskip\kanjiskip\relax
\let\next=\@kenten
\fi\next}
%菊地のプリアンブル
\usepackage{amsmath}
\usepackage{amssymb}
\usepackage{ascmac}
\usepackage[dvipdfmx]{graphicx}


%定理環境とceoを導入
\usepackage{wrapfig}
\usepackage{multicol}
\setlength{\columnseprule}{0.4pt}
%\usepackage{okumacro}
\usepackage[all]{xy}
%\usepackage{theorem}
%\usepackage{ceo,nyushi}

%定理環境
%菊地
\newcommand{\kikuthm}{\underline{\bf 定理}}
\newcommand{\kikuprop}{\underline{\bf 命題}}
\newcommand{\kikueg}{\underline{\bf 例}}
\newcommand{\D}{\partial}

%鯖白
\usepackage{mathrsfs}
\usepackage[all]{xy}
\newcommand{\proofend}{\begin{flushright} $\blacksquare$ \end{flushright}}
%\renewcommand{\labelenumi}{(\roman{enumi})}
\newcommand{\nkgr}{・}

\theoremstyle{definition}
\newtheorem{theorem}{定理}
\renewcommand{\thetheorem}{}
\newtheorem{defi}{定義}
\newtheorem{thm}[defi]{定理}
\newtheorem{lem}[defi]{補題}
\newtheorem{cor}[defi]{系}
\newtheorem{prop}[defi]{命題}
\newtheorem{ex}[defi]{例}

\newtheorem{defm}{定義}
\newtheorem{thmm}[defm]{定理}
\newtheorem{lemm}[defm]{補題}
\newtheorem{corm}[defm]{系}
\newtheorem{propm}[defm]{命題}
\newtheorem{exm}[defm]{例}
%
\theoremstyle{definition}
\newtheorem{defn}{定義}[section]
%
\theoremstyle{remark}
\newtheorem{rem}{注意}
\newtheorem{prf}{証明}

%山本
\usepackage{ascmac}
\usepackage{amsmath}
\usepackage{amssymb}

%笠浦

%田村
\usepackage{wrapfig}

%sp1
\usepackage{here}

%satoken
\newtheorem{sprop}{命題}
\newtheorem{slem}{補題}
\newtheorem{sthm}{定理}

\newcommand{\Image}{\mathop{\mathrm{Im}}\nolimits}
\newcommand{\Ker}{\mathop{\mathrm{Ker}}\nolimits}
\newcommand{\Z}{\mathbb{Z}}

%Section等先頭を大文字にすると番号付けしない.
\newcommand{\Chapter}[1]{\chapter*{{\Huge #1}}
\markboth{#1}{#1}
\addcontentsline{toc}{chapter}{#1}}
\newcommand{\Section}[1]{\section*{{\huge #1}}
\addcontentsline{toc}{section}{#1}}
\newcommand{\Subsection}[1]{\subsection*{\underline{#1}}}
\newcommand{\Subsubsection}[1]{\subsubsection*{#1}}
\setcounter{tocdepth}{0} %Chapterのみ表示する
%インタビュー
%本文
\begin{document}
\mainmatter

\Chapter{追跡!だいあぐらむ ちぇいしんぐ!}
\Section{\S 0.はじまり〜大事なのはカーネルとイメージ〜}
\textbf{集合}とは\underline{ものの集まり}です。\\
例:果物全体の\textbf{集合}:りんごやみかんが入っている。\\
整数全体の\textbf{集合}:$0,1,2,3$とか$-1,-2,-3$とかが入っている。\par
集合$A$に$a$が入っていることを$a \in A$と書く。
  \\
\textbf{写像}とは\underline{集合間の対応}です。\\
例:りんごを$100$円に、みかんを$80$円にする\\ 
りんご $\mapsto 100$ 、 みかん $\mapsto 80$ と書く。\\
写像には$f$や$g$と言った名前をつける。\\
$f \colon a \mapsto b$と書いて元の対応、$f \colon A\rightarrow B$とかいて集合の対応。\\
$f$で$a$を送った先を$f(a)$とかく。\par
   \\
整数全体の集合を$\mathbb{Z}$と書いて、その写像を見てみましょう。\\
(1) $a: n \mapsto n $  つまり $ 1 \mapsto 1 , 2 \mapsto 2 , 3 \mapsto 3 $ \\
(2) $b: n \mapsto 2n $  つまり $ 1 \mapsto 2 , 2 \mapsto 4 , 3 \mapsto 6 $ \\
(3) $c: n \mapsto n-2 $ つまり $ 1 \mapsto -1 , 2 \mapsto 0 , 3 \mapsto 1 $ \\
(4) $d: n \mapsto n^2 $ つまり $ 1 \mapsto 1 , 2 \mapsto 4 , 3 \mapsto 9 $ \\
   \\
写像$f$によって$0$に送られるもの全部を $\Ker f$(かーねる えふ)と書く。\\
写像$f$によって送られてきたもの全部を $\Image f$(いめーじ えふ)と書く。\\
   \\
(1)$\Image a = \Z  ,  \Ker a = \{0\}$\\
(2)$\Image b = 2\Z = \{… , -4 , -2 , 0 , 2 , 4 , 6 , …\} =偶数  ,  \Ker b = \{0\}$\\
(3)$\Image c = \Z  ,  \Ker c = \{2\}$\\
(4)$\Image d = \{0,1,4,9,16,…\} , \Ker d = \{0\}$\\
$f:A\rightarrow B$について、$f$が\textbf{全射}とは、$\Image f = B$が成り立つこと。\\
つまり、\\すべての$b \in B$に対して、毎回 $f(a) = b $となってくれる$a \in A$がいる。\\
$f$が\textbf{単射}とは、$ a \neq a’$ならば$f(a) \neq f(a’)$が成り立つこと。\\
つまり、\\別々のものは、全部別々のものに移るということ。\\ 
     \\
(1)$a$は全射かつ単射。\\
(2)$b$は全射ではないが単射。\\
(3)$c$は全射かつ単射。\\
(4)$d$は全射でも単射でもない。\\

$f$が\textbf{準同型}とは、$f(a+b) = f(a) + f(b)$が成り立っていること。\\
$a,b$は準同型だが、$c,d$は準同型でない。
\Section{\S 1.そしてダイアグラムへ}
これからは、写像といえば準同型を表すものとする。
集合をならべて、その間を準同型で結んだもの\textbf{図式(ダイアグラム)}という。例えば、
\[
\xymatrix
{
A \ar[d]^{b} \ar[r]^{c} 	& C \ar[d]^{d_2}\\
B \ar[r]^{d_1}			& D
}
\]
のようなものが図式。\\
図式をどのようにたどっても同じ所にたどり着くとき、\textbf{可換}であるという。\\
また、横一列に並んだ図式を\textbf{系列}という。\\
\[
\xymatrix{
A \ar[r]^{f} & B \ar[r]^{g} &C  \\
}
\]\\
上の系列で、$B$において\textbf{完全}であるとは、$\Image f = \Ker g$となっていること。\\
\textbf{ダイアグラムチェイシングとは、ある図式が完全であることやある写像が全射・単射であることを示すことである。}\\
完全性を示すには、$\Image f$が$\Ker g$に含まれていることと、$\Ker g$が$\Image f$に含まれていることを示す。\\
$\Image f \subset \Ker g$はたやすい。何故ならば、$ g \circ f = 0$を示せば良いからである。\\
一方 $\Image f \supset \Ker g$は面倒なことがある。\\
\Section{\S 2.ダイアグラムチェイシング(入門編)}
(1) \[
\xymatrix{
0 \ar[r]^{0_i} & A \ar[r]^{f} &B  \\
}
\]\\
が完全であるとき、$f$は単射である。\\
(解)\\
$ 0_i \colon 0 \rightarrow A $とは $0$を$0$に送るだけの写像である。よって、$\Image 0_i = 0$。\\
完全なので、$ \Ker f = \Image 0_i = \{0\} $である。\\
これは単射にほかならない。何故ならば、$0$に行くものは$0$唯一つということを示しているからである。\\
$f(a)=f(b)$ならば、$f(a-b) = f(a)-f(b) = 0$ よって$ a-b \in \Ker f$つまり、$a-b=0$よって$a=b$とも言える。

(2) \[
\xymatrix{
A \ar[r]^{g} & B \ar[r]^{0_t} &0  \\
}
\]\\
が完全であるとき、$g$は全射である。\\
(解)\\
$ 0_t \colon B \rightarrow 0 $とは、すべての$B$の元を$0$に送る写像である。よって、$\Ker 0_t = B$。\\
完全なので、$ \Image g = \Ker 0_t = B $である。よって全射。\\
      \\
  \\
(3) \[
\xymatrix{
0 \ar[r]^{f} & A \ar[r]^{g} &0  \\
}
\]\\
が完全であるとき、$A=0$である。\\
(解)\\
いままでの通り、$\Image f = 0$、$\Ker g = A$である。\\
よって、$A = \Ker g = \Image f = 0$

(4) \[
\xymatrix{
0 \ar[r]^{} &2\Z \ar[r]^{i} &\Z \ar[r]^{p} &2 \ar[r]^{} &0  \\
}
\]\\
は完全である。ただし、\\
$i$とは、$2n \mapsto 2n$となるような写像である。\\
$2$とは、\{0,1\}のことである。\\
$p$とは、偶数を$0$に、奇数を$1$に移すような写像である。\\
(解)\\
$i$は単射で、$p$は全射である。\\
$\Ker p = 2\Z , \Image i = 2\Z$より完全である。\\

\Section{\S 3.ダイアグラムチェイシング(中級編)}
\S 2でダイアグラムチェイシングに必要なものは全て揃った。もしわからないところがあったら展示係に聞くなどして完璧に理解して欲しい。\\
(1)
\[
\xymatrix{
0  
\ar[r]{} 
& A  \ar[d]^p \ar[r]^f & B \ar[d]^q &(完全)
\\ & C \ar[r]^g & D}
\]
このとき$q$が単射ならば、$p$も単射である。\\
(解)\\
いま示したいのは、$p$が単射、つまり、$\Ker p = 0$を示したいのである。\\
つまり、$p(x) = 0$ならば、$x=0$を示す。\\
\[
\xymatrix{ 
& x  \ar[d]^p \\
& 0}
\]
このとき上のようにかいて、$x$を$p$で送ると$0$を表す。さて、$0$を$g$で送ってみると、
\[
\xymatrix{ 
&x \ar[d]^p \\
&0 \ar[r]^g &0}
\]
図式の可換性より、
\[
\xymatrix{ 
&x \ar[d]^p \ar[r]^f &f(x) \ar[d]^q\\
&0 \ar[r]^g &0}
\]
ここで、$q$は単射と仮定したので、$q$で送って$0$に行くのは$0$のみである。
\[
\xymatrix{ 
&x \ar[d]^p \ar[r]^f &f(x)=0 \ar[d]^q\\
&0 \ar[r]^g &0}
\]
となり、$f$は単射であるので、$x=0$となった。よって、$p$は単射である。\\
やったぜ!\\
(2)
\[
\xymatrix{
& A  \ar[d]^p \ar[r]^f & B \ar[d]^q 
\\ & C \ar[r]^g & D \ar[r]^{0_t} &0 &(完全)}
\]
このとき$p$が全射ならば、$q$も全射である。\\
(解)\\
示したいことは、$d \in D$に対して、$b \in B$があって、$ q(b)=d $となることである。\\
よって、まず$d \in D$ とする。$d$を右に送ると$0$に行くので$\Ker 0_t =  \Image g$に入ってる。よって、\\
\[
\xymatrix{
& c \ar[r]^g & d \ar[r]^{0_t} &0}
\]
となるような$c$が存在する。$p$は全射であると仮定したので、\\

\[
\xymatrix{
& a \ar[d]^p\\
& c \ar[r]^g & d \ar[r]^{0_t} &0}
\]
となるような$a$が存在する。これを$f$で送ってみると、\\
\[
\xymatrix{
& a \ar[d]^p \ar[r]^f &f(a)\\
& c \ar[r]^g & d \ar[r]^{0_t} &0}
\]
$f(a)$を$q$で送ってみると、図式の可換性より$q(f(a)) = g(p(a)) = d$となる。\\
よって、\\
\[
\xymatrix{
& a \ar[d]^p \ar[r]^f & f(a) \ar[d]^q \\
& c \ar[r]^g & d \ar[r]^{} &0}
\]
となって、$d$に行ってくれる$f(a)$という元が見つかった。やったぜ!\\
  \\
  \\
(3)
\[
\xymatrix{
& A  \ar[d]^p \ar[r]^f & B \ar[d]^q \ar[r]^g & C \ar[d]^r \ar[r]^{0_t} &0 &(完全)
\\ & D \ar[r]^h & E \ar[r]^i & F & &(完全)}
\]
このとき$p$が全射、$q$が単射ならば、$r$も単射である。\\
(解)\\
同様に、$c$を$r$で送った時$0$であれば、$c=0$を示したい。よってまず、
\[
\xymatrix{ 
& c  \ar[d]^r \\
& 0}
\]
次に、$c$を右に送ると、$0$より、$c \in \Ker 0_t = \Image g$よって、
\[
\xymatrix{ 
&b \ar[r]^g & c  \ar[d]^r \ar[r]^{0_t} &0 \\
& & 0}
\]
となるような$b$が存在する。これを$q$で送って、$i$で送ってみると可換性より、
\[
\xymatrix{ 
&b \ar[r]^g \ar[d]^q & c  \ar[d]^r \ar[r]^{0_t} &0 \\
&q(b) \ar[r]^i & 0}
\]
よって、$q(b)$は$\Ker i = \Image h$に入っている事がわかり、
\[
\xymatrix{ 
& &b \ar[r]^g \ar[d]^q & c  \ar[d]^r \ar[r]^{0_t} &0 \\
&d \ar[r]^h &q(b) \ar[r]^i & 0}
\]
となるような、$d$が存在する。ここで、$p$は全射より、
\[
\xymatrix{ 
&a \ar[d]^p &b \ar[r]^g \ar[d]^q & c  \ar[d]^r \ar[r]^{0_t} &0 \\
&d \ar[r]^h &q(b) \ar[r]^i & 0}
\]
となるような$a$が存在する。ここで、$f(a)$について、$q(f(a))=q(b)$かつ$q$は単射より$f(a)=b$よって、
\[
\xymatrix{ 
&a \ar[d]^p \ar[r]^f &b \ar[r]^g \ar[d]^q & c  \ar[d]^r \ar[r]^{0_t} &0 \\
&d \ar[r]^h &q(b) \ar[r]^i & 0}
\]
となった。いま上の行の完全性より、$ c = g(f(a)) = 0 $ がなりたち、$c=0$がわかった。やったぜ!\\
さてこれでダイアグラムチェイシングに必要なテクニックは全てそろった。\\
\Section{\S 4.ダイアグラムチェイシング(上級編)}
e$\pi$isodeの「アーベル圏入門(鯖白(奴隷))」を開いて、補題35・36を自分で示してみよう!



%\Chapter{編集後記}
%\thispagestyle{empty}
\vspace*{10zw}
\vfill

\parindent=0pt
\begin{picture}(110,1)
\setlength{\unitlength}{1truemm}
\put(5,2){\Large\textbf{$e^{\pi i}sode$ Vol.4 !!!volを変える!!! }} 
\thicklines
\put(0,1){\line(2,0){110}}
\thinlines
\put(0,0){\line(2,0){110}}
\end{picture}

\small{2016年11月25日発行!!!変更する!!!}\\
 \normalsize{著 者・・・・・東京大学理学部数学科有志}\\
 \normalsize{発行人・・・・・!!!名前を書く!!!}\\
\begin{picture}(100,1)
\setlength{\unitlength}{1truemm}
\thinlines
\put(0,1){\line(2,0){110}}
\thicklines
\put(0,0){\line(2,0){110}}
\put(0,-5){\small{\copyright  Students at Department of Mathematics,The University of Tokyo 2016 Printed in Japan}}
\end{picture}


\backmatter
\end{document}