\Chapter{まえがき}
本日は「数学科展示 ますらぼ」にご来場いただき誠にありがとうございます.本企画は今年度を持ちまして5年目となります.私達の上の上のそのまた上の学年から始まり,今回先代から私達数学科2016年度進学(現在数学科4年生)が引き継ぎました.受け継いだ「ますらぼ」「$e^{\pi i}sode$(えぴそーど)」の名前の重さに押しつぶされそうになりながらも,先輩方の多大なるご助力のもと,何とか一つの形にすることができました.数学科や「ますらぼ」の名前に泥を塗るようなことになっていないことを祈るばかりです.

数学科の学生は普段はここ本郷キャンパスではなく,駒場キャンパスという少し離れた別の場所で活動しています.他学部と比べて実験や実習のようなものがほとんどないため,みんな1日の多くの時間を数学に没入しながら,日々数学がわかったり,数学がわからなかったりに一喜一憂しています.

ところが,一人一人がどのような数学をやっているかとなると,これは人によってバラバラです.

「数学」というものはよくひっくるめて一緒くたに扱われますし,「数学は本質的には一つなのだ」という考えはごく自然なもののように思えます.しかし実際にはそんなことは無く,数学の世界にも「畑違い」「よその庭」「人には人の乳酸菌」があります.どんな大数学者も,その時代の数学を全て理解したことはいまだかつてありません.思うに,数学は統一的に意識されながらも,決して統一されることは無さそうです.そしてこれはむしろ嬉しいことのように思えます.というのも,これは数学の多種多様な楽しみ方,それも自分だけの楽しみ方を,そっくりそのまま保証してくれるからです.ただ残念なことに,全ての数学に出会うことは人の短い一生ではどうやら不可能そうです.

今回の$e^{\pi i}sode$には,人生では出会うことがむしろ稀な数学がたくさん詰まっています.これはとても私一人のなせるわざではなく,執筆者になってくれた同期達の深くそれでいて個性的な知識の賜です.本企画・冊子が,数学との新しい出会いのきっかけになっていただけたのならば,これ以上に嬉しいことはありません.是非1冊お手に取ってみてください.
(高木)
